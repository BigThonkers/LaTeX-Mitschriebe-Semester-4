\setcounter{chapter}{9}

\chapter{Pulsform und Signalübertragung}

\section{Signale}

Unterscheiden
\begin{enumerate}[a)]
	\item analog:
	\begin{itemize}
		\item kontinuierlich
		\item beliebige Amplitude
		\item wechselnde Spannungs- und Stromspannungspegel
		\item Pulshöhe und Form enthalten Information
	\end{itemize}
	\item digital: (Spezialform eines Analogsignals)
	\begin{itemize}
		\item diskrete Amplituden
		\item definierte Regel
		\item Info im Zeitpunkt und Abfolge der Signale
	\end{itemize}
	\item unipolar (Gaus Peak oder einzelner Peak von Rechteck Signal)
	\item bipolar (Teil einer Sinus Schwingung Oder Rechteckspannung mit pos. und neg. Anteilen)
\end{enumerate}

%T1
\hft

\noindent
Signalbreite $ \widehat{=} $ Auge\\
Bei Ursachen aus Kern- und Teilchenphysik\\
Signalbreite 5 ns bis 100 $ \mu $s, steigende Flanke 2 ns bis 20 ns, fallende Flanke 10 ns bis 100 $ \mu $s

\section{Fourierzerlegung}

Jedes zeitkonstante Signal $ f(t) $ mit Periodenlänge $ T $ und Grundfrequenz $ f_0 = \frac{1}{T} $ ($ \Rightarrow $ Kreisfrequenz $ \omega_0 = 2 \pi f_0 $) kann angenähert werden durch
\begin{equation*}
f'(t) =  \sum_{m=0}^{M-1} g_m(\omega_0) \phi_m(t)
\end{equation*}
mit $ M $ Basisfunktionen $ \phi_m $ und $ M $ Amplituden $ g(\omega_0) $\\
$ \Rightarrow $ Fourierreihenentwicklung
\begin{equation*}
f(t) = \sum_{k=-\infty}^{\infty} g_k(\omega_0) e^{ik\omega_o t}
\end{equation*}
mit komplexen Fourierkoeffizienten
\begin{equation*}
g_k = \frac{\omega_0}{2 \pi} \int_{0}^{T} f(t) e^{-i\omega_0 k t} \dd t
\end{equation*}
d.h. \textbf{alle} Frequenzen tragen zum Signal bei. Typisches Bild bei Signalübertragung:

%T2
\hft

\noindent
\textbf{Bandbreite:} Def: Abfall eines Signals um $ 3 \tx{dB} $: $ 10 \cdot \log \frac{P}{P_0} = - 3 \tx{ dB} $ mit $ \frac{P}{P_0} = \frac{1}{2} $

\section{Signalübertragung}

%T3
\hft

\noindent
Betrachtung für unendlichlanges Kabel\\[5pt]
Pegelsprung:
\begin{align*}
\Delta U (z,t) &= - R \Delta z I(z,t) = - L \Delta z \prt{I}{t}(z,t) \\
\Delta I (z,t) &= - G \Delta z U(z,t) = - C \Delta z U(z,t)
\end{align*}
$ \Rightarrow $ \textbf{Telegraphengleichung}
\begin{equation*}
\prt{^2 U }{z^2} - LC \prt{^2 U}{t^2} = (LG + RC) \prt{U}{t} + RGU
\end{equation*}
Lösung:
\begin{equation*}
U(z,t) = U_0 e^{i \omega t - \gamma} \quad \tx{mit} \quad \gamma = \pm \sqrt{(R+i\omega L)(G + i \omega C)}
\end{equation*}
auf unendlichlanger Leitung z.B. für Koaxialkabel\\[5pt]
Charakteristische Impedanz
\begin{equation*}
Z_0 = \frac{U}{I} = \sqrt{\frac{L}{c}}
\end{equation*}
für $ R, G = 0 $ d.h. verlustfreies Kabel.\\[10pt]
Im Praktikum Koaxialkabel $ Z_0 = 50 \Omega $\\[5pt]
Verlustbehaftete Kabel
\begin{equation*}
Z = \frac{U}{I} = \frac{R + i \omega L}{\gamma^2}
\end{equation*}
\begin{equation*}
c = \frac{1}{\sqrt{LC}} = \frac{1}{\sqrt{\epsilon_\mu \epsilon_0 \mu}} = \frac{c}{\sqrt{\epsilon}}
\end{equation*}

\subsection{Impedanzanpassung}

\begin{enumerate}[a)]
	\item %T4
	\hft
	
	\noindent
	\begin{equation*}
	Z_1 \overset{!}{=} \frac{R \cdot Z_2}{R + Z_2} \quad \Rightarrow \quad R = \frac{Z_1 Z_2}{Z_2 - Z_1}
	\end{equation*}
	(Parallelschaltung)
	\item %T5
	\hft
	
	\noindent
	\begin{equation*}
	Z_1 \overset{!}{=} R + Z_2 \quad \Rightarrow \quad R = Z_1 - Z_2
	\end{equation*}
\end{enumerate}

\section{Analoge Signalfilter}

\begin{enumerate}[a)]
	\item CR-RC Pulsformung
	
	%T6
	\hft
	
	\noindent
	Hochpass: Unterdrückt Frequenz
	\begin{equation*}
	f \le \frac{1}{2 \pi R \cdot C}
	\end{equation*}
	differenziert Eingangssignal wenn $ \tau \ll $ Pulslänge am Eingang
	
	%T7
	\hft
	
	\noindent
	Tiefpass: Unterdrückt Frequenzen
	\begin{equation*}
	f \ge \frac{1}{2 \pi R \cdot C}
	\end{equation*}
	integriert Eingangssignal\\[5pt]
	Für $ \tau \gg $ Pulslänge am Eingang\\[10pt]
	Anwendung von Filtern von Rauschen:
	
	%T7
	\hft
	
	\noindent
	\color{red} Nachteil: Ratenabhängige Nullpunktverschiebung! \color{black}\\[10pt]
	Beste Signalumformung: $ \tau = R_D C_D = R_1 C_1 $ $ \Rightarrow $ Signal/Rauschabstand $ \tau_{Diff} \approx \tau_{Int} $. möglicher Ausweg für Verschiebung der Nulllinie: Nachschalten einer weiteren Diff-Stufe $ \Rightarrow $ Resultat bipolares Signal.
\end{enumerate}