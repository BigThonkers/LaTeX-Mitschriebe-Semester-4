\chapter{Detektoren für die Orts- und Zeitmessung}

Programm Heute:
\begin{itemize}
	\item Ionisationsdetektoren
	\item Szintillation
	\item Photomultiplier (PM)
\end{itemize}
,,Rekation`` im Material auf elektrische geladenes Teilchen oder Quanten
\begin{equation*}
\tx{Ionisation durch Projektil} \begin{array}{c}
\nearrow \\ \searrow
\end{array} \begin{array}{c c l}
\tx{freie Ladungsträger} & \rightarrow & \tx{Nachweis } e^- \tx{ / Ionen} \\[10pt]
\tx{Szintillation / Fluoreszens} & \rightarrow & \tx{Nachweis Licht}
\end{array}
\end{equation*}

\section{Ionisationsdetektoren}

\begin{itemize}
	\item mit flüssigem oder gasförmigen Edelgas $ + $ Beimischungen als ,,Quentscher``
	\item Halbleiter
\end{itemize}
\begin{equation*}
\tx{beide messen} \begin{array}{c}
\nearrow \\[5pt] \searrow
\end{array} \begin{array}{c}
\begin{array}{c}
\tx{nur primär erzeugte } e^- \\ \tx{(Halbleiter, Ionisationskammer)}
\end{array} \\[20pt] \begin{array}{c}
\tx{primäre $ e^- $ $ + $ Influenz von driftenden Ionen} \\ \tx{(Prop Zähler)}
\end{array}
\end{array}
\end{equation*}



%T1
\hft Strohalmdetektor


\noindent
Elektrisches Feld aus statischen Maxwellgleichungen:
\begin{align*}
\vabla \cdot \vec{E} &= \frac{1}{\epsilon_0} \rho \\
\int_{S} \vec{E} \dd \vec{S} &= \frac{1}{\epsilon_0} \int_{V} \rho \dd V
\end{align*}
auf Drahtlänge $ \Delta z $ befindet sich die Ladung $ \Delta Q $
\begin{equation*}
E(r) \cdot 2 \pi r \Delta z = \frac{1}{\epsilon_0} \Delta Q
\end{equation*}
\begin{equation*}
\Rightarrow \quad E(r) = \frac{1}{2 \pi \epsilon_0} \frac{1}{r} \prd{Q}{z}
\end{equation*}
Das $ \vec{E} $-Feld wird durch angelegte Spannung erzeugt. Aus der Abbildung %ref T1
oben folgt dann:
\begin{equation*}
\int_{r_i}^{r_a} E(r) \dd r = U = \frac{1}{2 \pi \epsilon_0} \cdot \ln\left(\frac{r_a}{r_i}\right) \cdot \prd{Q}{z}
\end{equation*}
\begin{equation*}
E(r_0) = \frac{U}{r_0 \ln \left(\frac{r_a}{r_i}\right)}
\end{equation*}
\folie{Arbeitsbereiche von Gasionisationsdetektoren}\\
\folie{Funktionsprinzip Gasionisationsdetektoren}\\
Ionisationszähler $ \rightarrow $Dosimetrie / Dosimeter\\
Proportionalitätsbereich $ \rightarrow $ Teilchennachweis (Ort und Zeit)\\
Geiger Müller Zähler ist selbst verstärkend $ \rightarrow $ keine extra Geräte notwendig\\[5pt]
Elektronen \& Ionen, die im Abstand $ r_0 $ erzeugt werden driften zur Anode/Kathode.\\
z:.B.
\begin{equation*}
\Delta t^- = \int_{r_i}^{r_a} \frac{\dd r}{\theta_0^-} = \int_{r_i}^{r_a} \frac{\dd r}{\mu^- \cdot E} = \frac{\ln \left(\frac{r_a}{r_i}\right)}{2 \mu^- E}
\end{equation*}
\folie{Eigenschaften von Edelgasen}\\
\folie{Vieldrahtproportional- und Driftkammern}\\
\folie{MICROMEGAS}\\
\folie{GEM / THGEM}

\section{Halbleiterzähler}

Kristallines Si \& Ge. Ideoal für $ \prd{E}{x} $ hochauflösende Ortsmessung\\[5pt]

\subsection{Funktionsprinzip:}

\begin{itemize}
	\item Diode in Sperrrichtung
	\item ionisierende Strahlung erzeugt $ e^- $/Loch Paare
	\item äußere Betriebsspannung saugt $ e^- $/Löcher ab
\end{itemize}
\textbf{Vorteile:}
\begin{enumerate}[a)]
	\item $ \langle E \rangle $ zur Erzeugung eines $ e^- $/Loch Paars $ \langle E \rangle_{\tx{Si}} = 3{,}6 \, \tx{eV} $ und $ \langle E \rangle_{\tx{Ge}} = 2{,}8 \, \tx{eV} $.\par
	Zum Vergleich $ \langle E \rangle _{\tx{Gas}} \approx 10{.}40 \, \tx{eV} $ und $ \langle E \rangle_{\tx{Szint}} = 100 \, \tx{eV} \tx{-} 1 \, \tx{keV} $
	\item hohe spezifische Dichte $ \Rightarrow \prd{E}{x} $ groß
	\item sehr schnelle \hfw
	\item kompakte \hfw
\end{enumerate}

\subsection{Grundlagen}

Festkörper:


%T2
\hft



\noindent
Direkte Rekombination $ \mathcal{O}(s) $ weil $ e^- $ \& Loch Energie- und Impulserhaltung.\\
\folie{Funktionsprinzip (Halbleiter)}\\
\folie{Funktionsprinzip: Streifenzähler}\\
\folie{Ultrasonic Bonding}\\
\folie{ATLAS Silizium Spurdetektor}\\
\folie{Silizium Detektoren als Spur Detektor (CMS: Currently the Most Silicon)}\\
\folie{Halbleiter-Pixelzähler}\\
\folie{Zukunft: 3D-Technologie}

\section{Szintillationsdetektor}

$ \prd{E}{x} \to $ Anregung der Atome/Moleküle $ \to $ Lichtemission $ \propto \prd{E}{x} $.\\[5pt]
Wichtig dabei ist die Transparenz des Detektors für das erzeugte Licht. Vorteilhaft ist deswegen, wenn die Spektralemission im sichtbaren Bereich ist.\\[5pt]
Typen:
\begin{itemize}
	\item organische Kristalle, Flüssigkeiten oder Plastik
	\item anorganische Kristalle
	\item flüssige, gasförmige Edelgase
\end{itemize}

\subsection{Funktionsprinzip}

\begin{enumerate}[a)]
	\item Anorganisch
	\begin{itemize}
		\item Dotieren mit Farbzentren (Aktivatorzentren) (Leerstellen im Gitter)
		\item Ionisation führt zu freien $ e^- $
		\item $ \Rightarrow $ Rekombination in Aktivatorzentren $ \to $ Anregung selbige $ \to $ Übergang in Grundzustand unter \hfw $ \left. \right\} \mathcal{O}(\mu s) $
	\end{itemize}
	\item Organisch
	\begin{itemize}
		\item Ionisation \& Anregung von Molekülen
		\item $ \to $ emittiert beim Zerfall UV- Licht + Wellenlängenverschiebung $ \Rightarrow $ sichtbares Licht
	\end{itemize}
\end{enumerate}
\folie{Szintillatoren}\\
\folie{Einsatzprinzip}\\
\folie{Emission}\\
\folie{Organische Szintillatoren - Licht Absorption}

\section{Photomultiplier}

\folie{Photomultiplier}\\
\folie{Quanteneffizienz}\\
\folie{PMT und Szintillator Handhabung}