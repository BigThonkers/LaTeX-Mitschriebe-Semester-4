\chapter{Dynamik}

\textbf{Bisher:} statische Struktur des Zustandsraums der Quantenmechanik.\\[5pt]
\textbf{Jetzt:} Parametrisierung in der zeit (zunächste aber \textbf{ohne} Begründung der ,,Richtung`` der Zeit ($ \to $ Ausblick: Dekohärenztheorie als ein aktueller Kandidat für den Ursprung des Zeitpfeils.)).

\section{Die Schrödinger-Gleichung}

Forderungen an die zeitliche Entwicklung eines Zustandes $ |\psi\rangle: $ Stetigkeit, d.h.
\begin{equation}
\lim_{t\to t_0} |\psi(t)\rangle = |\psi(t_0)\rangle = |\psi_0\rangle \equiv \tx{Anfangszustand}
\label{3.1}
\end{equation}
wobei $ t > t_0 $.\\
Weiter werde der Zustand $ |\psi(t)\rangle $ durch Operation des \textbf{Zeitentwicklunngsoperators} $ U(t,t_0) $ auf $ |\psi_0\rangle $ erzeugt, d.h.
\begin{equation}
|\psi(t) \rangle = U(t,t_0) |\psi_0\rangle
\label{3.2}
\end{equation}
Im Sinne der Interpretation von $ |\psi\rangle $ also Wahrscheinlichkeitsamplitude für Messergebnisse fordern wir die Normerhaltung im Laufe der Zeit, was die Unitarität von $ U(t,t_0) $ impliziert (daher auch Sprechweise $ U \equiv $ ''the unitary``).
\begin{equation}
U^{+}(t,t_0) U(t,t_0) = \mathbbm{1} \quad \tx{bzw.} \quad U^{+}(t,t_0) = U^{-1}(t,t_0)
\label{3.3}
\end{equation}
Dies garantiert:
\begin{equation}
||\, |\psi(t)\rangle || = 1 \ \forall t
\label{3.4}
\end{equation}
Im Sinne einer kontinuierlichen und in einzelnen Zeitintervalle zerlegbaren Entwicklung fordern wir außerdem:
\begin{equation}
U(t,t_0) = U(t_1,t_2) U(t,t_0) \qquad \forall t_2 > t_1 > t_0
\label{3.5}
\end{equation}
Mit der Stetigkeitsbedingung \eqref{3.1} folgt für ein infinitesimales Zeitintervall $ \dd t $:
\begin{equation}
\lim_{\dd t \to 0} U(t_0 + \dd t , t_0) = \mathbbm{1}
\label{3.6}
\end{equation}
Der Ansatz
\begin{equation}
U(t_0 + \dd t, t_0) = \mathbbm{1} - i \Omega \dd t
\label{3.7}
\end{equation}
mit $ \Omega = \Omega^{\dagger} $, erfüllt die Forderungen aus \eqref{3.3}, \eqref{3.5} und \eqref{3.6} - unter der Voraussetzung, dass Terme der Ordnung $ \dd t^2 $ vernachlässigbar sind.

%27.05.19

\begin{equation*}
\mathbbm{1} - i \Omega \dd t + i \Omega \dd t + \cancel{c \dd t^2}
\end{equation*}
$ \Omega $ in Gleichung \eqref{3.7} hat die Dimension $ [S^{-1}] $, was, zusammen mit der Rolle der klassischen Hamiltonfuntion als Erzeuger der Phasenraumdynamik (siehe insbesondere Liouville-Satz bzw. Liouville-Gleichung, die die Zeitentwicklung der Phasenraumvolumens beschreibt. $ \sim i \prt{f}{t} = L f $) und
\begin{equation}
E \sim \hbar \omega
\label{3.8}
\end{equation}
(nach Planck) zu der Identifikation:
\begin{equation}
\Omega \defeq \frac{H}{\hbar}\
\label{3.9}
\end{equation}
mit $ H \equiv $ Hamiltonfunktion, führt. In der QM bezeichnen wir $ H $ als den \textbf{Hamilton-Operator}, der für autonome (d.h. Zeitunabhängige) $ H $ mit dem Energieoperator identifiziert wird.\par
Wir leiten nun die Bewegungsgleichungen für $ U $ her, um über beliebige Zeitintervalle propagieren zu können. Propagieren $ \leftrightarrow $ $ U = $ Propagator (siehe HöMa, Cauchy Integrale und Zusammenhang Propagator und Greensfunktionen).
\begin{align*}
U(t + \dd t, t_0) \overset{\eqref{3.5}}{=} U(t + \dd t, t) U(t, t_0) \overset{\substack{\eqref{3.7} \\ \eqref{3.9}}}{=} \left(\mathbbm{1} - i \frac{H}{\hbar} \dd t\right) U(t,t_0)
\end{align*}
\begin{equation*}
U(t + \dd t, t_0) - U(t, t_0) = - i \frac{H}{\hbar} \dd t U(t, t_0)
\end{equation*}
woraus der Differenzquotient
\begin{equation*}
\lim_{\dd t \to 0} \frac{U(t + \dd t, t_0) - U(t,t_0)}{\dd t} = \prt{U}{t} = - i \frac{H}{\hbar} U(t, t_0)
\end{equation*}
\begin{equation}
\rmbox{ i \hbar \prt{U(t,t_0)}{t} = H U(t,t_0) }
\label{3.10}
\end{equation}
Durch Multiplikation von rechts mit $ |\psi_0\rangle $ erhalten wir die Schrödingergleichung:
\frbox{Schrödingergleichung}{\begin{equation}
i \hbar \prt{}{t} | \psi(t) \rangle = H | \psi(t) \rangle
\label{3.11}
\end{equation}}
\noindent
welche $ \forall t $ gültig ist.

\subsection{Heisenberg- vs. Schrödinger-Bild}

Die Gleichung \eqref{3.2} liefert die zeitliche Entwicklung des Zustandsvektors
$$ |\psi(t)\rangle \custo{\rightarrow}{\sim}{\mathclap{\tx{Kap.I}}} c_0(t) | 0 \rangle + c_1(t) | 1 \rangle $$
d.h. der Zustand $ |\psi(t)\rangle $ bewegt sich bezüglich eines festen Koordinatensystems. Völlig gleichberechtigt ist die Perspektive, wonach der Zustandsvektor zeitlich unveränderlich ist, sich jedoch das Koordinatensystem bewegt oder dreht. Während nach \eqref{3.2} also die Zustände die Zeitabhängigkeit tragen, wird in der alternativen Sichtweise die Zeitabhängigkeit vollständig auf die Observablen abgewälzt (die Observablen helfen ja über ihre Eigenvektoren die Vektorraumbasis geliefert).
\begin{figure}[ht]
	\centering
	%t1
	\begin{tikzpicture}
		\draw[->] (0,0) -- (4,0) node[anchor=north west] {$ |0\rangle $};
		\draw[->] (0,0) -- (0,4) node[anchor=south east] {$ |1\rangle $};
		\draw[->,blue] (0,0) -- (30:4) node[right] {$ |\psi(t=0)\rangle $};
		\draw[->,red] (0,0) -- (60:4) node[right] {$ |\psi(t>0)\rangle $};
		\centerarc[->](0,0)(30:60:2.5);
		\node at (45:3) {\eqref{3.2}};
		\node at (2,-2) {\tx{\textbf{Schrödinger-Bild}}};
		% 2tes
		\coordinate (o) at (7,0);
		\draw[->,blue] (o) -- ++(90:4) node[anchor=south east] {$ |1\rangle $};
		\draw[->,blue] (o) -- ++( 0:4) node[right] {$ |0\rangle $};
		\draw[->,red] (o) -- ++(70:4) node[right] {$ |1\rangle' $};
		\draw[->,red] (o) -- ++(-20:4) node[right] {$ |0\rangle' $};
		\draw[->] (o) -- ++(35:4) node[right] {$ |\psi(t=0)\rangle = |\psi_0\rangle $};
		\node at ($ (o) + (2,-2) $) {\tx{\textbf{Heisenberg-Bild}}};
	\end{tikzpicture}
	\caption{Verschiedene Modelle zur Vorstellung einer Zeitabhängigen Wellenfunktion von Schrödinger und Heisenberg.}
	\label{SHBild}
\end{figure}
\begin{equation}
|\psi(t)\rangle _{\tx{Schrödinger}} \overset{\eqref{3.2}}{=} U(t,t_0) |\psi_0\rangle \quad \curvearrowright \quad |\psi_0\rangle \eqdef |\psi \rangle_{\tx{Heisenberg}} = U^{\dagger}(t,t_0) |\psi(t)\rangle_{\tx{Schrödinger}}
\label{3.12}
\end{equation}
Erwartungswerte von Observablen sollten nicht von der Wahl des Bildes abhängen. Daher:
\begin{equation}
\begin{aligned}
_{\tx{Schrödinger}} \left\langle \psi(t) \right| A_{\tx{Schrödinger}} \left| \psi(t) \right\rangle_{\tx{Schrödinger}} &\overset{\eqref{3.2}}{=} \ub{\left\langle \psi_0 \right| }_{\mathclap{\tx{Heisenberg} \langle\psi |}} U^{\dagger} (t,t_0) A_{\tx{Schrödinger}} U(t,t_0) \left| \psi_0 \right\rangle\\
&\overset{\eqref{3.12}}{=} _{\tx{Heisenberg}} \langle\psi| A_{\tx{Heisenberg}} (t) | \psi \rangle_{\tx{Heisenberg}}
\end{aligned}
\label{3.13}
\end{equation}
mit
\begin{equation}
A_{\tx{Heisenberg}}(t) = U^{\dagger}(t,t_0) A_{\tx{Schrödinger}} U(t,t_0)
\label{3.14}
\end{equation}
\noindent
Anstelle der Schrödingergleichung \eqref{3.11} benötigen wir Entwicklungsgleichung für $ A_{\tx{Heisenberg}} $ die sich aus \eqref{3.10} und \eqref{3.14} gewinnen lässt:
\begin{equation*}
i \hbar \prt{}{t} A_{\tx{Heisenberg}} \overset{\eqref{3.14}}{=} i \hbar \left[\prt{U^{\dagger}}{t} A_{\tx{Schrödinger}} U + U^{\dagger} \prt{A_{\tx{Schrödinger}}}{t} U + U^{\dagger} \prt{U}{t}\right]
\end{equation*}
Umformen mit $ i \hbar \prt{U}{t} |\psi_0\rangle = H \ub{U |\psi_0\rangle}_{|\psi(t)\rangle} $
\begin{equation*}
= - U^{\dagger} H_{\tx{Schrödinger}} A_{\tx{Schrödinger}} U + U^{\dagger} A_{\tx{Schrödinger}} H_{\tx{Schrödinger}} U + i \hbar U^{\dagger} \prt{A_{\tx{Schrödinger}}}{t} U
\end{equation*}
\begin{equation}
\Rightarrow \quad i \hbar \prt{}{t} A_{\tx{Heisenberg}} = U^{\dagger} \left[A_{\tx{Schrödinger}}, H_{\tx{Schrödinger}}\right] U + i \hbar U^{\dagger} \prt{A_{\tx{Schrödinger}}}{t} U
\label{3.15}
\end{equation}
Mit $ U U^{\dagger} \overset{\eqref{3.3}}{=} \mathbbm{1} $ und $ A_{\tx{Heisenberg}} \overset{\eqref{3.14}}{=} U^{\dagger} A_{\tx{Schrödinger}} U $ wird \eqref{3.15} zu:
\frbox{Äquivalente dynamik im Heisenberg Bild zu SGL}{\begin{equation}
i \hbar \prt{}{t} A_{\tx{Heisenberg}} = \left[A_{\tx{Heisenberg}}, H_{\tx{Heisenberg}}\right] + i \hbar \left(\prt{A_{\tx{Schrödinger}}}{t}\right)_{\tx{Heisenberg}}
\label{3.16}
\end{equation}}
\noindent
\textbf{äquivalente} Dynamik zur Schrödinger-Gleichung \eqref{3.11} im Heisenberg Bild. Die direkte Analogie zur klassischen Hamilton'schen Dynamik in den Übungen.

\subsection{Das Wechselwirkungsbild}

$ \to $ Eine zwischen Schrödinger- und Heisenberg-Bild ,,interpolierende`` Perspektive, die eine Zerlegung des Hamiltonoperators voraussetzt, wobei $ H_0 $ die ,,triviale`` Dynamik (ungestört und bekannt!) erzeugt, $ V $ die Abweichung oder Störung davon induziert.\\[10pt]
Ergänzung zum letzten Mal: Die Bewegungsgleichung für die durch \eqref{2 66} und \eqref{2 67} definierten Dichteoperatoren (analog zur Herleitung) von \eqref{3.16} $ \to $ Übungen.
\begin{equation}
i \hbar \dot{\rho}_{\tx{Schrödinger}} = \left[H_{\tx{Schrödinger}}, \rho_{\tx{Schrödinger}}\right]
\label{3.17}
\end{equation}
Beachte das andere Vorzeichen als in \eqref{3.16} und \eqref{3.24}.
\begin{equation}
H = H_0 + V \footnote{Kato: Pertubation theory of linear Operators.}
\label{3.18}
\end{equation}
Das \textbf{Wechselwirkungsbild} wälzt die $ H_0 $-Dynamik auf die Observable ab, die $ V $-Dynamik auf den Zustand ($ \to $ rotating wave approximation).\par
Zustandsdynamik:
\begin{equation}
\begin{aligned}
|\psi(t) \rangle_{\tx{Wechselwirkung}} &= U_{\tx{Wechsewirkung}} |\psi\rangle_{\tx{Heisenberg}} \\
&= U_{\tx{Wechselwirkend}} |\psi_{0}\rangle
\end{aligned}
\label{3.19}
\end{equation}
und im Schrödinger-Bild:
\begin{equation}
|\psi(t)\rangle_{\tx{Schrödinger}} = U_0(t,t_0) U_{\tx{Wechselwirkung}} |\psi_{0}\rangle
\label{3.20}
\end{equation}
beziehungsweise:
\begin{equation}
U(t,t_0) = U_0(t,t_0) U_{\tx{Wechselwirkend}} (t,t_0)
\label{3.21}
\end{equation}
wobei $ U_{\tx{Wechselwirkung}} $ von $ V $, $ U_0 $ von $ H_0 $ erzeugt wird
\begin{equation*}
\begin{aligned}
i \hbar \prt{}{t} U(t,t_0) &\overset{\eqref{3.21}}{=} i \hbar \prt{}{t} (U_0 U_{\tx{WW}}) = i \hbar \left[\prt{U_0}{t} U_{\tx{WW}} + U_0 \prt{U_{\tx{WW}}}{t}\right] \\
&\overset{\eqref{3.10}}{=} \rmbox{H_0 U_0 U_{\tx{WW}} + i \hbar U_0 \prt{U_{\tx{WW}}}{t}} \\
&\overset{\substack{\eqref{3.10} \\ \eqref{3.18}}}{=} i \hbar (H_0 + V) U \overset{\eqref{3.21}}{=} \rmbox{H_0 U_0 U_{\tx{WW}} + V U_0 U_{\tx{WW}}}
\end{aligned}
%\label{3.22}
\end{equation*}
d.h.:
\begin{equation*}
i \hbar U_0 \prt{U_{\tx{WW}}}{t} = V U_0 U_{\tx{WW}}
\end{equation*}
\begin{equation}
\begin{aligned}
\overset{U_{0}^{\dagger} \cdot | \eqref{3.22}}{\Rightarrow} \quad i \hbar \prt{U_{\tx{WW}}}{t} &= U_0^{\dagger} V U_0 U_{\tx{WW}} \\
&\defeq V_{\tx{WW}} U_{\tx{WW}}
\end{aligned}
\label{3.22}
\end{equation}

%28.05.19

\noindent
Wiederholung:
\begin{align*}
|\psi(t) \rangle_{\tx{WW}} &\overset{\eqref{3.19}}{=} U_{\tx{WW}}(t,t_0) |\psi\rangle_{H} \\
&\overset{\eqref{3.20}}{=} U_{0}^{\dagger} (t,t_0) |\psi(t) \rangle _{S}
\end{align*}
\begin{equation*}
|\psi(t) \rangle _{\tx{WW}} = |\psi\rangle_{H} = |\psi_{0}\rangle \overset{\substack{\eqref{3.12} \\ \eqref{3.18}}}{=} U_{0}^{\dagger}(t,t_0) |\psi(t) \rangle_{S}
\end{equation*}
\begin{equation*}
U(t,t_0) \overset{\eqref{3.21}}{=} U_0(t,t_0) U_{\tx{WW}}(t,t_0)
\end{equation*}
Es folgt dann \eqref{3.23}:
\begin{equation*}
\begin{aligned}
i \hbar \prt{U_{\tx{WW}}}{t} &= U_0^{\dagger} V U_0 U_{\tx{WW}} \\
&\defeq V_{\tx{WW}} U_{\tx{WW}}
\end{aligned}
\end{equation*}
\frbox{Schrdingergleichung im Wechselwirkungs-Bild}{
\begin{equation}
\overset{\eqref{3.19}}{\rightarrow} \quad i \hbar \prt{}{t} |\psi(t)\rangle_{\tx{WW}} = V_{\tx{WW}} |\psi(t)\rangle_{\tx{WW}}
\label{3.23}
\end{equation}}
\noindent
Wiederum aus der Unabhängigkeit der Erwartungswerte von Observablen von dem gewählten Bild folgt für die Bewegungsgleichung der Observablen im WW-Bild (siehe: \eqref{3.22}):
\begin{align*}
&_{S}\langle \psi(t)|A_S|\psi(t)\rangle_{S} \overset{\eqref{3.20}}{=} \langle \psi_0| U_{\tx{WW}}^{\dagger} U_{0}^{\dagger} A_{S} U_0 U_{\tx{WW}} | \psi_0\rangle \\
&= _{\tx{WW}} \langle \psi(t) | U_{0}^{\dagger} A_S U_0 |\psi(t)\rangle _{\tx{WW}} \overset{\substack{\eqref{3.22} \\ \eqref{3.14}}}{=} _{\tx{WW}} \langle \psi(t) | A_{\tx{WW}} | \psi(t) \rangle_{\tx{WW}}
\end{align*}
\begin{equation}
\begin{aligned}
i\hbar \prt{}{t} A_{\tx{WW}} &= \overset{\tx{s.a. Umgebung von \eqref{3.14}}}{\dots \dots \dots} = - U_{0}^{\dagger} H_0 \color{red} U_0 U_0^{\dagger} \color{black} A_S U_0 + U_0^{\dagger} A_S \color{red} U_0 U_0^{\dagger} \color{black} H_0 U_0 + i \hbar U_0^{\dagger} \prt{A_S}{t} U_0 \\
&= - H_{0,\tx{WW}} A_{\tx{WW}} + A_{\tx{WW}} H_{0,\tx{WW}} + i \hbar \left(\prt{A_S}{t}\right) _{\tx{WW}} \\
&= \left[A_{\tx{WW}} , H_{0,\tx{WW}}\right] + i \hbar \left(\prt{A_S}{t}\right)_{\tx{WW}}
\end{aligned}
\label{3.24}
\end{equation}
mit:
\begin{equation}
A_{\tx{WW}} = U_0^{\dagger} A_{\tx{Schrödinger}} U_0
\label{3.25}
\end{equation}

\section{Dynamik eines Zweiniveausystems unter dem Einfluss einer Störung}

Häufig anzutreffendes Szenario in aktueller AMo (Atom-,Molecule and Quantumoptics) | CondMat (Condesed Matter) - Experimentalphysik: ,,effektive`` Zwei-Niveau-Systeme ($ \equiv H_0 $) unter dem Einfluss (häufig oszillierender (think of lasers)) Felder
\begin{equation*}
\tx{spin up/down:} \quad |\uparrow \rangle \quad |\downarrow\rangle \quad \tx{NMR} \qquad \qquad |0\rangle \quad |1\rangle \quad \tx{Qbits}
\end{equation*}


%T1
\hft Cond Mat oder Zweiniveausystem und deren Bra-Ket Symbol Schreibweise


\noindent
Der Hamiltonoperator der Zweiniveausystems ist definiert durch die Eigenwertgleichung
\begin{equation}
H_0 |j\rangle = E_j |j\rangle \qquad j = 0,1
\label{3.26}
\end{equation}
Die Störung $ V $ von $ H_0 $ sei gegeben als ,,Dipolnäherung``:
\begin{equation}
V = e \vec{r} \vec{F} \cos(\omega t)
\label{3.27}
\end{equation}
mit $ H \overset{\eqref{3.18}}{=} H_0 + V $.\\
$ e $ ist die Elementarladung des Elektrons, $ \vec{r} $ der Ortsvektor bzgl. des ortsfesten Kerns, $ \vec{F} $ ist die Feldamplitude und $ \omega $ Feldfrequenz.\\
Diese Störung ist klassisch also nicht Quantisiert, der Anteil $ H_0 $ ist quantisiert. Das ganze nennt man die \textbf{semi-klassische} Beschreibung der Licht-Materie-Wechselwirkung.


%T2
\hft mit Effektivem Potential


\noindent
Die Lösung der Schrödinger-Gleichung \eqref{3.11} (im Schrödinger-Bild, wobei wir die Indizes b sofort weglassen) im durch $ |0\rangle $ und $ |1\rangle $ aufgespannten Hilbertraum setzen wir an als
\begin{equation}
|\psi(t)\rangle = c_0(t) e^{-i E_0 t / \hbar} |0\rangle + c_1(t) e^{- i E_1 t / \hbar} |1\rangle
\label{3.28}
\end{equation}
Dies macht Sinn, da die Eigenvektoren von $ H_0 $ die (durch $ H_0 $ induzierte) Zeitentwicklung
\begin{equation}
U_0(t,t_0) |j\rangle = e^{-iE_j (t-t_0) / \hbar} |j\rangle
\label{3.29}
\end{equation}
tragen.\\
Dies stammt von:
\begin{align*}
i \hbar \prt{}{t} |\psi\rangle &= H_0 |\psi \rangle = (E_0 |0\rangle\langle 0| + E_1 |1\rangle\langle 1|)|\psi\rangle = E_0 |0\rangle\langle 0|\psi\rangle + E_1 |1\rangle\langle 1|\psi\rangle \\
&= i \hbar \prt{}{t} (\langle 0|\psi\rangle |0\rangle + \langle 1|\psi\rangle |1\rangle)
\end{align*}
durch Linksmultiplikation von $ \langle 0 | \cdot || $:
\begin{equation*}
i \hbar \prt{}{t} \langle 0|\psi\rangle = E_0 \langle 0 |\psi \rangle
\end{equation*}
Wir setzen $ t_0 \defeq 0 $ und $ |\psi_0\rangle $ als beliebige Linearkombination von $ |0\rangle $ und $ |1\rangle $ mit $ c_0(t=0) $ und $ c_1(t=0) $ an. Eine nicht verschwindende Störung $ V $ in \eqref{3.18} wird dann gerade eine nichttriviale Dynamik von $ c_0 $ und $ c_1 $ induzieren.\par
Einsetzen von \eqref{3.28} in die Schrödingergleichung \eqref{3.11} (mit dem Ziel, Entwicklungsgleichungen für $ c_0 $ und $ c_1 $ zu gewinnen):
\begin{equation*}
\begin{aligned}
i \hbar \prd{|\psi\rangle}{t} &= i \hbar \dot{c}_0 e^{-iE_0t/\hbar} |0\rangle + E_0 c_0 e^{-iE_0t/\hbar} |0\rangle + i \hbar \dot{c}_1 e^{-iE_1t/\hbar} |1\rangle + E_1 c_1 e^{-iE_1 t/\hbar} |1\rangle \\
&\overset{\mathclap{\tx{R.S.}}}{=} H|\psi\rangle \overset{\substack{ H = H_0 + V \\ \eqref{3.18} \\ \eqref{3.26} \\ \eqref{3.28} }}{=} E_0 c_0 e^{-i E_0 t/\hbar} |0\rangle + E_1 c_1 e^{-i E_1 t/\hbar} |1\rangle + c_0 e^{-i E_0 t/\hbar} V |0\rangle + c_1 e^{-i E_1 t/\hbar} |1\rangle
\end{aligned}
\end{equation*}
ergo:
\begin{equation}
i\hbar \dot{c}_0 e^{-i E_0 t/\hbar} |0\rangle + i \hbar \dot{c}_1 e^{-i E_1 t/\hbar} |1\rangle = c_0 e^{-i E_0 t/\hbar} V |0\rangle + c_1 e^{-i E_1 t/\hbar} V |1\rangle
\label{3.30}
\end{equation}

%03.06.19

\noindent
Um Entwicklungsgleichungen für $ c_0 $ und $ c_1 $ zu erhalten, filtern wir die entsprechenden Anteile durch (Skalare) Multiplikation von \eqref{3.30} von links  mit $ \langle 0 | $ bzw. $ \langle 1 | $ heraus. Dabei setzen wir zusätzlich voraus, dass die Diagonalelemente von $ V $ aus Paritätsgründen (d.h. Symmetriegründen, s. weiter unten - Begründung wird nachgeliefert) verschwinden sollen (häufig wahr):
\begin{equation}
\langle j | V | j \rangle = 0
\label{3.31}
\end{equation}
Somit:
\begin{equation}
i \hbar \dot{c}_0 = c_1 \langle \phi_0 | V | \phi_0 \rangle \qquad i \hbar \dot{c}_1 = c_0 \langle \phi_1 | V | \phi_1 \rangle
\label{3.32}
\end{equation}
mit $ |\phi_j\rangle = e^{-i E_j t / \hbar} | j \rangle $. Mit \eqref{3.27} wird das zu:
\begin{equation*}
\begin{aligned}
\langle \phi_0 | V | \phi_0 \rangle &\overset{\eqref{3.32}}{=} e^{-i (E_1-E_0) t / \hbar} \langle 0 | V | 1 \rangle = e^{-i\omega_0t} e F \langle 0 | x | 1 \rangle \cos(\omega t) \\
=& e^{-i \omega_0 t} \mathcal{V} \hbar \cos(\omega t)
\end{aligned}
\end{equation*}
\begin{equation}
\omega_0 \defeq \frac{E_1 - E_0}{\hbar} \qquad \qquad \mathcal{V} = \frac{e F \langle 0 | x | 1 \rangle}{\hbar}
\label{3.33}
\end{equation}
worin $ \omega_0 $ die Übergangsfrequenz ist und $ \mathcal{V} $ die Kopplungsstärke, mit der Feldamplitude $ F $.\par
Damit wird \eqref{3.32} zu:
\begin{equation}
i \dot{c}_0 = c_1 e^{-i\omega_0 t} \mathcal{V} \cos(\omega t) \qquad i \dot{c}_1 = c_0 e^{+i\omega_0 t} \mathcal{V}^{*} \cos(\omega t)
\label{3.34}
\end{equation}
[Übungsaufgabe: Man zeige dass: $ ||\, |\psi\rangle \, || ^2 = 1 \forall t $]\par
\eqref{3.34} kann für die Matrixelemente der $ |\psi\rangle $ repräsentierenden Dichtematrix $ \sigma $ umformuliert werden: Mit \eqref{2 71}
\begin{align}
\sigma_{00} &= |c_0|^2 = c_0 c_0^* \quad \ \ \, \sigma_{11} = |c_1|^2 = c_1 c_1^* \quad \equiv \tx{ Besetzung bzw. Population}
\label{3.35}\\
\sigma_{01} &= c_0 c_1^* \qquad \qquad \quad \sigma_{10} = c_1 c_0^* = \sigma_{01}^* \ \, \quad \equiv \tx{ Kohärenzen ,,coherences``}
\label{3.36}
\end{align}
Aus der Kohärenz $ \to $ Phaseninformation, kodiert Energieskalare ($ E_0 - E_1 $) (für $ \mathcal{V} = 0 $).\par
Mit \eqref{3.34} ergibt sich daher für die Zeitentwicklung der $ \sigma_{ij} $:
\begin{equation}
\begin{aligned}
\dot{\sigma}_{11} &= - \dot{\sigma}_{00} = - i \cos(\omega t) \left[ \sigma_{01} \mathcal{V}^* e^{i \omega_0 t} - \sigma_{10} \mathcal{V} e^{-i\omega_0 t}\right] \\
\dot{\sigma}_{01} &= \phantom{-}\dot{\sigma}_{10}^* = i \mathcal{V} \cos(\omega t) e^{-i \omega_0 t} \left[\sigma_{00} - \sigma_{11}\right]
\end{aligned}
\label{3.37}
\end{equation}
Mit $ \cos(\omega t) = \frac{1}{2} \left(e^{i\omega t} + e^{-i\omega t}\right) $ erhält man in \eqref{3.37} Terme der Form:
\begin{equation}
e^{\pm i (\omega + \omega_0) t} \quad \tx{ und } \quad e^{\pm i (\omega - \omega_0) t}
\label{3.38}
\end{equation}
Unter der Voraussetzung nah-resonanten Antriebs [! remember your classical mechanics class - chapter on driven harmonic oszillators !] d.h. $ \omega \simeq \omega_0 $ \eqref{3.38} [``near-resonant driving''], \textbf{separieren} die mit $ \omega + \omega_0 $ und $ \omega - \omega_0 $ assozierten Zeitskalen, was die Vernachlässigung der schnell oszillierenden Terme $ (\sim e^{\pm i (\omega + \omega_0) t}) $ erlaubt. [Integration über $ t $ liefert Beiträge $ \sim \frac{1}{\omega + \omega_0} e^{+ i (\omega + \omega_0) t} $ bzw. $ \sim \frac{1}{\omega - \omega_0} e^{i(\omega - \omega_0) t} $, dessen letzterer mit einem \textbf{Resonanznenner} behaftet ist.]\\[10pt]
\emph{Bemerkung:} Mindestens zwei Möglichkeiten: ,,Starke Kopplung`` zu induzieren: a) $ F $ sehr groß! oder b) resonante Kopplung.\par
In dieser \textbf{Drehwellnäherung} / ``\textbf{rotating wave approximation}'' (RWA) [in anderem Zusammengang auch: Säkularnäherung] nimmt \eqref{3.37} die folgende Form an:
\frbox{optische Bloch-Gleichungen}{
\begin{equation}
\begin{aligned}
\dot{\sigma}_{11} &= - \frac{i}{2} \left[\sigma_{01} \mathcal{V}^* e^{i (\omega_0 - \omega) t} - \sigma_{10} \mathcal{V} e^{-i(\omega_0 - \omega) t}\right] \\
\dot{\sigma}_{01} &= \frac{i}{2} \mathcal{V} e^{i (\omega - \omega_0) t} \left[\sigma_{00} - \sigma_{11}\right]
\end{aligned}
\label{3.39}
\end{equation}}
\noindent
Die Voraussetzung dafür war die Gleichung \eqref{3.38}!\\[10pt]
\textbf{Lösungsansatz:}
\begin{equation}
\begin{aligned}
\sigma_{00} &= \sigma_{00}^{(0)} \exp(\lambda t) \qquad \sigma_{01} = \sigma_{01}^{(0)} \exp[ - i (\omega_0 - \omega) t] \exp(\lambda t) \\
\sigma_{11} &= \sigma_{11}^{(0)} \exp(\lambda t) \qquad \sigma_{10} = \sigma_{10}^{(0)} \exp[ + i (\omega_0 - \omega) t] \exp(\lambda t)
\end{aligned}
\label{3.40}
\end{equation}
\begin{equation*}
\sigma_{ij} \sim \dots \exp(\lambda t) \left[ \exp\left[ i (\omega - \omega_0) t\right]\right]
\end{equation*}
Einsetzen von \eqref{3.40} in \eqref{3.39} liefert:
\begin{equation}
\begin{pmatrix}
-\lambda & 0 & \frac{1}{2} i \mathcal{V}^* & - \frac{1}{2} i \mathcal{V} \\[5pt]
0 & - \lambda & - \frac{1}{2} i \mathcal{V} ^* & \frac{1}{2} i \mathcal{V} \\[5pt]
\frac{1}{2}i \mathcal{V} & - \frac{1}{2} i \mathcal{V} & i ( \omega_0 - \omega) - \lambda & 0 \\[5pt]
- \frac{1}{2} i \mathcal{V}^* & \frac{1}{2} i \mathcal{V} ^* & 0 & - i (\omega_0 - \omega) - \lambda
\end{pmatrix} \begin{pmatrix}
\sigma_{00}^{(0)} \\[5pt] \sigma_{11}^{(0)} \\[5pt] \sigma_{01}^{(0)} \\[5pt] \sigma_{10}^{(0)}
\end{pmatrix} = 0
\label{3.41}
\end{equation}
Die möglichen Werte ergeben sich aus dem Verschwinden der Determinante der Koeffizientenmatrix in \eqref{3.41}:
\begin{equation}
\lambda^2 \left[\lambda^2 + (\omega_0 - \omega)^2 + |\mathcal{V}|^2\right] = 0
\label{3.42}
\end{equation}
mit den Wurzeln:
\begin{equation}
\begin{aligned}
\lambda_1 = 0 , \quad \lambda_2 = i \Omega , \quad \lambda_3 = - i \Omega  \\
\tx{mit} \quad \Omega = \sqrt{(\omega_0 - \omega)^2 + |\mathcal{V}|^2}
\end{aligned}
\label{3.43}
\end{equation}
Die allgemeinste Lösung von \eqref{3.40} ist demnach von der Form:
\begin{equation}
\sigma_{ij} = \sigma_{ij}^{(1)} + \sigma_{ij}^{(2)} \exp(i \Omega t) + \sigma_{ij}^{(3)} \exp(-i\Omega t)
\label{3.44}
\end{equation}
wobei $ (1) $ sich hierbei auf die Lösung $ \lambda_1 $ bezieht.\par
Wobei die $ \sigma_{ij}^{(k)} $ durch die $ \lambda $-unabhängigen Faktoren der rechten Seite der jeweiligen (durch $ i,j $ identifizierten) Gleichung in \eqref{3.40} gegeben sind.

%04.06.19

\noindent
RWA (at resonance), opt. Bloch Gleichungen.\\[5pt]
Lösungen für spezielle Anfangsbedingungen
\begin{equation}
\begin{aligned}
\sigma_{11}(t = 0) &= 0 \\
\sigma_{01}(t = 0) &= 0
\end{aligned}
\label{3.45}
\end{equation}
Mit \eqref{3.44} und \eqref{3.45} in \eqref{3.39} ergeben sich die Lösungen ($ \to $ ÜA 25)
\begin{equation}
\begin{aligned}
\sigma_{11}(t) &= \frac{|\mathcal{V}|^2}{\Omega^2} \sin^2 \left[\frac{1}{2} \Omega t\right] \\
\sigma_{01}(t) &= e^{-i(\omega_0 - \omega )t} \frac{\mathcal{V}}{\Omega^2} \sin^2 \left[\frac{1}{2} \Omega t\right] \left[- (\omega_0 - \omega) \sin\left[\frac{1}{2} \Omega t\right] + i \Omega \cos \left[\frac{1}{2} \Omega t\right]\right]
\end{aligned}
\label{3.46}
\end{equation}
Speziell für \textbf{resonanten Antrieb} $ \omega = \omega_0 $ (mit \eqref{3.43})
\frbox{Rabi.Oszillationen}{
\begin{equation}
\sigma_{11}(t) = \sin^2 \left(\frac{1}{2} | \mathcal{V} | t\right) \qquad \sigma_{01}(t) = i \frac{\mathcal{V}}{|\mathcal{V}|} \sin\left(\frac{1}{2} \Omega t\right) \cos\left(\frac{1}{2} \Omega t\right)
\label{3.47}
\end{equation}}
Die Gleichungen \eqref{3.46}, \eqref{3.47} beschreiben die \textbf{Rabi-Oszillationen} der atomaren Population zwischen Grund- und angeregtem Zustand (s. \cref{3.43}, \eqref{3.46}). und \textbf{Rabi-Frequenzen} $ \Omega $ von der \textbf{Verstimmung} (``\textbf{detuning}'') der treibenden Frequenz $ \omega $ von der \textbf{Übergangsfrequenz} $ \omega_0 $ (``\textbf{Iteration frequency''}) \footnote{Colloquium 3.6.19 ``antibunching''}.
\begin{equation}
\Delta = \omega - \omega_0
\label{3.48}
\end{equation}

\subsection{Geometrische Interpretation - die Bloch-Kugel}

Diese erhalten wir durch Re-Formulierung der optischen Bloch-Gleichungen \eqref{3.39}:
\begin{equation}
\tilde{\sigma}_{00} \defeq \sigma_{00} \quad \tilde{\sigma}_{11} \defeq \sigma_{11} \quad \tilde{\sigma}_{01} \defeq \sigma_{01} e^{i(\omega_0 - \omega)t} \quad \tilde{\sigma}_{10} = \tilde{\sigma}_{01}^*
\label{3.49}
\end{equation}
\eqref{3.39} transformiert sich mit \eqref{3.49} zu:
\begin{equation}
\rmbox{%
\begin{aligned}
\dot{\tilde{\sigma}}_{11} = - \frac{i}{2} \left[\tilde{\sigma}_{01}\mathcal{V}^* - \tilde{\sigma}_{10} \mathcal{V}\right] \quad \ \ \dot{\tilde{\sigma}}_{00} = - \dot{\tilde{\sigma}}_{11} \qquad \dot{\tilde{\sigma}}_{01} = \frac{i}{2} \mathcal{V} \left[\tilde{\sigma}_{00} - \tilde{\sigma}_{11}\right] + i \Delta \tilde{\sigma}_{01} \quad \ \  \dot{\tilde{\sigma}}_{10} = \dot{\tilde{\sigma}}_{01}^*
\end{aligned}}
\label{3.50}
\end{equation}
Wobei $ \Delta = \omega - \omega_0 $, also die Verstimmung, ist.\\[10pt]
Neue Variablen:
\begin{equation}
\begin{aligned}
u &\defeq \tilde{\sigma}_{10} + \tilde{\sigma}_{01} \equiv \tx{ Realteil der Kohärenz} \\
v &\defeq i (\tilde{\sigma}_{10} - \tilde{\sigma}_{01}) \equiv \tx{ Imaginärteil der Kohärenz} \\
w &\defeq \tilde{\sigma}_{11} - \tilde{\sigma}_{00} \equiv \tx{ Populationsdifferenz}
\end{aligned}
\label{3.51}
\end{equation}
Unter der weiteren Anmerkung,dass $ \mathcal{V} $ reell sein soll (wenig einschränkend), nehmen die transformierten Bloch-Gleichungen \eqref{3.50} folgende Gestalt an:
\begin{equation}
\dot{u} = - \Delta v \qquad \dot{v} = - \mathcal{V}w + \Delta u \qquad \dot{w} = \mathcal{V} v
\label{3.52}
\end{equation}
Der \textbf{Bloch Vektor}
\begin{equation}
\vec{S} = \underbracket{u \hat{x} + v \hat{y}}_{\tx{Kohärenzen}} + w \hat{z}
\label{3.53}
\end{equation}
(wobei w den Anteil des Grund- bzw. Angeregten Zustands ist) hat die Norm
\begin{equation}
|\vec{S}|^2 = |u|^2 + |v|^2 + |w|^2 \overset{!}{=} 1
\label{3.54}
\end{equation}
mit
\begin{equation}
\prd{}{t} |\vec{S}|^2 = 0
\label{3.55}
\end{equation}
und gehorcht der Bewegungsgleichung (komponentenweise)
\begin{equation}
\dot{\vec{S}} = \vec{Q} \times \vec{S} \quad \tx{, wobei } \quad \vec{Q} = \mathcal{V} \hat{x} + \Delta \hat{z} = (\mathcal{V},0,\Delta)
\label{3.56}
\end{equation}
(i8m wesentlichen ist das \eqref{3.52} in disguise).\\[5pt]
Wegen \eqref{3.55} kan eqref multiple equationsn daher die Dynamik des nah-resonant getriebenen Zwei-Niveau-Systems auf der Einheitskugel (\textbf{Bloch Kugel}) vollständig beschrieben werden.\par
Mittelpunkt der Bloch-Kugel ist gegeben durch $ \vec{S} = 0 $, was
\begin{equation}
\sigma = \begin{pmatrix}
\frac{1}{2} & 0 \\ 0 & \frac{1}{2}
\end{pmatrix}
\label{3.57}
\end{equation}
impliziert. $ \sigma_{\vec{S} = 0} $ heißt \textbf{maximal gemischter Zustand}.\\[10pt]
\emph{Bemerkungen:}
\begin{enumerate}[1.)]
	\item Die Situation $ w = 1 $, d.h. wegen \eqref{3.51}, $ \sigma_{00} = 0 $, $ \sigma_{11} = 1 $ heißt \textbf{vollständige Inversion} (``complete inversion'') der atomaren Population (oben in der Bloch Kugel) ($ \to $ insbesondere Laser-Theorie).
	\item $ w = -1 $, $ \sigma_{11} = 0 $, $ \sigma_{00} = 1 $, Atom im Grundzustand (unten in der Bloch Kugel).
	\item $ \Delta = 0 \ \overset{\eqref{3.56}}{\Rightarrow} \ \vec{Q} = \mathcal{V} \hat{x} $, d.h. $ \vec{S} $ rotiert in einer durch $ \hat{x} $ definierten Ebene. \label{drei}
	%
	%17.06.19
	%
	\item Ist die Feldamplitude $ \vec{F} = \vec{F}(t) $ zeitabhängig, so vererbt sich diese Zeitabhängigkeit auf $ \mathcal{V} $ (wegen \eqref{3.33}). Es lässt sich dann zeigen (insbesondere für $ \Delta = 0 $, s.o. \ref{drei}), dass der Blochvektor unter dem Einfluss von $ \vec{F}(t) $ eine Rotation um den Winkel
	\begin{equation}
	\theta(T) = \int_{0}^{T} \mathcal{V}(t') \dd t'
	\label{3.58}
	\end{equation}
	durchläuft. Man nennt diese Größe auch die \textbf{Fläche unter dem Puls} (oder ``\textbf{area under the pulse}'').\par
	Dies erlaubt die deterministische Initialisierung eines Zweiniveausystems (eines Qubits - für Quantencomputer bzw. eines Quantenregisters) beiispielsweise in einer kohärenten Überlagerung \`a la $ \frac{1}{\sqrt{2}} (|0\rangle \pm |1\rangle) $ per \textbf{$ \frac{\pi}{2} $-Puls} ($ \theta = \frac{\pi}{2} $) bzw. des angeregten Zustands $ |1\rangle $ durch einen \textbf{$ \pi $-Puls} (\textbf{Kohärente Kontrolle}).
	\item \eqref{3.43}, \eqref{3.46} zeigen, dass im Falle nichttrivialer Störung die ungestörte charakteristische Frequenzdifferenz $ \omega - \omega_0 $ (einzige relevante Skala) ersetzt wird durch die \textbf{Rabi-Frequenz} $ \Omega $. Dynamik wird vermittelt durch $ U = e^{i H t / \hbar} $ vermittelt, mit durch die Eigenwerte von $ H $ per $ E_j \sim \hbar \omega_j $ definierten Frequenzen. Dass heißt die Modifikation $ (\omega - \omega_0) \overset{F = 0 \to F \gneq 0}{\longrightarrow} \Omega $ geht mit einer Modifikation der Eigenwerte von $ H $ einher.\par
	Um die Verhältnisse zu veranschaulichen, trägt man $ \Omega $ als Funktion von $ \Delta $ auf.
	
	
	%T1
	\hft
	
	\noindent
	\textbf{Vermiedene Kreuzung} oder ``\textbf{anti-crossing} vs. \textbf{Kreuzung}/``\textbf{crossing}'' von Energiieniveaus. $ \to $ entspricht allgemein dem Szenario der parametrischen \textbf{Entwicklung der Eigenniveaus} eines Hamiltonoperators (``parametric-level-evolution'' oder ``level dynamics'').\par
	In obigem Bild ist der relevante Parameter die Feldfrequenz $ \omega $, bei festem $ \omega_0 $ und $ F $ (experimentell: frequency chirp).\par
	Die Wirkung der Störung auf die spektrale Struktur ist am Stärksten bei $ \Delta = 0 $ [s.u. $ \to $ entartete Störungsrechnung (im Gegensatz zur nicht-entarteten) Störungstheorie].\par
	Bei starker Verstimmung, d.h. $ \Delta \gg |\mathcal{V}| $, wird die Verschiebung gegenüber den ungestörten Niveaus immer kleiner und ist von der Ordnung $ \frac{|\mathcal{V}|^2}{\Delta^2} $, denn:
	\begin{equation}
	\Omega \overset{\substack{\eqref{3.48} \\ \eqref{3.44}}}{=} \sqrt{\Delta^2 + |\mathcal{V}|^2} = \Delta \sqrt{1 + \frac{|\mathcal{V}|^2}{\Delta^2}} \ \overset{|\mathcal{V}| \ll \Delta}{\simeq} \ \Delta \left(1 + \frac{1}{2} \frac{|\mathcal{V}|^2}{\Delta^2}\right)
	\label{3.59}
	\end{equation}
	[Wichtige Konsequenz z.B. in der Atomphysik: unterschiedliches Verhalten der Spektren von atomarem Wasserstoff und von Alkaliatomen in statischen elektrischen Feldern] heißt im Jargon \textbf{AC Stark shift}, \textbf{dynamical Stark effect}. Betrachtet man nun die räumliche Variation der Feldamplitude (natürlicher weise gegeben durch das Profil der Feldmode), so ergibt sich eine \textbf{ortsabhängige}) Verschiebung der Energieniveaus, m.a.W. eine nichttriviale Potentiallandschaft. Dies impliziert per Kraft $ = - $ grad Potantial eine Kraft auf das Zweiniveauatom, die lediglich in Gleichgewichtspunkten verschwindet.
	
	
	%T2
	\hft
	
	\noindent
	$ \to $ s.a. optical tweezers
	\item $ \Omega^{-1} $ kleiner (d.h. schneller) als Zeitskala typischer Störeffekte (talk to Ranak o. Guiseppe Sansone. they know!)
\end{enumerate}