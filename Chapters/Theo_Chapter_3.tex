\chapter{Dynamik}

\textbf{Bisher:} statische Struktur des Zustandsraums der Quantenmechanik.\\[5pt]
\textbf{Jetzt:} Parametrisierung in der zeit (zunächste aber \textbf{ohne} Begründung der ,,Richtung`` der Zeit ($ \to $ Ausblick: Dekohärenztheorie als ein aktueller Kandidat für den Ursprung des Zeitpfeils.)).

\section{Die Schrödinger-Gleichung}

Forderungen an die zeitliche Entwicklung eines Zustandes $ |\psi\rangle: $ Stetigkeit, d.h.
\begin{equation}
\lim_{t\to t_0} |\psi(t)\rangle = |\psi(t_0)\rangle = |\psi_0\rangle \equiv \tx{Anfangszustand}
\label{3.1}
\end{equation}
wobei $ t > t_0 $.\\
Weiter werde der Zustand $ |\psi(t)\rangle $ durch Operation des \textbf{Zeitentwicklunngsoperators} $ U(t,t_0) $ auf $ |\psi_0\rangle $ erzeugt, d.h.
\begin{equation}
|\psi(t) \rangle = U(t,t_0) |\psi_0\rangle
\label{3.2}
\end{equation}
Im Sinne der Interpretation von $ |\psi\rangle $ also Wahrscheinlichkeitsamplitude für Messergebnisse fordern wir die Normerhaltung im Laufe der Zeit, was die Unitarität von $ U(t,t_0) $ impliziert (daher auch Sprechweise $ U \equiv $ ''the unitary``).
\begin{equation}
U^{+}(t,t_0) U(t,t_0) = \mathbbm{1} \quad \tx{bzw.} \quad U^{+}(t,t_0) = U^{-1}(t,t_0)
\label{3.3}
\end{equation}
Dies garantiert:
\begin{equation}
||\, |\psi(t)\rangle || = 1 \ \forall t
\label{3.4}
\end{equation}
Im Sinne einer kontinuierlichen und in einzelnen Zeitintervalle zerlegbaren Entwicklung fordern wir außerdem:
\begin{equation}
U(t,t_0) = U(t_1,t_2) U(t,t_0) \qquad \forall t_2 > t_1 > t_0
\label{3.5}
\end{equation}
Mit der Stetigkeitsbedingung \eqref{3.1} folgt für ein infinitesimales Zeitintervall $ \dd t $:
\begin{equation}
\lim_{\dd t \to 0} U(t_0 + \dd t , t_0) = \mathbbm{1}
\label{3.6}
\end{equation}
Der Ansatz
\begin{equation}
U(t_0 + \dd t, t_0) = \mathbbm{1} - i \Omega \dd t
\label{3.7}
\end{equation}
mit $ \Omega = \Omega^{\dagger} $, erfüllt die Forderungen aus \eqref{3.3}, \eqref{3.5} und \eqref{3.6} - unter der Voraussetzung, dass Terme der Ordnung $ \dd t^2 $ vernachlässigbar sind.

%27.05.19

\begin{equation*}
\mathbbm{1} - i \Omega \dd t + i \Omega \dd t + \cancel{c \dd t^2}
\end{equation*}
$ \Omega $ in Gleichung \eqref{3.7} hat die Dimension $ [S^{-1}] $, was, zusammen mit der Rolle der klassischen Hamiltonfuntion als Erzeuger der Phasenraumdynamik (siehe insbesondere Liouville-Satz bzw. Liouville-Gleichung, die die Zeitentwicklung der Phasenraumvolumens beschreibt. $ \sim i \prt{f}{t} = L f $) und
\begin{equation}
E \sim \hbar \omega
\label{3.8}
\end{equation}
(nach Planck) zu der Identifikation:
\begin{equation}
\Omega \defeq \frac{H}{\hbar}\
\label{3.9}
\end{equation}
mit $ H \equiv $ Hamiltonfunktion, führt. In der QM bezeichnen wir $ H $ als den \textbf{Hamilton-Operator}, der für autonome (d.h. Zeitunabhängige) $ H $ mit dem Energieoperator identifiziert wird.\par
Wir leiten nun die Bewegungsgleichungen für $ U $ her, um über beliebige Zeitintervalle propagieren zu können. Propagieren $ \leftrightarrow $ $ U = $ Propagator (siehe HöMa, Cauchy Integrale und Zusammenhang Propagator und Greensfunktionen).
\begin{align*}
U(t + \dd t, t_0) \overset{\eqref{3.5}}{=} U(t + \dd t, t) U(t, t_0) \overset{\substack{\eqref{3.7} \\ \eqref{3.9}}}{=} \left(\mathbbm{1} - i \frac{H}{\hbar} \dd t\right) U(t,t_0)
\end{align*}
\begin{equation*}
U(t + \dd t, t_0) - U(t, t_0) = - i \frac{H}{\hbar} \dd t U(t, t_0)
\end{equation*}
woraus der Differenzquotient
\begin{equation*}
\lim_{\dd t \to 0} \frac{U(t + \dd t, t_0) - U(t,t_0)}{\dd t} = \prt{U}{t} = - i \frac{H}{\hbar} U(t, t_0)
\end{equation*}
\begin{equation}
\rmbox{ i \hbar \prt{U(t,t_0)}{t} = H U(t,t_0) }
\label{3.10}
\end{equation}
Durch Multiplikation von rechts mit $ |\psi_0\rangle $ erhalten wir die Schrödingergleichung:
\frbox{Schrödingergleichung}{\begin{equation}
i \hbar \prt{}{t} | \psi(t) \rangle = H | \psi(t) \rangle
\label{3.11}
\end{equation}}
\noindent
welche $ \forall t $ gültig ist.

\subsection{Heisenberg- vs. Schrödinger-Bild}

Die Gleichung \eqref{3.2} liefert die zeitliche Entwicklung des Zustandsvektors
$$ |\psi(t)\rangle \custo{\rightarrow}{\sim}{\mathclap{\tx{Kap.I}}} c_0(t) | 0 \rangle + c_1(t) | 1 \rangle $$
d.h. der Zustand $ |\psi(t)\rangle $ bewegt sich bezüglich eines festen Koordinatensystems. Völlig gleichberechtigt ist die Perspektive, wonach der Zustandsvektor zeitlich unveränderlich ist, sich jedoch das Koordinatensystem bewegt oder dreht. Während nach \eqref{3.2} also die Zustände die Zeitabhängigkeit tragen, wird in der alternativen Sichtweise die Zeitabhängigkeit vollständig auf die Observablen abgewälzt (die Observablen helfen ja über ihre Eigenvektoren die Vektorraumbasis geliefert).
\begin{figure}[ht]
	\centering
	%t1
	\begin{tikzpicture}
		\draw[->] (0,0) -- (4,0) node[anchor=north west] {$ |0\rangle $};
		\draw[->] (0,0) -- (0,4) node[anchor=south east] {$ |1\rangle $};
		\draw[->,blue] (0,0) -- (30:4) node[right] {$ |\psi(t=0)\rangle $};
		\draw[->,red] (0,0) -- (60:4) node[right] {$ |\psi(t>0)\rangle $};
		\centerarc[->](0,0)(30:60:2.5);
		\node at (45:3) {\eqref{3.2}};
		\node at (2,-2) {\tx{\textbf{Schrödinger-Bild}}};
		% 2tes
		\coordinate (o) at (7,0);
		\draw[->,blue] (o) -- ++(90:4) node[anchor=south east] {$ |1\rangle $};
		\draw[->,blue] (o) -- ++( 0:4) node[right] {$ |0\rangle $};
		\draw[->,red] (o) -- ++(70:4) node[right] {$ |1\rangle' $};
		\draw[->,red] (o) -- ++(-20:4) node[right] {$ |0\rangle' $};
		\draw[->] (o) -- ++(35:4) node[right] {$ |\psi(t=0)\rangle = |\psi_0\rangle $};
		\node at ($ (o) + (2,-2) $) {\tx{\textbf{Heisenberg-Bild}}};
	\end{tikzpicture}
	\caption{Verschiedene Modelle zur Vorstellung einer Zeitabhängigen Wellenfunktion von Schrödinger und Heisenberg.}
	\label{SHBild}
\end{figure}
\begin{equation}
|\psi(t)\rangle _{\tx{Schrödinger}} \overset{\eqref{3.2}}{=} U(t,t_0) |\psi_0\rangle \quad \curvearrowright \quad |\psi_0\rangle \eqdef |\psi \rangle_{\tx{Heisenberg}} = U^{\dagger}(t,t_0) |\psi(t)\rangle_{\tx{Schrödinger}}
\label{3.12}
\end{equation}
Erwartungswerte von Observablen sollten nicht von der Wahl des Bildes abhängen. Daher:
\begin{equation}
\begin{aligned}
_{\tx{Schrödinger}} \left\langle \psi(t) \right| A_{\tx{Schrödinger}} \left| \psi(t) \right\rangle_{\tx{Schrödinger}} &\overset{\eqref{3.2}}{=} \ub{\left\langle \psi_0 \right| }_{\mathclap{\tx{Heisenberg} \langle\psi |}} U^{\dagger} (t,t_0) A_{\tx{Schrödinger}} U(t,t_0) \left| \psi_0 \right\rangle\\
&\overset{\eqref{3.12}}{=} _{\tx{Heisenberg}} \langle\psi| A_{\tx{Heisenberg}} (t) | \psi \rangle_{\tx{Heisenberg}}
\end{aligned}
\label{3.13}
\end{equation}
mit
\begin{equation}
A_{\tx{Heisenberg}}(t) = U^{\dagger}(t,t_0) A_{\tx{Schrödinger}} U(t,t_0)
\label{3.14}
\end{equation}
\noindent
Anstelle der Schrödingergleichung \eqref{3.11} benötigen wir Entwicklungsgleichung für $ A_{\tx{Heisenberg}} $ die sich aus \eqref{3.10} und \eqref{3.14} gewinnen lässt:
\begin{equation*}
i \hbar \prt{}{t} A_{\tx{Heisenberg}} \overset{\eqref{3.14}}{=} i \hbar \left[\prt{U^{\dagger}}{t} A_{\tx{Schrödinger}} U + U^{\dagger} \prt{A_{\tx{Schrödinger}}}{t} U + U^{\dagger} \prt{U}{t}\right]
\end{equation*}
Umformen mit $ i \hbar \prt{U}{t} |\psi_0\rangle = H \ub{U |\psi_0\rangle}_{|\psi(t)\rangle} $
\begin{equation*}
= - U^{\dagger} H_{\tx{Schrödinger}} A_{\tx{Schrödinger}} U + U^{\dagger} A_{\tx{Schrödinger}} H_{\tx{Schrödinger}} U + i \hbar U^{\dagger} \prt{A_{\tx{Schrödinger}}}{t} U
\end{equation*}
\begin{equation}
\Rightarrow \quad i \hbar \prt{}{t} A_{\tx{Heisenberg}} = U^{\dagger} \left[A_{\tx{Schrödinger}}, H_{\tx{Schrödinger}}\right] U + i \hbar U^{\dagger} \prt{A_{\tx{Schrödinger}}}{t} U
\label{3.15}
\end{equation}
Mit $ U U^{\dagger} \overset{\eqref{3.3}}{=} \mathbbm{1} $ und $ A_{\tx{Heisenberg}} \overset{\eqref{3.14}}{=} U^{\dagger} A_{\tx{Schrödinger}} U $ wird \eqref{3.15} zu:
\frbox{Äquivalente dynamik im Heisenberg Bild zu SGL}{\begin{equation}
i \hbar \prt{}{t} A_{\tx{Heisenberg}} = \left[A_{\tx{Heisenberg}}, H_{\tx{Heisenberg}}\right] + i \hbar \left(\prt{A_{\tx{Schrösinger}}}{t}\right)_{\tx{Heisenberg}}
\label{3.16}
\end{equation}}
\noindent
\textbf{äquivalente} Dynamik zur Schrödinger-Gleichung \eqref{3.11} im Heisenberg Bild. Die direkte Analogie zur klassischen Hamilton'schen Dynamik in den Übungen.

\subsection{Das Wechselwirkungsbild}

$ \to $ Eine zwischen Schrödinger- und Heisenberg-Bild ,,interpolierende`` Perspektive, die eine Zerlegung des Hamiltonoperators voraussetzt, wobei $ H_0 $ die ,,triviale`` Dynamik (ungestört und bekannt!) erzeugt, $ V $ die Abweichung oder Störung davon induziert.


\setcounter{equation}{17}


\noindent
\begin{equation}
H = H_0 + V
\label{3.18}
\end{equation}
Das \textbf{Wechselwirkungsbild} wälzt die $ H_0 $-Dynamik auf die Observable ab, die $ V $-Dynamik auf den Zustand ($ \to $ rotating wave approximation).\par
Zustandsdynamik:
\begin{equation}
\begin{aligned}
|\psi(t) \rangle_{\tx{Wechselwirkung}} &= U_{\tx{Wechsewirkung}} |\psi\rangle_{\tx{Heisenberg}} \\
&= U_{\tx{Wechselwirkend}} |\psi_{0}\rangle
\end{aligned}
\label{3.19}
\end{equation}
und im Schrödinger-Bild:
\begin{equation}
|\psi(t)\rangle_{\tx{Schrödinger}} = U_0(t,t_0) U_{\tx{Wechselwirkung}} |\psi_{0}\rangle
\label{3.20}
\end{equation}
beziehungsweise:
\begin{equation}
U(t,t_0) = U_0(t,t_0) U_{\tx{Wechselwirkend}} (t,t_0)
\label{3.21}
\end{equation}
wobei $ U_{\tx{Wechselwirkung}} $ von $ V $, $ U_0 $ von $ H_0 $ erzeugt wird
\begin{equation}
\begin{aligned}
i \hbar \prt{}{t} U(t,t_0) &\overset{\eqref{3.21}}{=} i \hbar \prt{}{t} (U_0 U_{\tx{WW}}) = i \hbar \left[\prt{U_0}{t} U_{\tx{WW}} + U_0 \prt{U_{\tx{WW}}}{t}\right] \\
&\overset{\eqref{3.10}}{=} \rmbox{H_0 U_0 U_{\tx{WW}} + i \hbar U_0 \prt{U_{\tx{WW}}}{t}} \\
&\overset{\substack{\eqref{3.10} \\ \eqref{3.18}}}{=} i \hbar (H_0 + V) U \overset{\eqref{3.21}}{=} \rmbox{H_0 U_0 U_{\tx{WW}} + V U_0 U_{\tx{WW}}}
\end{aligned}
\label{3.22}
\end{equation}
d.h.:
\begin{equation*}
i \hbar U_0 \prt{U_{\tx{WW}}}{t} = V U_0 U_{\tx{WW}}
\end{equation*}
\begin{equation}
\begin{aligned}
\overset{U_{0}^{\dagger} \cdot | \eqref{3.22}}{\Rightarrow} \quad i \hbar \prt{U_{\tx{WW}}}{t} &= U_0^{\dagger} V U_0 U_{\tx{WW}} \\
&\defeq V_{\tx{WW}} U_{\tx{WW}}
\end{aligned}
\label{3.23}
\end{equation}