\chapter{Wechselwirkungen von Quanten / Photonen}

\section{Photoeffekt}

Photoeffekt = Absorption eines Protons ist gebunden an Hüllenelektron
\begin{equation*}
\gamma e^- A \to e^- A^+
\end{equation*}


%T1
\hft


\noindent
Wichtig $ E_{\gamma} \overset{<}{\approx} E_{\tx{bindung}} \approx \mathcal{O}(100 \, \tx{keV}) $. 10\% der WW an $ e^- $ der inneren Schalen.
\begin{equation*}
\sigma_{\tx{tot}} \propto Z^5 \cdot \left(\frac{m_e c}{E_{\gamma}}\right)^{- \nicefrac{7}{2}}
\end{equation*}
\textbf{Wichtig:} $ \sigma_{\tx{Photoeffekt}} $ ist pro Atom

\section{Compton Streuung}

\folie{Wechselwirkung von Photonen mit Materie}\\[5pt]
Streuung an quasi-freien $ e^- $:

%T2
\hft


\noindent
Energie \& Impulserhaltung
\begin{equation*}
E_{\gamma'} = \frac{E_{\gamma}}{1 + \frac{E_{\gamma}}{m_e c^2} (1 - \cos \theta)}
\end{equation*}
\begin{equation*}
\lambda_{\gamma '}' = \lambda_{\gamma} + \lambda_C (1 - \cos \theta)
\end{equation*}
\frbox{Compton Wellenlänge}{
\begin{equation*}
\lambda_C \le \frac{\hbar}{m_e c^2} = \frac{r_e}{\alpha_{\tx{em}}} = 39 \cdot 10^{-13 \, \tx{m}}
\end{equation*}}
\textbf{Wichtig:}
\begin{equation*}
E_{\gamma'}^{\tx{max}} (\theta = 0) = E_{\gamma}
\end{equation*}
\begin{equation*}
E_{\gamma'}^{\tx{min}} = \frac{E_{\gamma}}{1 + 2 \frac{E_{\gamma}}{m_e c^2}}
\end{equation*}
Wegen der Impulserhaltung gilt:
\begin{equation*}
\theta_{e}^{\tx{max}} \le \frac{\pi}{2}
\end{equation*}
Wirkungsquerschnitt (aus der Quantenelektrodynamik (QED))
\begin{equation*}
\sigma_{\tx{Compton}} \propto \frac{1}{E_{\gamma}} \cdot \ln \frac{2 E_{\gamma}}{m_e c^2}
\end{equation*}
Erzeugung hochenergetischer Photonen durch inverse Compton-Streuung.

\section{Paarbildung}

Paarbildung ist nur möglich in der Nähe eines Kerns (wegen Energie- und Impulserhaltung).

\subsection{Schwellen}

\begin{equation*}
E_{\gamma} > 2 m_e \approx 1{,}02 \, \tx{MeV} \quad \tx{im Kernfeld}
\end{equation*}
\begin{equation*}
E_{\gamma} > 4 m_e \approx 2{,}04 \, \tx{MeV} \quad \tx{im Elektronenfeld}
\end{equation*}
\begin{equation*}
\gamma + A \to e^- e^+ (A)
\end{equation*}


%T3
\hft


\begin{equation*}
\gamma + e^- \to e^- e^+ e^-
\end{equation*}
(Indent-Reaktion)


%T4
\hft


\begin{equation*}
\sigma_{\tx{Paar}} \propto \ln 183 Z^{-\nicefrac{1}{3}} \propto \frac{1}{L_{\tx{rad}}}
\end{equation*}
Insgesamt erhalten wir also für den Photoeffekt, die Compton-Streuung und die Paarbildung zusammen:
\begin{equation*}
\sigma_{\tx{tot}} \propto \sigma_{\tx{Photo}} + \sigma_{\tx{Compton}} + \sigma_{\tx{Paar}}
\end{equation*}
\begin{equation*}
\sigma_{\gamma} \propto c_1 Z^5 E^{\nicefrac{7}{2}} + c_2 Z \frac{1}{E} \ln E + c_3 Z^2
\end{equation*}
Minimum bei $ \mathcal{O} (10 \, \tx{MeV}) $\\
$ \Rightarrow $ große Reichweite !
