\setcounter{chapter}{-1}

\chapter{Einführung}

\section{Wichtige Infos}

\begin{description}
	\item[e-mail] \texttt{harals.ita@physik.uni-freiburg.de}
	\item[Zimmer] 803
	\item[Homepage] \texttt{www.qft.physik.uni-freiburg/Teaching}
	\item[Tutorate] 24. Aprill ab 14:00 Einschreibungsbeginn\\
	60\% sind zum bestehen der Studienleistung erforderlich. Die Teilnahme an der Prüfung ist nicht daran gebunden und kann auch ohne bestehen mitgeschrieben werden.
\end{description}

\section{Inhalt der Vorlesung}

Die Vorlesung orientiert sich stark am Script von Prof. Dittmeier.

\begin{description}
	\item[Funktionentheorie] Theorie der Funktionen in einer komplexen Veränderlichen
	\begin{enumerate}[1.]
		\item \textbf{Komplexe Zahlen}
		\begin{itemize}
			\item natürliche Zahlen $ \mathbb{N} = \{1, 2, \dots\} $ mit definierten Operatoren $ + $ und $ \times $
			\item ganze Zahlen $ \mathbb{Z} = \{0, \pm 1, \pm 2, \pm 3, \dots\} $ mit den Operationen $ + $ mit Inversion und $ \times $ ohne Inversion
			\item rationale Zahlen $ \mathbb{Q} = \left\{\frac{a}{b} \big| a, b \in \mathbb{Z}, b \neq 0 \right\} $ mit den Operatoren $ + $ und $ \times $ und ihren Inversionen\\[5pt]
			$ x^2 = z $ algebraisch unvollständig, konvergente Folge, die nicht in $ \mathbb{Q} $ liegenden Limes hat (Cauchy Folge \footnote{Mit einer Cauchy Folge kann gezeigt werden, dass eine Folge konvergiert, ohne dass der Limes bekannt ist.}).
			\item reelle Zahlen $ \mathbb{R} = \mathbb{Q} \cup \{ \tx{irrationale Zahlen} \} $. Vollständiger Körper \footnote{Bei einem vollständigen Körper liegen die Grenzwerte aller konvergenter Folgen wieder in dem Körper.} aber algebraisch nicht abgeschlossen.\\[5pt]
			$ x^2 = -1 $ nicht lösbar in $ \mathbb{R} $
			\item komplexe Zahlen $ \mathbb{C} = \mathbb{R}, i $ algebraisch abgeschlossen, vollständiger Körper\\[5pt]
			konstuktion über imaginäre Einheit $ i $ mit $ (i)^2 = (-1) $, Euler 1777
		\end{itemize}
		\textbf{Def: komplexe Zahlen}
		\begin{enumerate}[a)]
			\item komplexe Zahl $ z $ ist ein Zahlenpaar $ z = (x,y) $ mit $ x, y \in \mathbb{R} $. $ x $ ist der Realteil von $ z $ mit $ \Re(z) = x $ und $ y $ der Imaginärteil von $ z $ mit $ \Im(z) = y $.\\[10pt]
			Definieren wir zwei komplexe Zahlen $ z_1 = (x_1, y_1), z_2 = (x_2, y_2) $, so ist:\\
			die \textbf{Addition} definiert als:\\
			$$ z_1 + z_2 = (x_1 + x_2, y_1 + y_2) $$
			die \textbf{Multiplikation} definiert als:\\
			$$ z_1 \cdot z_2 = (x_1 x_2 + y_1 y_2, x_1 y_2 + x_2 y_1) $$
			\item Das Symbol der Menge der komplexen Zahlen ist $ \mathbb{C} $.
			\begin{equation*}
			\ol{\mathbb{C}} = \mathbb{C} \cup \{-\infty\}
			\end{equation*}
			\item Kurzschreibweise: $ i = (0, 1) $; $ z = (x,y) = x + i \cdot y $
			\item komplex konjugierte Zahl
			$$ z = (x,y) = x + i y \to \ol{z} = (x, -y) = x -  i y $$
			\item Betrag einer komplexen Zahl
			$$ |z| = \sqrt{z \cdot \ol{z}} = \sqrt{x^2 + y^2} $$
			\item Polardarstellung
			\begin{equation*}
			z = (r \cos \varphi, r \sin \varphi) = r \cos \varphi + i \cdot r \sin \varphi
			\end{equation*}
			\begin{equation*}
			\varphi \in (- \pi , \pi] \qquad r \in \mathbb{R}^+
			\end{equation*}
			$ r $ ist der \textbf{Betrag} von $ z $: $ r  = |z| $. $ \varphi $ ist das \textbf{Argument} von $ z $: $ \varphi = \tx{arg}(z) $
			
			% T1
		\end{enumerate}
		
		\subsection*{Satz: Rechenregeln im \texorpdfstring{$ \mathbb{C} $}{C}}
		
		für $ z_i \in \mathbb{C} $ gilt:
		\begin{equation*}
		\ol{z_1 + z_2} = \ol{z_1} + \ol{z_2} \qquad \qquad \ol{z_1 \cdot z_2} = \ol{z_1} \cdot \ol{z_2} \qquad \qquad \ol{\ol{z_1}} = z_1
		\end{equation*}
		\begin{equation*}
		\Re(z) = \frac{1}{2} ( z + \ol{z}) \qquad \qquad \Im(z) = \frac{1}{2i} ( z - \ol{z})
		\end{equation*}
		\begin{equation*}
		| z_1 z_2 | = | z_1 | | z_2 | \qquad | \ol{z} | = | z |
		\end{equation*}
		\begin{equation*}
		| z | \ge 0 \quad \tx{und} \quad | z | = 0 \quad \Rightarrow \quad z = (0,0) = 0 + i 0 = 0
		\end{equation*}
		\begin{equation*}
		| z_1 | + | z_2 | \ge | z_1 + z_2 | \ge | z_1 | - | z_2 |
		\end{equation*}
		
		%T2 
		
		\subsection*{Gaußsche Zahlenebene}
		
		$ \mathbb{C} $ bildet einen 2-dimensionale Vektorraum wie $ \mathbb{R}^2 $. Es gibt also eine gemeinsame Struktur mit dem $ \mathbb{R}^2 $, dennoch ist $ \mathbb{C} $ eine Erweiterung.
		\begin{enumerate}[a)]
			\item Vektoraddition, Multiplikation mt reeller zahl, Länge und Abstandsbegriff. 
			
			% T3
			
			% T4
			
			\item Multiplikation komplexer Zahlen $ \to $ Darstellung in Polarform
			\begin{align*}
			z_1 z_2 &= ( r_1 \cos \varphi_1 , r_1 \sin \varphi_1) \cdot (r_2 \cos \varphi_2 , r_2 \sin \varphi_2) \\
			& = r_1 r_2 \cdot (\cos \varphi_1 \cos \varphi_2 - \sin \varphi_1 \sin \varphi_2 , \cos \varphi_1 \sin \varphi_2 + \cos \varphi_2 \sin \varphi_1) \\
			& = r_1 r_2 \cdot (\cos(\varphi_1 + \varphi_2) , \sin(\varphi_1 + \varphi_2))
			\end{align*}
			$ \Rightarrow $ Beträge multiplizieren, Argumente addieren\\[5pt]
			Kehrbruch einer komplexen Zahl
			\begin{align*}
			\frac{1}{z} = \frac{\ol{z}}{z \ol{z}} & = \frac{1}{r^2} (r \cos \varphi, - r \sin \varphi) \\
			& = \frac{1}{r} ( \cos(- \varphi), \sin(- \varphi)
			\end{align*}
			mit $ r' = \frac{1}{r} $, $ \varphi' = - \varphi $
			
			% T5
			
			% T6
		\end{enumerate}
		
		\subsection*{Riemannsche Sphäre}
		
		Kompaktifizierung der komplexen Zahlen Ebene $ \mathbb{C} $ durch stereographische Projektion: $ \mathbb{\hat{C}} = \ol{\mathbb{C}} = \mathbb{C} \cup \{ -\infty \} $.\par
		Es wird also ein Punkt im unendlichen zu $ \mathbb{C} $ hinzugefügt.
		
		% T7
		
		% T8
		
		\noindent
		$ N = (0,0,1) $\\
		Sphäre mit Radius $ R = 1 $, um Koordinatenuhrsprung in $ \mathbb{R}^3 $. $ \mathbb{C} $ wird identifiziert mit der $ (x,y) $-Ebene.
		\begin{description}
			\item[stereographische Projektion] = Zuordnung von Punkten auf Sphäre mit Punkten in $ (x,y) $-Ebene.
		\end{description}
		Vorschrift: Gerade durch Punkt $ (x_{\Re}, y_{\Im}, 0) $ und den Nordpol $ N $. Durchstoßpunkt = projezierter Punkt auf Sphäre. Bildpunkte: $ \vec{w}(z) $.
		
		\subsection*{Def: Chordaler Abstand}
		
		$ \chi(z_1, z_2) = $ ,, Abstand der Bilder $ \vec{w}_1 = \vec{w}(z_1) $, $ \vec{w}_i = \vec{w}(z_i) $ unter stereographischen Projektion im $ \mathbb{R}^3 $.``
		\begin{equation*}
		\chi (z_1, z_2) = |\vec{w}(z_1) - \vec{w}(z_2)|
		\end{equation*}
		
		% T9
		
		% T10
	\end{enumerate}
\end{description}

