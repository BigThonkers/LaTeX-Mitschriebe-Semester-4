%18.06.19

\chapter{Darstellungen}

\textbf{Bisher:} Formulierung der QM in abstraktem Vektorraum, durch Zustände bzw. deren durch die Schrödingergleichung induzierte Dynamik, ohne jedoch die Frage zu stellen, wo das ,,Teilchen`` nun denn eigentlich sei (etwa im Orts- oder Impulsraum).\par
\textbf{Komplikationen:} Orts- und Impulsvariable jeweils kontinuierliche Größen, d.h. die assoziierten Observablen haben kontinuierliches Eigenspektrum.

\section{kontinuierliche Eigenwerte}

Wir erinnern uns zunächst an die Vollständigkeitsrelation \eqref{zerlegungeins} ($ \sum_{j}|j\rangle\langle j| $), insbesondere also
\begin{equation}
|j\rangle = \sum_{m} |m\rangle\langle m| j \rangle = \sum_{m} |m\rangle \delta_{mj} \qquad \forall |j\rangle \in \tx{ONB}
\label{4.1}
\end{equation}
Übergang zu kontinuierlichem Wertebereich von $ m $ legt folgende Verallgemeinerung nahe:
\begin{equation}
|q'\rangle = \int \dd q |q\rangle \langle q | q'\rangle
\label{4.2}
\end{equation}
\eqref{4.2} erfordert entsprechend die Normierung
\begin{equation}
\langle q|q'\rangle = \delta(q-q')
\label{4.3}
\end{equation}
anstelle von \eqref{2 7}, worin $ \delta(q-q') $ die Punktmasse, Diracsche Deltafunktion, allgemeiner ($ \to $ Mathematik, O.Forster, Analysis III) \textbf{Testfunktion} oder \textbf{Distribution} mit bekannten Eigenschaften (s.o. Skript (IV.4 - 11) o.z.B. Theo ED). Vermöge \eqref{4.3} liegen die mit unendlicher Norm ausgestattet $ |q\rangle $ nicht mehr in unserem Hilbertraum [$ \to $ konkretes Beispiel: ebene Wellen zu Wellenvektor $ \vec{k} $, s.u.]. Um demnach eine physikalisch konsistente Anschauung zu gewinnen, führt man mit $ |q\rangle $ assoziierte \textbf{Eigendifferentiale} ein [Funktionalanalysis $ \to $ Reed, Simon Band I].
\setcounter{equation}{11}
\begin{equation}
|q\rangle_{\Delta q} = \frac{1}{\sqrt{\Delta q}} \int_{q}^{q + \Delta q} |q'\rangle \dd q'
\label{4.12}
\end{equation}
,die folgende beruhigende Eigenschaft haben:
\begin{equation}
\begin{aligned}
_{\Delta q} \langle q | q \rangle_{\Delta q} &= \frac{1}{\Delta q} \int_{q}^{q + \Delta q} \dd q' \int_{q}^{q + \Delta q} \dd q'' \langle q'' | q' \rangle \\
&\overset{(\tx{IV}.3)}{=} \frac{1}{\Delta q} \int_{q}^{q + \Delta q} \dd q' \int_{q}^{q + \Delta q} \dd q'' \delta(q'' - q') \overset{(\tx{IV}.4)}{=} 1
\end{aligned}
\label{4.13}
\end{equation}
d.h. $ |q\rangle_{\Delta q} $ ist wieder ein normierbarer, näherungsweiser Eigenvektor zum Eigenwert $ q $ [$ \to $ s.u. Konstruktion von Wellenpaketen als normierbare Darstellung freier Teilchen]. Mit der so definierten Schar $ \{ |q\rangle \}_{q} $ von Basiszuständen lässt sich ein beliebiger Zustand wie folgt darstellen:
\begin{equation}
|\psi\rangle = \int \dd q c(q) |q\rangle
\label{4.14}
\end{equation}
Zunächst: Wahrscheinlichkeit für $ q = y $, im Zustand $ |\psi\rangle $:
\begin{equation}
\begin{aligned}
|\langle y|\psi\rangle|^2 &\overset{\eqref{4.14}}{=} \left|\int \dd q c(q) \langle y | q \rangle \right|^2 = \int \dd q' c^*(q') \langle q' | y \rangle \int \dd q c(q) \langle y | q \rangle \\
&\overset{\eqref{4.3}}{=} \int \int \dd q' \dd q c^*(q') c(q) \delta(q'-y) \delta(y - q) \overset{\int \dd q}{=} \int \dd q' c^*(q') c(y) \delta (q' - y) \\
&\hspace{3pt}\overset{\int \dd q'}{=} c^*(y) c(y) = |c(y)|^2\\
|\langle y|\psi\rangle|^2 \hspace{-8pt}&\hspace{8pt}= |c(y)|^2
\end{aligned}
\label{4.15}
\end{equation}
[Beachte \textbf{strikte} Analogie zum diskreten Fall!]\\
Entsprechend erhält man für die Norm von $ |\psi \rangle $ in der Basis $ \{ |q\rangle \} $ 
\begin{equation}
\langle \psi | \psi \rangle = \int \langle \psi | q \rangle \langle q | \psi \rangle \dd q = \int \ub{|\langle \psi | q \rangle |^2}_{\overset{\eqref{4.15}}{=} |c(q)|^2} \dd q = 1
\label{4.16}
\end{equation}
Somit gibt $ |\langle \psi | q \rangle |^2 \dd q $ die Wahrscheinlichkeit an, den Ort $ q $ im Intervall $ [q , q + \dd q] $ (Analog zum $ \Delta q $ oben) zu messen.

\begin{equation*}
\delta (q - q') = \frac{1}{2 \pi} \int e^{ik(q-q')} \dd k \tag{IV.4}
\end{equation*}
\begin{equation*}
l_2 \leftrightarrow \int \dd \mu (q) |c(q)|^2 < \infty
\end{equation*}
\begin{equation*}
L_p \leftrightarrow \int \dots |c(q)|^2
\end{equation*}
\begin{equation*}
\langle q | q' \rangle = \delta (q - q') \overset{(\tx{IV}.5)}{=} \frac{1}{2\pi} \int e^{i k (q - q')} \dd k = \frac{1}{2\pi} \int e^{i k q} e^{-i k q'} \dd k
\end{equation*}
Mit der de Broglie-Beziehung
\begin{equation}
p = \hbar k
\label{4.17}
\end{equation}
wird $ \langle q | q' \rangle $ zu
\begin{equation}
\langle q | q' \rangle = \frac{1}{2 \pi \hbar} \int e^{i p q / \hbar} e^{-u p q' / \hbar} \dd p
\label{4.18}
\end{equation}
Aufgrund der Interpretation von $ e^{i p q / \hbar} $ als (Ortsdarstellung) einer ebenen Welle mit Impuls $ p $ - vgl. ebene Welle mit Wellenvektor $ k $ in E-Dyn. identifizieren wir diese Größe als Ortsprojektion des Impulseigenzustandes $ |p\rangle $ zum Eigenwert p. D.h.
\begin{equation}
\langle q | p \rangle = \frac{1}{\sqrt{2 \pi \hbar}} e^{\nicefrac{i}{\hbar} p q}
\label{4.19}
\end{equation}
Die Darstellung quantenmechanischer Zustände $ |\psi\rangle \in \ham $ in der Ortsbasis, d.h. durch Projektion von $ |\psi\rangle $ auf die kontinuierlich verteilte Ortszustände $ |q\rangle $ heißt daher auch (historisch) \textbf{Wellenmechanik}.

%24.06.19

\section{Ortsdarstellung quantenmechanischer Zustände und Operatoren}

Analog zu obiger Verallgemeinerung der Zerlegung der $ \mathbbm{1} $ jetzt Verallgemeinerung des Spektralsatzes \eqref{spektralsatz} für Systeme mit kontinuierlichem Spektrum ($ \to $ s.a. Reed \& Simon Vol I for the mathematically more inclines).
\begin{equation}
Q \overset{\substack{\eqref{spektralsatz} \\ \eqref{4.2}}}{=} \int \dd q' \ q' | q' \rangle \langle q' |
\label{4.20}
\end{equation}
Entsprechend ergibt sich als Eigenwertgleichung
\begin{equation}
\begin{aligned}
Q | q\rangle  & \overset{\eqref{4.20}}{=} \int \dd q' \ q' |q'\rangle \ub{\langle q'|q\rangle}_{\overset{\eqref{4.3}}{=} \delta (q' - q)} \\
&= q | q\rangle
\end{aligned}
\label{4.21}
\end{equation}
Für die Matrixelemente von Potenzen bzw. verallgemeinert Potenzreihen von $ Q $ folgt weiter:
\begin{equation}
\begin{aligned}
\langle q | Q^{\nu} | q' \rangle &= q'^{\nu} \langle q | q' \rangle = q'^{\nu} \delta(q - q') \qquad \nu \in \mathbb{Z} \\
\langle q |f(Q) | q' \rangle &= \sum_{j} c_{\nu} q'^{\nu} \langle q| q' \rangle = f(q') \delta(q - q')
\end{aligned}
\label{4.22}
\end{equation}
für $ f(Q) = \sum_{\nu = - \infty}^{+ \infty} c_{\nu} Q^{\nu} $ (Konvergenz vorausgesetzt; $ f(Q) \equiv $ verallgemeinerte Potenzreihe, setzt Existenz von $ Q^{-1} $ voraus).\par
Ähnlich für die Matrixelemente des Impulsoperators $ P $:
\begin{equation}
\begin{aligned}
\langle q | P | q' \rangle &\overset{\eqref{4.2}}{=} \int  \dd p \ \langle q | P | p \rangle \langle p | q' \rangle \custo{\leftarrow}{=}{P | p \rangle = p | p \rangle} \int \dd p \ p \langle q | p \rangle \langle p | q' \rangle \\
&\overset{\eqref{4.18}}{=} \frac{1}{2 \pi \hbar} \int \dd p \ p e^{\tfrac{i}{\hbar} p (q - q')}\\
\langle q | P | q' \rangle &\overset{(IV.10)}{=} - i \hbar \prd{}{q} \delta(q - q')
\end{aligned}
\label{4.23}
\end{equation}
Matrixelemente ganzzahliger Potenzen $ \nu $ von $ P $:
\begin{equation}
\langle q | P^{\nu} | q' \rangle = \int \dd p \ p^{\nu} \langle q | p \rangle \langle p | q' \rangle = \frac{1}{2 \pi \hbar} \int \dd p \ p^{\nu} e^{\tfrac{i}{\hbar} p (q - q')}
\label{4.24}
\end{equation}
Speziell für $ \nu \ge 0 $, mit (IV.5), als Verallgemeinerung von (IV.10), folgt:
\begin{equation}
\langle q | P^{\nu} | q' \rangle = \left(- i \hbar \prd{}{q}\right)^{\nu} \delta(q - q') \quad \nu \ge 0 \tx{ ganzzahlig}
\label{4.25}
\end{equation}
Somit haben wir eine Darstellung von $ Q^{\nu} $ und $ P^{\nu} $ in der Ortsbasis gewonnen und können die Wirkung dieser Operatoren auf beliebige $ |\psi\rangle \in \ham $, dargestellt in der Ortsbasis, erschließen.
\begin{equation*}
\langle q | Q^{\nu} | \psi \rangle \overset{(*)}{=} \int \dd q' \langle q | Q^{\nu} | q' \rangle \langle q' | \psi \rangle = \int \dd q' \ q'^{\nu} \langle q | q' \rangle \langle q' | \psi \rangle = q^{\nu} \langle q|\psi\rangle
\end{equation*}
$ (*) $: Einsetzen der Eins, mit $ \langle q | \psi \rangle $ gegeben\\[5pt]
D.h.
\begin{equation}
\langle q | Q^{\nu} | \psi \rangle = q^{\nu} \langle q | \psi \rangle = q^{\nu} \psi(q)
\label{4.26}
\end{equation}
$ \psi(q) $ heißt ,,\textbf{Wellenfunktion}`` des Zustandes $ |\psi\rangle $. M.a.W. ist die Wellenfunktion die Amplitude des Zustandes $ |\psi \rangle $ am Ort $ q $, repräsentiert durch $ |q\rangle $.\par
Für $ P^{\nu} $ folgt entsprechend
\begin{equation}
\begin{aligned}
\langle q | P^{\nu} | \psi \rangle & \overset{(*)}{=} \int \dd q' \ \langle q | P^{\nu} | q' \rangle \langle q' | \psi \rangle\\
&= \int \dd q' \int \dd p \ \langle q | P^{\nu} | p \rangle \langle p | q' \rangle \langle q' | \psi \rangle = \int \int \dd q' \dd p \ p^{\nu} \langle q | p \rangle \langle p | q' \rangle \langle q' | \psi \rangle \\
&\overset{\eqref{4.19}}{=} \int \dd q' \ \langle q' | \psi \rangle \frac{1}{2 \pi \hbar} \int \dd p \ p^{\nu} e^{\tfrac{i}{\hbar} p ( q - q')} = \langle q | P^{\nu} | \psi \rangle
\end{aligned}
\label{2.27}
\end{equation}
Dies vereinfacht sich für $ \nu \ge 0 $ (s. \eqref{4.24}, \eqref{4.25})
\begin{equation}
\begin{aligned}
\langle q | P^{\nu} | \psi \rangle &\overset{\eqref{4.25}}{=} \int \dd q' \left(- i \hbar \prd{}{q}\right)^\nu \delta(q - q') \langle q' | \psi \rangle = \left(- i \hbar \prd{}{q}\right)^\nu \langle q| \psi \rangle\\
\langle q | P^{\nu} | \psi \rangle &\overset{\nu > 0}{=} \left(- i \hbar \prd{}{q}\right)^\nu \psi(q)
\end{aligned}
\label{4.28}
\end{equation}
Die Gleichungen \eqref{4.25} und \eqref{4.28} besagen insbesondere, dass Orts- und Impulsoperator im (Vektor-) Raum der Wellenfunktionen $ \langle q | \psi \rangle = \psi (q) $ als \textbf{Multiplikations-} bzw. \textbf{Differentialoperator} wirken; d.h.:
\begin{equation}
\begin{aligned}
\langle q | Q | \psi \rangle &\eqdef Q_{\{q\}} \langle q | \psi \rangle = Q_{\{q\}} \psi(q) = q \psi(q) \\
\langle q | P | \psi \rangle &\eqdef P_{\{q\}} \langle q | \psi \rangle = P_{\{q\}} \psi(q) = - i \hbar \prd{}{q} \psi(q)
\end{aligned}
\label{4.29}
\end{equation}
Wiederholte Anwendung, $ \nu \ge 0 $:
\begin{equation}
\begin{aligned}
Q^{\nu} \psi(q) &= q^{\nu} \psi(q) \\
P^{\nu} \psi(q) &= \left(- i \hbar \prd{}{q} \right)^{\nu} \psi(q)
\end{aligned}
\label{4.30}
\end{equation}
Entsprechend wird eine Hamiltonfunktion bzw. der assoziierte quantenmechanische Hamiltonoperator der Form:
\begin{equation}
H = \frac{p^2}{2m} + V(q)
\label{4.31}
\end{equation}
im Ortsraum die Darstellung
\begin{equation*}
H = - \frac{\hbar^2}{2m} \prd{^2}{q^2} + V(q)
\end{equation*}
haben, mit der zugehörigen Schrödingergleichung
\begin{equation}
H \psi(q,t) = i \hbar \partial_t \psi(q,t)
\label{4.32}
\end{equation}
[$ V(q) $ ist hier Multiplikationsoperator, mit $ V $ angewendet am Skalar $ q $]

\section{Impulsdarstellung}

$ \to $ wichtig für Festkörperühysik o.a. Skrenphysik\\
formal völlig analoges Vorgehen wie oben. Zwiedert ??, wegen \eqref{4.18}, \eqref{4.19}
\begin{equation}
\langle p | q \rangle = \frac{1}{\sqrt{2 \pi \hbar}} e^{-\tfrac{i}{\hbar} p q} = \ol{\langle p | q \rangle}
\label{4.33}
\end{equation}
Angenommen $ \langle q | \psi \rangle $ sein bekannt, so ergibt sich $ \langle p | \psi \rangle $ wie folgt, mittels Einsetzen der Eins (in $ \{ |q\rangle \} $)
\begin{equation}
%
%
%
\setcounter{equation}{35}
%
%
%
\begin{aligned}
\langle q | \psi \rangle &= \int \dd q \ \langle p | q \rangle \langle q | \psi \rangle \\
& \overset{\eqref{4.33}}{=} \frac{1}{\sqrt{2 \pi \hbar}} \int \dd q \ e^{- \tfrac{i}{\hbar} p q} \psi(q) \\ & \eqdef \hat{\psi}(p) \equiv \tx{Fouriertransformierte der Wellenfunktion}
\end{aligned}
\label{4.35}
\end{equation}
[$ \hat{\phantom{0}} $ oder alternativ $ \tilde{\phantom{0}} $, lässt man oft einfach weg, da das Argument von $ (\hat{\psi} = )\psi $ hinreichend klärt, in welcher Darstellung man arbeiten!]\\[5pt]
Analog zu \eqref{4.22} bis \eqref{4.32} erklärt man \eqref{4.37}:
\begin{equation}
\langle p | P | \psi \rangle = p \langle p|\psi\rangle \eqdef p \psi(p)
\label{4.36}
\end{equation}
\begin{equation}
\begin{aligned}
\langle q | Q | \psi \rangle &= \int \dd q \ \langle p | q \rangle \langle q | Q | \psi \rangle \\
&= \int \dd q \ q \langle p | q \rangle \langle q | \psi \rangle \overset{\eqref{4.33}}{=} \int \dd q \ i \hbar \prd{}{p} \langle p | q \rangle \langle q | \psi \rangle\\
&= i \hbar \prd{}{p} \int \dd q \langle p | q \rangle \langle q | \psi \rangle \overset{\eqref{4.35}}{=} i \hbar \prd{}{p} \psi(p)
\end{aligned}
\label{4.37}
\end{equation}
\begin{equation}
\begin{aligned}
P_{\{p\}} \langle p | \psi \rangle &= P_{\{p\}} \psi(p) = p \psi(p) \\
Q_{\{p\}} \langle p | \psi \rangle &= Q_{\{p\}} = i \hbar \prd{}{p} \psi(p)
\end{aligned}
\label{4.38}
\end{equation}
ganzzahlige Potenzen:
\begin{equation*}
\langle p | Q^{\nu} | \psi \rangle = Q_{\{p\}}^{\nu} \langle p | \psi \rangle = Q_{\{p\}}^{\nu} \psi(p)
\end{equation*}
mit:
\begin{equation*}
Q_{\{p\}}^{\nu} = \frac{1}{2 \pi \hbar} \int \dd p' \ocircle \int \dd q q^{\nu} e^{- \tfrac{i}{\hbar} (p - p')q}
\end{equation*}
mit $ \ocircle $ dem Argument der Operators $ Q_{\{p\}}^{\nu} $
\begin{equation}
\langle p | P^{\nu} | \psi \rangle = P_{\{p\}}^{\nu} \langle p | \psi \rangle = P_{\{p\}}^{\nu} \psi(p) \quad \tx{mit} \quad P_{\{p\}}^{\nu} = p^{\nu}
\label{4.39}
\end{equation}
Für $ \nu \ge 0 $ wird der Ausdruck für $ Q_{\{p\}}^{\nu} $ zu 
\begin{equation}
Q_{\{p\}}^{\nu} = \left(- i \hbar \prd{}{p}\right)^{\nu}
\label{4.40}
\end{equation}