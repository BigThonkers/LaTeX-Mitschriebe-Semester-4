
%20.05.19

\chapter{Teilchenidentifikation}

\subsubsection*{Programm Heute}

\begin{itemize}
	\item Cherenkov Strahlung
	\item Übergangsstrahlung
	\item Energiemessung
	\item Elektromagnetische Kalorimeter
	\item Hadronische Kalorimeter
\end{itemize}

\section{\texorpdfstring{\v{C}}{C}herenkov Strahlung}


%T1
%T2
\hft


\noindent
\v{C}herenkov Strahlung wird emittert wenn $ v > v_{\tx{Phase}} $ (Medium) also bei $ \beta > \frac{1}{n_{\tx{Medium}}} $.\\[5pt]
Der Abstrahlwinkel ist:
\begin{equation*}
\cos \theta = \frac{1}{\beta \cdot n} = \frac{v_{\tx{Phase} (\tx{Medium}) }}{v_{\tx{Teilchen}}} \Rightarrow \frac{c}{n} \cdot \frac{1}{v_{\tx{Teilchen}}}
\end{equation*}
mit $ n $ dem Brechungsindex im Medium.\\[5pt]
Der maximale Abstrahlwinkel $ \varangle $:
\begin{equation*}
\Theta^{\tx{max}} = \arccos \frac{1}{n}
\end{equation*}
Schwellenenergie $ E_s $ ab der \v{C}herenkov Strahlung auftritt
\begin{equation*}
\gamma_{s} = \frac{E_{s}}{\color{red} m \color{black} c^2} = \frac{1}{\sqrt{1 - \beta_{s}^{2}}} = \frac{1}{\sqrt{1 - \frac{1}{\color{red} n^2 \color{black}}}}
\end{equation*}
\folie{Leuchten eines Kernreaktors}\\
\folie{\v{C}herenkov Effekt}\\[5pt]
Anzahl emittierter Photonen
\begin{equation*}
\prd{\mu}{\lambda} = 2 \pi \alpha Z^2 \left(1 - \frac{1}{(p_n)^2}\right) \frac{\dd \lambda}{\lambda^2} \approx 2 \pi \alpha Z^2 \frac{\lambda_2 - \lambda_1}{\lambda_1 \lambda_2} \sin^2 \theta
\end{equation*}
$ n = n(\lambda) $\\
für $ [400 \, \tx{nm}, 700 \, \tx{nm}] $ $ \frac{\dd N}{\dd x} \approx \hfw $
\begin{equation*}
\left(\prd{E}{x}\right)_{\tx{\v{C}}} \approx 10^{-2} \left(\prd{E}{x'}\right)_{\tx{Ionis}}
\end{equation*}
als Radiator: alle transparente Stoffe: NaCl, Diamant, Bleiglas, H$ _{2} $O.\\[5pt]
\folie{\v{C}renenkov Winkel vs. Teilchengeschwinkdigkeit}\\
\folie{Photonenausbeute}\\
\folie{Verschiedene Typen}

\section{Übergangsstrahlung}

%T3
\hft


\noindent
Teilchen + Spiegelladung $ \Rightarrow $ veränderlicher Dipol $ \Leftrightarrow $ Übergangsstrahlung


\hfw

\noindent
Abstrahlungscharakteristik
\begin{equation*}
\prd{E}{\omega \dd \Omega} = \frac{h \alpha}{\pi^2} \beta^2 \cdot f(\theta)
\end{equation*}
$ \omega : $ Plasmafrequenz
\begin{enumerate}[a)]
	\item nicht relativistischer Fall $ f(\theta) = \sin^2 \theta $
	\item relativistischer Fall $ f(\theta) = \frac{\sin^2 \theta }{1 - \beta \cos ^2 \theta} $
	\item $ v = \frac{E}{m} \gg 1000 $
	u.s. Bereich der Röntgenstrahlung
\end{enumerate}

Polarisationsebene definiert durch
\begin{itemize}
	\item bewegte Ladung
	\item Abstrahlrichtung des Photonen
\end{itemize}
Bsp $ e^- $ mit $ E = 15 \, \tx{GeV} $, $ \gamma_e = 30000 $, $ \gamma_{\pi} = 110 $\\[5pt]
Wahrscheinlichster Abstrahlwinkel:
\begin{equation*}
\theta \approx \sqrt{\frac{1}{\gamma^2} + \frac{\omega_p}{\omega}} \approx \frac{1}{\gamma}
\end{equation*}
$ \Rightarrow $ \textbf{Kein Grenzwinkel!}
verstärkung des Effekts durch mehrfache Übergänge zwischen Medien.\\[5pt]
\folie{Winkelverteilung Übergangsstrahlungsstrahlung}\\
\folie{Übergangsstrahlungsdetektoren}

\section{Kalorimeter}

Aufgabe: Messung der Gesamtenergie in Abhängigkeit von der Bauweise
\begin{itemize}
	\item homogene Schauerzähler/Kalorimeter
	\item sampling Schauerzähler/Kalorimeter (Stichprobenmessung) $ \to $ Einfluss auf die Auflösung bei der Energiemessung
\end{itemize}
\begin{enumerate}[a)]
	\item Elektron-Photon Schauer\\
	Einfache Abschätzung
	\begin{equation*}
	N_{e^{\pm}, \gamma} \approx 2^t \qquad E(t) \approx \frac{E_0}{2^t}
	\end{equation*}
	$ t : $ konstante Zeit / Eindringtiefe Schnitte\\
	$ x_0 = \frac{1}{C_{\tx{RAD}}} $ Strahlungslänge
	\begin{itemize}
		\item Elektronen: $ 1 - \frac{1}{e} \approx 63\% $ der Energie wird durch Abstrahlung in Photonen abgegeben
		\item Photonen $ 1 - \frac{1}{e^{7/9}} \approx 54\% $ Intensität geht durch $ e^+, e^- $ Paarbildung verloren
	\end{itemize}
	$ t_{\tx{max}} = \frac{\ln\nicefrac{E_0}{E_{\tx{Krit}}}}{\ln2} \approx 10{,}5 X_0 $\\
	$ N_{\tx{max}} = \frac{E_0}{E_{\tx{Krit}}} \approx 1400 $ für $ Z = 82 $\\
	genau auf $ O(x3, \dots, x5) $ genauer mit EGS GEANT
	\begin{equation*}
	\prd{E}{t} \propto t^a e^{-bt}
	\end{equation*}
	mit $ t = \frac{x}{x_0} $ (Anzahl Strahlungslängen)
	
	\hfw
	
	
	\folie{Bremsstrahlung (Bethe-Heitler)}\\
	\folie{Naives Schauerbild}\\
	\folie{Longitudinal und Transverse Schauer Profile}\\
	\folie{Longitudinale Schauerentwicklung}
\end{enumerate}