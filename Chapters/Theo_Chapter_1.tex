\chapter{Quantenmechanik - Intro}

Quantenmechanik (QM) beschreibt den Mikrokosmos (im Gegensatz zum Makrokosmos).

\noindent
$ \rightarrow $ im CD-Player\\
$ \rightarrow $ im Handy\\
$ \rightarrow $ Kernspin\\
$ \rightarrow $ Zeitstandards\\
%\begin{enumerate}[$ \rightarrow $]
%	\item im CD-Player
%	\item im Handy
%	\item Kernspin
%	\item Zeitstandards
%\end{enumerate}

\noindent
QM ist ,,merkwürdig`` insofern, als anthropomorpher Anschauung unangepasst. $ \Rightarrow $ Sie sorgt noch heute für hitzige und kontroverse Debatten.

\begin{itemize}
	\item[$ \rightarrow $] siehe Podcasts PI - Kolloquium, z.B. Nicoles Gisin 15.04.2009, Reinhard Werner 24.12.2007
	\item[$ \rightarrow $] mathematischer Rahmen relativ einfach, doch Interpretation schwierig\\
	$ \Rightarrow $ Feynman: ,,Shut up and calculate{!}``
\end{itemize}
\textbf{Historische Genese:} Wie (fast?) alle physikalischen Theorien aus experimenteller Evidenz, die mit der,,klassischen`` Theorie nicht vereinbar war.\\[5pt]
Aus \textbf{theoretischer ,,Notlage`` angesichts bestehender Experimente:}\\
Balmer-Linien (1885), Franck-Hertz-Versuch (1913), Photoeffekt (Hallweds 1888 \& Einstein 1905), Schwarzkörperspektrum (Planck 1900), Compton-Effekt (1921), Kernspaltung (Halm, Meitner und Strassmann 1939), Stern-Gerlach-Versuch (1921).\par
\textbf{Große Namen:} N. Bohr, W. Heisenberg, E. Schrödinger, M. Born, John v. Neumann, A. Sommerfeld, L. de Broglie, P. Dirac, W. Pauli, L. Szil\'ard, R. Oppenheimer, Gamow, Siegelt, Hellmann, Etore Majorana.\par
Zu Majorana: Leonardo Sciascia: La Scomparsa di Majorana (Das Verschwinden des Majorana).\par
\textbf{Buchempfehlung:} Richard Rhodes: Die Atombombe oder die Geschichte des 8. Schöpfungstages\par
Weitere Quantenmechaniker: E. Teller, A. Sacharov, L. Landau, J. Belt, M. Gutzwiller.\\[10pt]

\noindent
\textbf{Korrespondenzprinzip:} Wie korrespondieren die QM-Theorien mit den klassischen Theorien? Wie sieht der Übergang vom diskreten zu einem kontinuierlichen Spektrum aus?\\[5pt]
\textbf{Beispiel:} Atommodell mit quantisierten Elektronen-Orbitalen von Bohr und dem Klassischeren Modell von Rutherford und kontinuierlichen Kepler-Orbitalen.\\
Die Energieniveaus eines Wasserstoff Atoms sind: $ E = \frac{1}{2n^2} $. Daraus folgt, dass höhere Energieniveaus immer näher aneinander liegen. Die Einergiedifferenzen $ E_{n+1} - E_{n} \sim \hbar \omega_{\tx{Kepler}} $ werden also immer geringer. Die Umlauffrequenz kann also mit zunehmender Hauptquantenzahl immer genauer bestimmbar.
