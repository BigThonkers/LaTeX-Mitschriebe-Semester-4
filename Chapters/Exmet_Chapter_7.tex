\setcounter{chapter}{6}

\chapter{Stichproben und Schätzwerte}

Die \textbf{Grundgesamtheit} entspricht allen (unendlich vielen) Werten. Die \textbf{Stichprobe} 1 und die Stichprobe 2 sind Teilmengen davon.\\[5pt]
Stichprobenvektor $ \vec{X} (x_1, x_2, x_3, \dots, x_n) $ entspricht der Durchführung eines Experiments.\\[5pt]
Die Wahrscheinlichkeitsdichte ist:
\begin{equation*}
f(X) = f(x_1) \cdot f(x_2) \cdot \dots \cdot f(x_n)
\end{equation*}
Die Aufgabe ist es dann die Wahrscheinlichkeitsdichte zu beschreiben mit
\begin{equation*}
f(X, \theta)
\end{equation*}
mittels einer geeigneten Schätzung.\\[5pt]
$ \theta $: unbekannter wahrer Parameter\\
$ \hat{\theta} $: Schätzwert von $ \theta $ (Achtung: ebenfalls Zufallsvariable).

\subsection{Def: Konsistenz}

\begin{equation*}
\lim_{n \to \infty} P \left(|\hat{\theta} - \theta| \ge \epsilon\right) = 0
\end{equation*}
$ \Rightarrow $ d.h. Messung ist sinnvoll. Konsistenz ist Mindestanforderung an guten Schätzwert.\\
Schätzwert finden: Parameter Anpassen

\subsection{Def: Bias (Versatz) des Schätzwerts}

Erwartungswert für Schätzwert $ \hat{\theta} $:
\begin{equation*}
E[\hat{\theta}(x)] = \int \int \int \hat{\theta}(x) f(x_1) f(x_2) f(x_3) \cdot \dots \cdot f(x_n) \dd x_1 \dots \dd x_n
\end{equation*}
Der Bias ist dann:
\begin{equation*}
b = E[\hat{\theta}] - \theta
\end{equation*}
Asymptotische Erwartungstreue bedeutet:
\begin{equation*}
\lim_{n \to \infty} b \to 0
\end{equation*}
$ \widehat{=} $ \textbf{``unbiased''} Messung

\section{Arithmetisches Mittel}

\subsection{Def:}


\begin{equation*}
\bar{x} = \frac{1}{n} \sum_{i = 1}^{n}x_i
\end{equation*}
$ \bar{x} $: Stichprobenmittel $ \neq $ Erwartungswert $ E[X] = \mu $\\
Gesetz der großen Zahlen $ n \to \infty $ $ \Rightarrow $ $ \bar{x} \to \mu $
\begin{equation*}
E[X] = E\left[\frac{1}{n} \sum x_i\right] = \frac{1}{n} \sum \mu_i = \mu
\end{equation*}
d.h. Mittelwert ist ohne Bias.

\subsection{Def: Varianz der Stichprobe}

\begin{equation*}
s^2 = \frac{1}{n-1} \sum (x_i - \bar{x})^2
\end{equation*}
Faktor $ \frac{1}{n-1} $ so gewählt, dass $ E[s^2] = \sigma^2 $\\[10pt]
Analog für Wertepaare\\
Kovarianz
\begin{align*}
V_{xy} &= \frac{1}{n-1} \sum (x_i - \bar{x}) (y_i - \bar{y}) \\
&= \frac{n}{n-1} (\ol{xy} - \bar{x} \bar{y})
\end{align*}

\section{Anmerkungen zu Unsicherheit}

\begin{enumerate}[a)]
	\item \textbf{Statistische Unsicherheit}
	\begin{itemize}
		\item statistische Fluktuationen
		\item zugrundeliegende Verteilung falsch angenommen
		\item Messbedingungen identisch?
		\item Weglassen von Datenpunkten
	\end{itemize}
	\textbf{Beispiel:} Messung mit Untergrund:
	
	%T1
	\hft
	
	\noindent
	gegeben: Messgröße $ \bar{x}_s = E[x_s] $\\[5pt]
	Sei Erwartungswert der Ereignisklasse ,,Signal`` mit Wahrscheinlichkeitsverteilung $ f_s(x) $ und sei nicht trennbar von Untergrund. Der Anteil des Untergrunds ist $ \alpha \pm \delta \alpha $.\\[10pt]
	\textbf{Lösung:} Unterteilen der Signalregion in $ 5 \sigma $ weg vom Signal $ \to $ 2 Messreihen
	\begin{enumerate}[1.)]
		\item Bestimme
		\begin{equation*}
		\bar{x}_n = E[x_n] = \frac{1}{n} \sum_{i=1}^{n} x_i
		\end{equation*}
		in Signalfreier Region.\\
		Varianz:
		\begin{equation*}
		\sigma^2(\bar{x}_n) = \frac{\sum x_i^2 - m \bar{x}_n^2}{m (m - 1)} = \frac{s_n^2}{m}
		\end{equation*}
		\item Bestimme
		\begin{equation*}
		\bar{x}_n = E[x_n] = \frac{1}{n} \sum_{i=1}^{n} x_i
		\end{equation*}
		mit Varianz
		\begin{equation*}
		\sigma^2(\bar{x}) = \frac{\sum x_i^2 - n \bar{x}_n^2}{n (n - 1)} = \frac{s^2}{n}
		\end{equation*}
	\end{enumerate}
	Wahrscheinlichkeitsverteilung in Signalregion:
	\begin{equation*}
	f(x) = (1-\alpha) f_s(x) + \alpha f_u(x)
	\end{equation*}
	\begin{equation*}
	E[x] = (1-\alpha) E[x_s] + \alpha E[x_n]
	\end{equation*}
	\begin{equation*}
	\bar{x}_s = \frac{1}{1-\alpha} \bar{x} - \frac{1}{1-\alpha} \bar{x}_n \pm d
	\end{equation*}
	\begin{equation*}
	d^2 \frac{1}{(1-\alpha)^2} \frac{s^2}{n} + \frac{\alpha^2}{(1-\alpha)^2} \frac{s_n}{m} + \left(\frac{\bar{x} - x_u}{(1-\alpha)^2}\right)^2 \delta \alpha^2
	\end{equation*}
	\item \textbf{Systematische Unsicherheiten}\\[5pt]
	Bsp.: Metermaß falsch skaliert. Messungen werden in gleicher Weise beeinträchtigt\\
	$ \Rightarrow $ Messungen untereinander konsistent also falsch. können zu Bias führen.\\[5pt]
	Unabhängige Systematische Unsicherheiten
	\begin{equation*}
	\sigma_{\tx{sys}}^2 = \sigma_{\tx{sys} 1}^2 + \sigma_{\tx{sys} 2}^2 + \dots
	\end{equation*}
	Am besten immer den statistischen und systematischen Fehler (Unsicherheit) getrennt angeben. z.B.:
	\begin{equation*}
	l = (1 \pm 0{,}5(\tx{stat}) \pm 0{,}1 (\tx{syst}))
	\end{equation*}
\end{enumerate}