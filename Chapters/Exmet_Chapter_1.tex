\chapter{Wechselwirkung geladener Teilchen mit Materie}

Nachweis durch Wirkung des Teilchens auf die Materie

\begin{itemize}
	\item Ionisation, Szintillation
	\item \v{C}evenkov-, Übergangsstrahlung
	\item Rückstoß
\end{itemize}
$ \Rightarrow $ Teilcheneigenschaften verändert
\begin{itemize}
	\item Energieverlust
	\item Richtungsänderung
	\item Identitätsverlust
\end{itemize}

\section{Klassische Betrachtung der Rutherfordstreuung}

\begin{itemize}
	\item stimmt mit QM in niederster Ordnung überein\\[5pt]
	solange: ,,schwere Teilchen``\\
	$ v \gg v_{e \tx{ in Hülle}} $\\
	$ \Delta E \gg \tx{Bindungsenergie von } e^- $
\end{itemize}

%T1
\hft

Typisches Beispiel:
\begin{equation*}
\mu^+ + \tx{Atom} \to \mu^+ + (\tx{Atom} + e^-)
\end{equation*}
Coulomb-Kraft
\begin{equation*}
F_{\parallel}(x) = F_{\parallel}(-x)
\end{equation*}
\begin{equation*}
F_{\perp} = \frac{1}{4 \pi \epsilon_0} \frac{z \cdot e \cdot Z \cdot e}{r^2} \frac{b}{|\vec{r}|}
\end{equation*}
Impulsübertrag
\begin{equation*}
\Delta \rho_{T} = \int_{- \infty}^{\infty} F_{\perp} \dd f = \frac{e^2}{4 \pi \epsilon_0} \cdot \frac{2 Z \cdot z}{\beta c b}
\end{equation*}
$ \beta = \frac{v}{c} $
\lcom{Mehr zum Thema und die genaue Rechnung findet man im Lehrbuch von Jackson.}\\
\textbf{Energieübertrag}\\
\folie{Energieverlust: klassisch nach Bohr}
\begin{equation*}
\Delta E = \frac{\Delta \rho_{T}^2}{2 M} = \frac{e^4}{(4 \pi \epsilon_0)^2} \cdot \frac{Z^2 z^2}{M \beta^2 c^2 b^2} \propto \frac{1}{b^2}
\end{equation*}
bei Kohärenter Streuung
\begin{equation*}
\frac{\Delta E \tx{ Elektronenhülle}}{\Delta E \tx{ Kern}} = \frac{2 m_p}{m_e} \approx 4000
\end{equation*}
Hülle: $ M = Z \cdot m_e $\\
Kern: $ M = A \cdot m_p = 2 Z \cdot m_p $\\
\ribox{$ \Rightarrow $ Die Streuung am Kern ist vernachlässigbar}\\
Der gesamte (mittlere) Energieverlust ist dann:
\begin{align*}
\langle\dd E\rangle &= \int \Delta E \cdot \ub{2 \pi b \ \dd b}_{\mathclap{\tx{Volumenelement}}} \ \cdot Z \cdot \ub{\frac{\rho \cdot N_A}{A}}_{= n_e} \dd x
\end{align*}
\frbox{Bethe-Bloch Beziehung}{
\begin{align*}
\left\langle \frac{\dd E}{\dd x} \right\rangle &= D \cdot \ub{\frac{Z \cdot \rho}{A}}_{\mathclap{\tx{Medium}}} \ \cdot \ \ub{\left(\frac{z}{\beta}\right)^2}_{\mathclap{\tx{Projektil}}} \cdot \ub{\ln\left(\frac{b_{\tx{max}}}{b_{\tx{min}}}\right)}_{\frac{1}{2} \ln \left( \frac{2 m_e c^2 \gamma^2 \beta^2}{I} T_{\tx{max}}\right) } \\
&= D \cdot \ub{\frac{Z \cdot \rho}{A}}_{\mathclap{\tx{Medium}}} \ \cdot \ \ub{\left(\frac{z}{\beta}\right)^2}_{\mathclap{\tx{Projektil}}} \cdot \ \frac{1}{2} \ln \left( \frac{2 m_e c^2 \gamma^2 \beta^2}{I} T_{\tx{max}} \right)
\end{align*}}
\noindent
mit $ I = \hbar \omega $: Ionisationspotential des Streuzentrums\\
und $ T_{\tx{max}} $: der Energie des $ e^- $ tragen kann\\[5pt]
\folie{Energieverlust}\\
\folie{Mittlerer Energieverlust nach Bethe Bloch}\\
\folie{Relativistischer Anstieg}\\
\folie{Materialabhängigkeit des mittleren Energieverlusts}\\
\folie{Minimaler Energieverlust}\\
\folie{Abhängigkeit vom Ionisationspotential}\\
\folie{Reichweite von Teilchen in Materie}\\
\folie{Bragg-Kurve} (Einstrahl-Tiefe in einen Menschen)\\
\folie{Anwendung Teilchenidentifizierung}\\
\folie{Energieverlust von Teilchen durch Ionisation}

% 6.05.19

\section{Energieverlust von Elektronen \texorpdfstring{$ e^- $}{e-} und Positronen \texorpdfstring{$ e^+ $}{e+}}

Bremsstrahlung führt zu zusätzlichem Energieverlust.
\begin{equation*}
E_{K} \approx \frac{600 \dots 700}{Z} \, \tx{MeV} \qquad \textbf{kritische Energie}
\end{equation*}
$ Z $ des Materials. Unterschiede zwischen fest, flüssig, gasförmig.
\begin{center}
	\begin{tabular}{l|c}
		Material & $ E_K $ \\
		\hline
		Luft: & 84,0 MeV\\
		Pb: & 7,4 MeV
	\end{tabular}
\end{center}
\begin{equation*}
\prd{E}{x} \bigg|_{\tx{Brems}} \propto \frac{Z^2}{m^2} \ \ \begin{array}{l} \tx{Target} \\ \tx{Projektil} \end{array}
\end{equation*}
Bremsstrahlung wichtig für $ e^{\pm} $
\begin{equation*}
\frac{m_\mu^2}{m_e^2} \left(\frac{100}{0{,}5}\right)^2 = 40000
\end{equation*}
(Eigentlich 105 statt 100)\\[5pt]
Bremsstrahlung führt zu
\begin{align*}
\prd{E}{x} &= E_e \cdot 4 \alpha r_e^2 N_A \frac{\rho Z}{A} \left\{ \ln \frac{183}{Z^{\nicefrac{1}{3}}} + \frac{1}{18} - f(z) \right\}\\
f(z) &= \alpha Z \left\{  \frac{1}{1 + \alpha^2 Z^2} + 0.2 + \mathcal{O}(\alpha Z^2)\right\}
\end{align*}
$ \alpha $: gemessene Konstante $ \alpha = 5{,}3 $ für H $ \big| 3 $ Pb

\subsection{Strahlungslänge}

\begin{equation*}
\frac{1}{L_{\tx{rad}}} = 4 \alpha r_e^2 N_A \frac{\rho Z}{A} \left\{ \ln \frac{183}{Z^{\nicefrac{1}{3}}} + \frac{1}{18} - f(z) \right\} = \prd{E}{x} \cdot \frac{1}{E_e}
\end{equation*}
(Die Formale stammt von Bether Heitler).\\
Die Strahlungslänge ist die Distanz, in der die $ e^{\pm} $ den Bruchteil $ (1 - \nicefrac{1}{e}) $ der Energie durch Bremsstrahlung verlieren.
\begin{equation*}
\prd{E}{x} \bigg|_{\tx{Brems}} = \frac{E_e}{L_{\tx{rad}}}
\end{equation*}