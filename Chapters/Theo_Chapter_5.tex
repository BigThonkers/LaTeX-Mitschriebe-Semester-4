\chapter{Zustände und Dynamik in einfachen Potentialproblemen}

Die einfachste. aber bereits experimentell relevante physikalische Szenarien; Einzelteilchendynamik in Potentialtöpfen (quantumdots $ \equiv $ Quantenpunkt; Quantendrähte - experimentell unter guter Kontrolle seit ca. Mitte 80'er des letzten Jahrhunderts $ \to $ mesoskopische Physik, Nanophysik, Elektronenbillard).


%T3
\hft Stadion für Elektronenbillard

\section{Gebundene Zustände im eindimensionalen Kastenpotential}

Gegeben sei:
\begin{equation}
V(x) = \casess{0}{-a \le x \le a}{V_0 < \infty}{x < -a  \ \ \vee \ \  x > a}
\label{5.1}
\end{equation}
Die zugehörige Eigenwertgleichung:
\begin{equation}
H | \psi \rangle = E | \psi \rangle
\label{5.2}
\end{equation}
für den Energie- bzw. Hamiltonoperator lautet in der Ortsdarstellung wegen \eqref{4.32}:
\begin{equation}
\left[\prd{^2}{x^2} + \frac{2m}{\hbar^2} (E - V(x))\right] \psi(x) = 0
\label{5.3}
\end{equation}

%T4
\hft Potential außerhalb von $ -a \le x \le a $

\noindent
Zunächst Eigenwertgleichung in den Bereichen $ I, II, III $.
\begin{equation}
\begin{aligned}
(I) \ \, \qquad \qquad x < -a \qquad &\left(\prd{^2}{x^2} - \kappa^2\right)  \psi_{I}(x) = 0 \\
(II) \quad -a \le x \le -a \qquad &\left(\prd{^2}{x^2} - k^2\right)  \psi_{II}(x) = 0 \\
(III) \ \ \qquad \qquad x > a \qquad &\left(\prd{^2}{x^2} - \kappa^2\right)  \psi_{III}(x) = 0 \\
\end{aligned}
\label{5.4}
\end{equation}
wobei
\begin{equation}
\kappa^2 = \frac{2 m (V_0 - E)}{\hbar^2} \qquad k^2 = \frac{2 m E}{\hbar^2}
\label{5.5}
\end{equation}
\begin{equation}
\tx{ Gebundene Zustände für } \quad E < V_0
\label{5.6}
\end{equation}
legen den Ansatz
\begin{equation}
\begin{aligned}
\psi_{I}(x) &= A_1 e^{\kappa x} \qquad x < -a \\
\psi_{II}(x) &= A_2 e^{i k x} + B_2 e^{- i k x} \qquad - a \le x \le a \\
\psi_{III}(x) &= B_3 e^{- \kappa x} \qquad x > a
\end{aligned}
\label{5.7}
\end{equation}
[Terme in $ e^{-\kappa x} $ bzw. $ e^{+\kappa x} $ unterdrückt, per $ B_1 = A_3 \defeq 0 $, wegen Normierbarkeit von $ \psi $]\\[5pt]
Zur Bestimmung der Koeffizienten $ A_j $, $ B_j $ in \eqref{5.7} fordern wir die Stetigkeit von $ \psi(x) $ und ihrer Ableitung $ \psi'(x) $ in $ x = \pm a $. Die Forderung an $ \psi'(x) $ ergibt sich aus der Kontinuitätsgleichung der \textbf{quantenmechanischen Wahrscheinlichkeitsstromdichte} $ \vec{j} $.
\begin{equation*}
\begin{aligned}
i \hbar \prt{}{t} |\psi(x)|^2 &= i \hbar \prd{}{t} (\psi^* \psi) = i \hbar (\partial_t \psi^*)\psi + i\hbar \psi^* \partial_t \psi \\
&= - \psi H \psi^* + \psi^* H \psi \overset{\eqref{4.32}}{=} - \psi \left(- \frac{\hbar^2}{2m} \Delta + V \right) \psi^* + \psi^* \left(- \frac{\hbar^2}{2m} \Delta + V \right) \psi \\
&\overset{(*)}{=} - \frac{\hbar^2}{2m} \left(\psi^* \Delta \psi -  \psi \Delta \psi^* \right) \overset{\Delta = \vabla^2}{=} \vabla \frac{\hbar^2}{2m} \left(\psi \vabla \psi^* -  \psi^* \vabla \psi \right)
\end{aligned}
\end{equation*}
$ (*) $: $ V $ Multiplikationsoperator\\[5pt]
Ergo:
\begin{equation}
\partial_t |\psi|^2 + \vabla \left[ \frac{\hbar}{2mi} \left(\psi^* \vabla \psi -  \psi \vabla \psi^*\right)\right] = 0
\label{5.8}
\end{equation}
\frbox{QM Stromdichte}{
\begin{equation}
\vec{j}(x) \defeq \frac{\hbar}{2 m i} \left(\psi^* \vabla \psi -  \psi \vabla \psi^*\right)
\label{5.9}
\end{equation}}

%01.07.19

\noindent
Aus der Stetigkeitsforderung an die Wellenfunktion und ihre erste Ableitung (wg. Kontinuitätsgleichung \eqref{5.8}) in $ x = \pm a $ folgt mit \eqref{5.7}:
\begin{equation}
\begin{pmatrix}
-e^{-\kappa a} & e^{-ik a} & e^{i k a} & 0 \\
- \kappa e^{- \kappa a} & i k e^{- i k a} & - i k e^{i k a} & 0 \\
0 & e^{i k a} & e^{- i k a} & - e^{- \kappa a} \\
0 & i k e^{i k a} & - i k e^{- i k a} & \kappa e^{- \kappa a}
\end{pmatrix} \begin{pmatrix}
A_1 \\ A_2 \\ B_2 \\ B_3
\end{pmatrix}
\label{5.10}
\end{equation}
Für nichttriviale Lösung muss Koeffizienten-Determinante verschwinden. Dies führt auf:
\begin{equation}
e^{- 2 \kappa a} \left[(\kappa^2 - k ^2) \sin(2 k a) + 2 \kappa k \cos(2 k a)\right] \overset{!}{=} 0
\label{5.11}
\end{equation}
$ V_0 < \infty $ (wg. \eqref{5.1}) $ \overset{\eqref{5.5}}{\Rightarrow} \kappa^2 < \infty \Rightarrow e^{-2 \kappa a} \neq 0 $, zusätzlich mit $ \sin(2 k a) \neq 0 $ (o.E.d.A (!)), führt auf
\begin{equation*}
\kappa^2 - k^2 + 2 \kappa k \cot(2 k a) = 0
\end{equation*}
Mit binomischer Formel für $ \kappa ^2 + \dots $ folgen die zwei Lösungen
%
%
%
\setcounter{equation}{13}
%
%
%
\begin{equation}
\kappa_1 = k \tan(k a) \qquad \kappa_2 = - l \cot(k a)
\label{5.14}
\end{equation}
Einsetzen von \eqref{5.14} in \eqref{5.10}, speziell $ \kappa_1 = k \frac{\sin(ka)}{\cos(ka)} $; in \eqref{5.10} zunächst 2. Zeile:
\begin{equation*}
A_1 \kappa e^{- \kappa a} \overset{\eqref{5.14}}{=} A_1 k \frac{\sin(ka)}{\cos(ka)} e^{- \kappa a} = A_2 i k e^{-ika} - B_2 i k e^{ika}
\end{equation*}
\begin{equation*}
A_1 e^{- \kappa a} \sin(ka) = A_2 i \cos/k a) e^{- i k a} - B2 i \cos(k a) e^{i k a} \tag{A}
\end{equation*}
1. Zeile aus \eqref{5.10}:
\begin{equation*}
A_1 e^{- \kappa a} = A_2 e^{i k a} + B_2 e^{i k a}
\end{equation*}
\begin{equation*}
A_1 \sin(k a) e^{- \kappa a} = A_2 \sin(ka) e^{-ika} + B_2 \sin(ka) e^{ika} \tag{B}
\end{equation*}
Gleichungen (A) und (B) gleichsetzen liefert:
\begin{equation*}
A_2 i \cos(ka) e^{-ika} - B_2 i \cos(ka) e^{ika} = A_2 \sin(ka) e^{ika} + B:2 \sin(ka) e^{ika}
\end{equation*}
\begin{equation*}
A_2 e^{-ika} (- \sin(ka) + i \cos(ka)) = B_2 e^{ika} ( \sin(ka) + i \cos(ka)
\end{equation*}
\begin{equation*}
A_2 e^{-ika} i \ub{\left(\cos(ka) + i \sin(ka)\right)}_{e^{ika}} = B_2 e^{ika} i \ub{\left(\cos(ka) - i \sin(ka)\right)}_{e^{-ika}}
\end{equation*}
\begin{equation*}
A_2 = B_2
\end{equation*}
insgesamt folgt
\begin{equation}
\begin{aligned}
B_2^{(1)} &= A_2^{(1)} \\
A_2^{(1)} &= A_1^{(1)} \\
2 A_2^{(1)} \cos(ka) &= A_1^{(1)} e^{-\kappa_1 a} \\
B_2^{(2)} &= - A_2^{(2)} \\
B_3^{(2)} &= - A_1^{(2)} \\
2 i A_2^{(2)} \sin(ka) &= - A_1^{(2)} e^{-\kappa_2 a}
\end{aligned}
\label{5.15}
\end{equation}
für $ \kappa_1 $ bzw. $ \kappa_2 $\\
\eqref{5.15} in \eqref{5.7} liefert die Lösungen für $ \kappa_1 $:
\begin{equation*}
\psi^{(1)} (x) = \left\{ \begin{array}{l}
A_1^{(1)} e^{\kappa_1 x} \\
A_1^{(1)} e^{- \kappa_1 a} \frac{\cos(kx)}{\cos(ka)} \\
A_1^{(1)} e^{- \kappa_1 x}
\end{array} \right\} = \psi^{(1)} (-x) \qquad \tx{,,gerade`` Eigenfunktionen}
\end{equation*}
und für $ \kappa_2 $
\begin{equation*}
\psi^{(2)} (x) = \left\{ \begin{array}{l}
A_1^{(2)} e^{\kappa_2 x} \\
- A_1^{(2)} e^{- \kappa_2 a} \frac{\sin(kx)}{\sin(ka)} \\
- A_1^{(2)} e^{- \kappa_2 x}
\end{array} \right\} = - \psi^{(2)} (-x) \qquad \tx{,,ungerade`` Eigenfunktionen}
\end{equation*}
Da $ \psi^{(1,2)} (x) $ wegen ihrer Asymptotik bzw. Beschränktheit auf $ [-a, +a] $ integrabel, lässt sich $ A_1^{(1,2)} $ per Normierungsbedingung:
%
%
%
\setcounter{equation}{16}
%
%
%
\begin{equation}
\int_{-\infty}^{\infty} |\psi^{(1,2)}(x)|^2 \dd x = 1
\label{5.17}
\end{equation}
gewinnen.\par
Eigenwerte zu \eqref{5.2}, \eqref{5.3} und (V.16) sind noch codiert durch \eqref{5.5} und \eqref{5.14}, lassen sich sber durch folgende Überlegung graphisch auffinden: wegen \eqref{5.5}
\begin{equation}
\kappa^2 a^2 + k^2 a^2 = \frac{2 m V_0 a^2}{\hbar^2}
\label{5.18}
\end{equation}
gleichzeitig, wegen \eqref{5.14}:
\begin{equation}
\kappa a = k a \tan(ka) \quad \tx{bzw.} \quad \kappa a = - ka \cot(ka)
\label{5.19}
\end{equation}


%T1
\hft

\noindent
Aus obiger Skizze ergibt sich, dass für $ V_0 $ klein genug derart, dass
\begin{equation}
\frac{2 m V_0 a^2}{\hbar^2} < \frac{\pi^2}{4} \quad \tx{d.h.} \quad V_0 a^2 < \frac{\pi^2 \hbar^2}{8m}
\label{5.20}
\end{equation}
nur ein einzelner gebundener Zustand existiert.\par
Weitere Beobachtungen anhand von (V.16): Das gebundene Teilchen mit $ E < V_0 $ besitzt eine endliche Wahrscheinlichkeitsdichte (freilich exponentiell abfallend) im klassisch verbotenen Gebiet $ |x| > a $ (dort: $ E < V_0 $, $ \frac{p^2}{2m} + V = E \Rightarrow p_{cl} = \sqrt{2 m (E-V_0)} $) mit einer charakteristischen Eindringtiefe (penetration depth) $ \sim \kappa_{1,2}^{-1} $ [s.a. Optik, Feldverlauf an Grenzflächen von Materialien mit unterschiedlichen Brechungsindizes, insbesondere jenseits des Grenzwinkels für Totalreflektion $ \to $ evaneszente Welle Spiegel für Atome].\par
Etwas einfacher: Grenzfall
\begin{equation}
V_0 \to \infty
\label{5.21}
\end{equation}
in \eqref{5.1}, d.h. wegen \eqref{5.5}, $ \kappa \to \infty $, somit, wegen \eqref{5.7}, vereinfacht sich der Ansatz für die Lösungen zu:
\begin{equation}
\psi_{I}(x) \overset{\kappa \to \infty}{\longrightarrow} 0 \quad \psi_{II}(x) = A e^{ikx} + B e^{-ikx} \quad \psi_{III}(x) \overset{\kappa \to \infty}{\longrightarrow} 0
\label{5.22}
\end{equation}
D.h. $ \psi $ verschwindet \textbf{am Rand} $ x = -a, +a $ des Potentials und in den Bereichen $ I $ und $ III $, als Gleichung
\begin{equation}
\psi_{II}(-a) = \psi_{II}(+a) = 0
\label{5.23}
\end{equation}
Daraus mit \eqref{5.22}:
\begin{equation}
A e^{-ika} + B e^{ika} = 0 \quad A e^{ika} + B e^{-ika} = 0
\label{5.24}
\end{equation}
und mit $ |\tx{Koeff. Matrix}| \overset{!}{=} 0 $ erhalten wir:
\begin{equation*}
e^{-2ika} - e^{2ika} = - 2 i \sin (2ka) = - 4 i \sin(ka) \cos(ka) = 0
\end{equation*}
\begin{equation}
\Rightarrow \quad \sin(ka) = 0 \ \tx{ oder } \ \cos(ka) = 0
\label{5.25}
\end{equation}
Daher
\begin{equation}
\begin{aligned}
ka &= n \frac{\pi}{2} \quad n = 1,3,5, \dots \quad (\cos(ka) = 0) \\
ka &= n \frac{\pi}{2} \quad n = 2,4,6, \dots \quad (\sin(ka) = 0)
\end{aligned}
\label{5.26}
\end{equation}
Eigenwerte \eqref{5.26} in Gleichungssystem \eqref{5.24} ergibt nach Addition/Subtraktion derselben:
\begin{equation*}
A 2 \cos(ka) + B 2 \cos(ka) = 0 \qquad - 2 i A \sin(ka) + 2 i B \sin(ka) = 0
\end{equation*}
ergo, für $ \cos(ka) = 0: A = B $ und für $ \sin(ka) = 0: A = - B $, und daher insgesamt
\begin{equation}
\psi_{II}(x) = \left\{ \begin{array}{ll}
\displaystyle \frac{1}{\sqrt{a}} \cos \left( \frac{n \pi x}{2 a} \right) & n = 1,3,5,\dots \\
\displaystyle \frac{1}{\sqrt{a}} \sin \left( \frac{n \pi x}{2 a} \right) & n = 2,4,6,\dots
\end{array} \right\}
\label{5.27}
\end{equation}
wobei der Vorfaktor wiederum aus der Normierungsbedingung folgt.

%02.07.19

\noindent
$ \Rightarrow $ \eqref{5.5} $ \kappa^2 = \frac{2 m (V_0 - E)}{\hbar^2} \qquad k^2 = \frac{2 m E}{\hbar^2} $ mit \eqref{5.26}
\begin{equation}
k^2 a^2 = \frac{n^2 \pi^2}{4} \quad \Rightarrow \quad E_n = \frac{n^2 \pi^2 \hbar^2}{8 m a^2} \quad n \in \mathbb{N}
\label{5.28}
\end{equation}
\textbf{Wichtige Beobachtung/Eigenschaft:} $ E_n / E_1 = n^2 $, d.h. Anregungsenergien sind ganzzahlige Vielfache von $ E_1 $ (später ähnliche Situationen beim harmonischen Oszillator, aber beispielsweise \textbf{nicht} beim Wasserstoffatom!).
\begin{equation*}
\psi \propto e^{-i \ob{E_n t}^{\varphi_n} / \hbar} \qquad |\phi(t)\rangle = \sum c_n e^{-i E_n t / \hbar} |E_n\rangle
\end{equation*}
Angenommen $ E_1 t / \hbar = 2 \pi \Rightarrow E_n t / \hbar = n^2 E_1 t/\hbar = n^2 2 \pi $ d.h. Rephasierungsphänomen führt zu ``revivals''.
\begin{equation*}
\dot{\Theta} = \prt{H}{I} \qquad \dot{I} = -\prt{H}{\Theta}
\end{equation*}

\section{Paritätsoperator}

Die Lösungen (V.16)  und \eqref{5.27} zerfallen in gerade und ungerade Lösungen unter Inversion $ x \to - x $ der Ortskoordinate, d.h. entweder $ \psi(x) = \psi(-x) $ oder $ \psi(x) = - \psi(-x) $. Diese \textbf{spezielle} Eigenschaft verdankt sich der \textbf{Invarianz} des Hamiltonoperators unter der Transformation $ x \mapsto -x $ (s. \eqref{5.1}). Allgemeiner spricht man von \textbf{Symmetrieeigenschaften} ($ \to $ Gruppentheorie, QM II).\par
Wir definieren den \textbf{Paritätsoperator} $ \Pi $, der diese Transformation vermittelt:
\begin{equation}
\begin{aligned}
\Pi | \psi\rangle &= \int |q\rangle \langle - q | \psi \rangle \dd q \qquad \forall | \psi \rangle \in \ham \\
\Pi_{\{q\}} \langle q | \psi \rangle &= \langle q | \Pi | \psi \rangle = \psi(-q)
\end{aligned}
\label{5.29}
\end{equation}
Invarianz von $ H $ unter $ \Pi $ ist äquivalent zu
\begin{equation}
[\Pi, H] = 0
\label{5.30}
\end{equation}
womit insbesondere folgt: $ H|\psi\rangle = i \partial_t |\psi\rangle $, dann auch
\begin{equation}
\begin{aligned}
H \Pi |\psi\rangle &= i \partial_t |\psi\rangle \\
H \Pi |\psi\rangle &\overset{\eqref{5.30}}{=} \Pi H | \psi\rangle = \Pi i \partial_t |\psi\rangle = i \partial_t \Pi |\psi\rangle
\end{aligned}
\label{5.31}
\end{equation}
Wegen $ \Pi^2 = \tx{id} $, folgt
\begin{equation}
\Pi^{-1} = \Pi
\label{5.32}
\end{equation}
, sowie
\begin{equation}
\Pi^2 |\pi\rangle = \pi^2 | \pi \rangle = | \pi \rangle
\label{5.33}
\end{equation}
mit $ |\pi\rangle $ EV von $ \Pi $ zum EW $ \pi $
\begin{equation}
\Rightarrow \quad \pi = \pm 1
\label{5.34}
\end{equation}
Entsprechend heißen Zustände zum Eigenwert $ \pi = + 1 $ \textbf{Zustände gerader Parität} und zu $ \pi = - 1 $ \textbf{Zustände ungerader Parität}.\par
Das zweiwertige Spektrum von $ \Pi $ erlaubt die Zerlegung des Hilbertraums in zwei Unterräume gerader bzw. ungerader Parität. Sei $ P_+ $ der Projektor auf den geraden, $ P_- $ jener auf den ungeraden Unterraum, so gilt $ \forall |\psi \rangle \in \ham $
\begin{equation}
|\psi \rangle = P_+ |\psi \rangle + P_- |\psi \rangle
\label{5.35}
\end{equation}
Anwendung von $ \Pi $ auf \eqref{5.35} liefert
\begin{equation}
\Pi |\psi \rangle = P_+ | \psi \rangle - P_- | \psi \rangle
\label{5.36}
\end{equation}
Entsprechend Definition gerader und ungerader Operatoren $ A $ und $ B $:
\begin{equation}
\Pi A \Pi^{-1} = A \qquad A \tx{ \textbf{gerade}} \qquad \Pi B \Pi^{-1} = - B \qquad B \tx{ \textbf{ungerade}}
\label{5.37}
\end{equation}
Speziell folgern wir für Matrixelemente ungerader Operatoren bzgl. Zuständen $ |\pi, \dots \rangle $ wohldefinierte Parität:
\begin{equation}
\begin{aligned}
\langle \pi \dots | B | \pi \dots \rangle &\overset{5.34}{=} \pi^2 \langle \pi \dots | B | \pi \dots \rangle \overset{\eqref{5.32}}{=} \langle \pi \dots | \Pi B \Pi^{-1} | \pi \dots \rangle \\
&\overset{\eqref{5.37}}{=} - \langle \pi \dots | B | \pi \dots \rangle \quad \Rightarrow \quad \langle \pi \dots | B | \pi \dots \rangle = 0
\end{aligned}
\label{5.38}
\end{equation}
Dies hatten wir in \eqref{3.31} bereits genutzt - unter der Annahme, dass atomare Eigenzustände wohldefinierte Parität haben (was für $ H $-Eigenzustände \textbf{in der sphärischen Basis} auch zutrifft (s. späteres Kapitel)!).

\section{Ungebundene Zustände des Kastenpotentials}

Im Gegensatz zu \eqref{5.6} betrachten wir um Energien mit
\begin{equation}
E > V_0
\label{5.39}
\end{equation}
im Kastenpotential, wählt aber auch den Energienullpunkt an anderer Stelle
\begin{equation}
V(x) = \casess{0}{x < -a \ \wedge \ x > a}{- V_0}{-a \le x \le a}
\label{5.40}
\end{equation}

%T1
\hft

\noindent
Mit der durch die Skizze vermittelten Intuition (einlaufende Amplitude von $ - \infty $) setzen wir als Lösung der Schrödingergleichung \eqref{5.3} an :
\begin{equation}
\begin{aligned}
\psi_I(x) &= A_1 e^{ikx} + B_1 e^{-ikx} \\
\psi_{II}(x) &= A_2 e^{iKx} + B_2 e^{-iKx} \\
\psi_{III}(x) &= A_3 e^{ikx}
\end{aligned}
\label{5.41}
\end{equation}
mit
\begin{equation}
k = \sqrt{\frac{2mE}{\hbar^2}} \qquad K = \sqrt{\frac{2m (E+V_0)}{\hbar^2}}
\label{5.42}
\end{equation}
\lcom{das negative $ V_0 $ sorgt für ein positives Vorzeichen und damit für ein reelles $ K $.}\\[5pt]
Stetigkeitsforderung an $ \psi(x) $ und $ \psi'(x) $ in $ x = \pm a $ führt auf folgende Gleichungen:
\begin{equation}
\begin{aligned}
A_1 e^{-ika} + B_1 e^{ika} &= A_2 e^{-iKa} + B_2 e^{iKa} \\
k \left[A_1 e^{-ika} - B_1 e^{ika}\right] &= K \left[A_2 e^{-iKa} - B_2 e^{iKa}\right] \\
A_3 e^{ika} &= A_2 e^{iKa} + B_2 e^{-iKa} \\
k A_2 e^{ika} &= K \left[A_2 e^{iKa} - B_2 e^{-iKa}\right]
\end{aligned}
\label{5.43}
\end{equation}
Da dies vier Gleichungen für fünf Koeffizienten sind, ergeben sich Lösungen, parametrisiert durch $ A_1 $, für beliebige $ k, K $, d.h. Kontinuierliches Energiespektrum. Mit:
\begin{equation}
\begin{aligned}
C &\defeq \left(1 - \frac{K}{k}\right)^2 e^{2ika} - \left(1 + \frac{K}{k}\right)^2 e^{-2iKa} \\
&= - \frac{4 K}{k} \cos(2Ka) + 2i \left[1 + \left(\frac{K}{k}\right)^2\right] \sin(2 K a)
\end{aligned}
\label{5.44}
\end{equation}
findet man
\begin{equation}
\begin{aligned}
B_1 &= A_1 \frac{2i}{c} \left[1 - \left(\frac{K}{k}\right)^2\right] \sin(2 K a) e^{-2ika} \\
A_2 &= - A_1 \frac{2}{C} \left(1 + \frac{K}{k}\right) e^{-i(K+k)a} \\
B_2 &= A_1 \frac{2}{C} \left(1 - \frac{K}{k}\right) e^{i(K-k)a} \\
A_3 &= - A_1 \frac{4K}{Ck} e^{-2ika}
\end{aligned}
\label{5.45}
\end{equation}
Aus \eqref{5.45} gewinnt man aus der Verhältnissen der einlaufenden und auslaufenden Wahrscheinlichkeitsdichten den Transmissions und Reflexionskoeffizienten $ T $ und $ R $:
\begin{equation}
T \defeq \frac{|A_3|^2}{|A_1|^2} = \frac{j_3}{j_{1,\tx{in}}} \qquad \qquad R \defeq \frac{|B_1|^2}{|A_1|^2} = \frac{j_{1, \tx{out}}}{j_{1,\tx{in}}}
\label{5.46}
\end{equation}
[vgl. aber Potentialschwelle]\\[5pt]
\eqref{5.45} in \eqref{5.46}:
\begin{equation}
T = \left[1 + \frac{1}{4} \left(\frac{K}{k} - \frac{k}{K}\right)^2 \sin^2(2Ka)\right]^{-1} \qquad R = \frac{1}{4} \left(\frac{K}{k} - \frac{k}{K}\right)^2 \sin^2(2Ka) \cdot T
\label{5.47}
\end{equation}
mit $ R+T=1 $.\\[5pt]
Obwohl $ E>V_0 $ finden wir i.a. $ R \ge 0 $, was im klassischen Fall nicht passieren würde $ \to $ Quantenreflexion.\par
Ergebnis bleibt unverändert, wenn statt Potantialmulde Potentialbarriere sofern $ E > V_0 $ (d.h. \eqref{5.42} reell bleibt).

%08.08.19

\section{Tunneleffekt}

Zum Schluss betrachten wir die Situation $ V = + V_0 $ (Potentialbarriere) mit $ E < V_0 $. Dann wird
\begin{equation}
K = \sqrt{\frac{2 m (E-V_0)}{\hbar^2}}
\label{5.48}
\end{equation}
imaginär und statt \eqref{5.47} findet man (jedoch völlig analoge Rechnung!).
\begin{equation}
T = \left[1 + \frac{1}{4} \left(\frac{\kappa}{k} - \frac{k}{\kappa}\right)^2 \sinh^2 \left(2 \kappa a\right)\right]^{-1}
\label{5.49}
\end{equation}
mit $ K \defeq i \kappa $ und $ \kappa^2 = \frac{2 m (V_0 - E)}{\hbar^2} $.\\[5pt]
$ T $ nimmt i.a. nichtverschwindende Werte an, was eine (klassisch verbotene) Durchlässigkeit der Potentialbarriere für Injektionsenergien $ E < V_0 $ impliziert. Dies ist der \textbf{Tunneleffekt}.

%T1
\hft Transmission in Abhängigkeit der Energie

%T2
\hft Verschiedene Bilder

\noindent
Bemerkungen:
\begin{enumerate}[(1)]
	\item Mathematisch betrachtet sind Tunneln durch Potentialbarrieren im Raum und Resonante Anregung eines Zweiniveau-Atoms und das (dynamische) Tunneln durch Phasenraumbarrieren (z.B. $ T^2 \equiv $ Zwei-Tori in einem 3D-Phasenraum) strikt äquivalent. Spektral werden alle diese Situationen durch antikreuzende Energieniveaus charakterisiert (Siehe Grafik $ E(\Delta) $ bei der ,,Vermiedenen Kreuzung``).
	\item \eqref{5.47} entnimmt man, dass $ T = 1 $ für
	\begin{equation}
	2 K a = n \cdot \pi \quad \tx{ für } \quad n \in \mathbb{N}
	\label{5.50}
	\end{equation}
	Für die Potentialmulde impliziert dies mit \eqref{5.42} \textbf{resonante} Energieniveaus
	\begin{equation}
	E_n = \ub{\frac{\pi^2 \hbar^2 n^2}{8 m a^2}}_{\mathclap{\substack{\tx{s.o. gebundene Energien} \\ \tx{im unendlich hohen} \\ \tx{Potentialtopf } \eqref{5.28}}}} - V_0
	\label{5.51}
	\end{equation}
	, die \eqref{5.50} genügen, wobei $ E_n > 0 $, ergo $ n^2 \ge \frac{8ma^2 V_0}{\pi^2 \hbar^2} $.\par
	Das Attribut ,,resonant```bezieht sich auf die Stehwellenbedingung \eqref{5.50}, wobei \eqref{5.28} die $ E_n $ aus \eqref{5.51} jedoch in das Kontinuum ungebundener Energien $ E > 0 $ eingebettet sind.
	\item In der \textbf{Störungstheorie} sucht man allgemein (d.h. in $ n $ Dimensionen) nach einer Abbildung, die einlaufende auf auslaufende Amplituden abbildet: $ c_L^{\tx{in}} $, $ c_R^{\tx{in}} $, $ c_L^{\tx{out}} $ und $ c_R^{\tx{out}} $. Diese Abbildung wird durch die Streumatrix (Streuoperator) $ S $ vermittelt:
	\begin{equation*}
	\begin{pmatrix}
	c_{L}^{\tx{out}} \\ c_{R}^{\tx{out}}
	\end{pmatrix} = S \begin{pmatrix}
	c_{L}^{\tx{in}} \\ c_{R}^{\tx{in}}
	\end{pmatrix}
	\end{equation*}
	[siehe QM II].
\end{enumerate}

\section{Der lineare harmonische Oszillator}

$ \to $ lokale Näherung an eine Vielzahl physikalischer Situationen\\
$ \to $ grundlegende Rolle in (Quanten-) Optik, Begriff des Photons, Gitterschwingungen in Festkörpern, Quanteninformation ($ \to $ Daten-Bus von Quantencomputern), Dekohärenztheorie, Quantenfeldtheorie\\
$ \to $ hier nur 1D harmonischer Oszillator
\begin{equation}
H = \frac{p^2}{2m} + \frac{m \omega^2}{2} x^2 = \frac{p^2}{2m} + \frac{1}{2} k x^2
\label{5.52}
\end{equation}
Zur Quantisierung reskalieren wir:
\begin{equation}
H \longrightarrow \tilde{H} \defeq \frac{1}{\hbar \omega} H = \frac{p^2}{2 m \hbar \omega} + \frac{m \omega}{2 \hbar} x^2
\label{5.53}
\end{equation}
\eqref{5.53} wird mit der Definition:
\begin{equation}
\tilde{p} \defeq \frac{p}{\sqrt{m \hbar \omega}} \qquad \tilde{x} \defeq \sqrt{\frac{m \omega}{\hbar}} x
\label{5.54}
\end{equation}
\begin{equation}
\tilde{H} = \frac{1}{2} (\tilde{p}^2 + \tilde{x}^2)
\label{5.55}
\end{equation}
Weiter erhält man für den reskalierten Komutator von $ \tilde{p} $ und $ \tilde{x} $
\begin{equation}
[\tilde{x}, \tilde{p}] = i \mathbbm{1}
\label{5.56}
\end{equation}
Zur algebraischen Lösung des Eigenwertproblems des HOs führen wir neue Operatoren ein:
\begin{equation}
a \defeq \frac{1}{\sqrt{2}} (\tilde{x} + i \tilde{p}) \qquad a^{\dagger} \defeq \frac{1}{\sqrt{2}} (\tilde{x} - i \tilde{p})
\label{5.57}
\end{equation}
, die nicht mehr selbsadjungiert (noch normal) sind:
\begin{equation}
\begin{aligned}
\left[a,a^\dagger\right] = \frac{1}{2} [\tilde{x}+ i \tilde{p}, \tilde{x} - i \tilde{p}] = \frac{1}{2} \left\{ [\tilde{x}, \tilde{x}] + [\tilde{p}, \tilde{p}] - i[\tilde{x}, \tilde{p}] + i [\tilde{p}, \tilde{x}] \right\} = \frac{1}{2} \cdot 2 \cdot i[\tilde{p}, \tilde{x}] = \mathbbm{1}
\end{aligned}
\label{5.58}
\end{equation}