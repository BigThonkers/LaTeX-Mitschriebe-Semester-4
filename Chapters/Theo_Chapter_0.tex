% Vorlesung 23.04.19

\setcounter{chapter}{-1}

\chapter{Einleitung}

\section{Wichtige Infos}

\begin{description}
	\item[Professor] Andreas Buchleitner  Zi. 901\\
	\verb|abu@uni-freiburg.de| \\
	\verb|buchleitner_office@physik.uni-freiburg.de|
	\item[Sekretäre] Gislinde Bühler \& Susanne Trantke Zi. 804
	\item[Übungsleiter] Eduardo Carnio, Zi. 910\\
	 \verb|eduardo.carnio@physik.uni-freiburg.de|
	\item[ILLIAS] Theorie III:  Password: \texttt{TPIIIss19}
	\item[Klausur] 15. Juli 13:00 - 16:00 Uhr im großen Hörsaal
\end{description}

\section{Programm}

\begin{itemize}
	\item Proseminar (BSc) zus. mit MSc-Seminar\\
	QM für Liebhaber \& Interpretation of QM
	\item Kolloquium montags 17:15 Uhr 27. Mai Göttinger Erklärung, CF v. W.
	\item 23.-27. September DPG Fall Meeting, Quantum Sciences and IT
\end{itemize}

\section{Litaratur}

(auch auf ILLIAS gelistet)

\begin{itemize}
	\item C. Cohen-Tannodji, B. Diu, F. Lafo\"e, M\'ecauique quantique, F,D,E, Vol I + II
	\item O. Hittmeier Lehrbuch d. Quantenmechanik, Thienig 1972
	\item B. - G. Engert, Lectures on quantum mechanics, I - IV, World Scientific 2006
	\item M. Bartelmann et al, Theoretische Physik, Springer 2015
	\item J.J. Sakurai, Modern Quantumechanics, Addison-Wesley 1995
	\item A. Peres, Quantum Theory: Concepts and Methods, Kluwer 1995
	\item M.A. Nielson, I. L. Chang, Quantum Computation \& Quantum Information, Cambridge University Press 2000
	\item Landau \& Lifschitz, Lehrbuch der Theoretischen Physik Bd. III
\end{itemize}

\noindent
\textbf{Formelsammlung:} Bronstein \& Sememdiciev. Taschenbuch d. Mathematik