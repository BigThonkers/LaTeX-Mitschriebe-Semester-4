% packages
\PassOptionsToPackage{dvipsnames}{xcolor}  % needed to get colors working in certain environments
\documentclass[titlepage,11pt,a4paper,ngerman]{report}
\usepackage[utf8]{inputenc}
\usepackage[T1]{fontenc}
\usepackage[german]{babel}
\usepackage{graphicx}
\usepackage{wrapfig}  % used for wrapping a figure alternative to using minipages
\usepackage{amsmath}
\usepackage{amsfonts}
\usepackage{amssymb}
\usepackage[hidelinks]{hyperref}  % removes coloring for links
\usepackage{cleveref}
\usepackage{tikz}
\usepackage{tikz-cd}
\usepackage{nicefrac}  % adds \nicefrac alternative to regular \frac
\usepackage{mathtools}
\usepackage{enumerate}
\usepackage{cancel}  % adds \cancel which adds strikethrough
\usepackage{tocloft}
\usepackage{tcolorbox}
\usepackage{bm}  % adds \bm to make symbols bold, replaces outdated \boldsymbol
\usepackage[shortlabels]{enumitem}
\usepackage{placeins}
\usepackage{booktabs}
\usepackage{wasysym}

\usepackage[margin=1in]{geometry}  % changes the margins on all pages
\usepackage{url}

%SI-unix
%\usepackage{array}
\usepackage[per=slash,
            decimalsymbol=comma,
			loctolang={DE:ngerman,UK:english},
			]{siunitx}	
\sisetup{locale = DE}

\usetikzlibrary{calc}
\usetikzlibrary{decorations.pathmorphing,patterns}
\usetikzlibrary{arrows}
\usetikzlibrary{decorations.pathreplacing}
%\usetikzlibrary{snakes}

% Andrez:
%\usepackage{epigraph}  % adds \epigraph used to add fancy quotes to the beginning of chapters
%\usepackage{fancyhdr}
%\setlength{\parskip}{1em}
%\setlength{\headheight}{35pt}
%\setlength\epigraphwidth{.8\textwidth}
% alt math font
%\usepackage{eulervm}  % switches to alternate math font, usually requires extra download


%Environments und Newcommands:


% general commands

% zu zeigen symbol
\newcommand{\zz}{\fontfamily{cmss} \selectfont{Z\kern-.61em\raise-0.7ex\hbox{Z}:}}
% build over
\newcommand{\bov}[2]{\buildrel{#2} \over{#1}}
% better looking := (defined as)
\newcommand*{\defeq}{\mathrel{\vcenter{\baselineskip0.5ex \lineskiplimit0pt \hbox{\scriptsize.}\hbox{\scriptsize.}}}=}
\newcommand*{\eqdef}{=\mathrel{\vcenter{\baselineskip0.5ex \lineskiplimit0pt \hbox{\scriptsize.}\hbox{\scriptsize.}}}}

% integral differential d
\newcommand{\dif}{\mathop{}\!\mathrm{d}}
\newcommand{\difi}[1]{\mathrm{d}#1\mathop{}\!}

\newcommand{\prt}[2]{\frac{\partial #1}{\partial #2}}  % used for partial derivatives, tip: input #1 can be left blank
\newcommand{\prd}[2]{\frac{\tx{d} #1}{\tx{d} #2}}  % used for absolute (standart) derivatives, tip: input #1 can be left blank

\newcommand{\dd}{\tx{d}}


%Mathe:
\newcommand{\verteq}{\rotatebox{90}{$\,=$}}  % used for commands below
\newcommand{\equalto}[2]{\underset{\scriptstyle\overset{\mkern4mu\verteq}{#2}}{#1}}  % adds equal to underneath
\newcommand{\equaltoup}[2]{\overset{\scriptstyle\underset{\mkern4mu\verteq}{#2}}{#1}}  % adds equal to above
\newcommand{\custo}[3]{\underset{\scriptstyle\overset{\mkern4mu\rotatebox{-90}{$\,#1$}}{#3}}{#2}}  % same as above but replaces equal sign with input #1
\newcommand{\custoup}[3]{\overset{\scriptstyle\underset{\mkern4mu\rotatebox{-90}{$\hspace{-3pt} #1$}}{#3}}{#2}}
\newcommand{\casess}[4]{\left\{ \begin{array}{ll} {#1} & {#2} \\ {#3} & {#4} \end{array} \right.}  % used to indicate to the reader that something is missing here


%Text:
\newcommand{\tx}[1]{\textrm{#1}}
\newcommand{\const}{\tx{const.}}

\newcommand{\ul}[1]{\underline{#1}}
\newcommand{\ol}[1]{\overline{#1}}
\newcommand{\ub}[1]{\underbrace{#1}}
\newcommand{\ob}[1]{\overbrace{#1}}

\newcommand{\hfw}{\color{RubineRed}\tx{ $\star$hier fehlt was$\star$ } \color{black}}  % used to indicate to the reader that something is missing here 


%Spezielles:


%Theo:
\newcommand{\lag}{\mathcal{L}}  % used for Lagrange function
\newcommand{\ham}{\mathcal{H}}  % used for Hamiltonian function
\newcommand{\gre}{\mathcal{G}}  % used for Green's function
\newcommand{\eofr}{\vec{E}(\vec{r})}
\newcommand{\pofr}{\Phi(\vec{r})}
\newcommand{\grr}{\mathcal G(\vec{r},\vec{r}')}
\newcommand{\vphi}{\varphi}
\newcommand{\vabla}{\vec{\nabla}}


%LA:
\newenvironment{bew}[1]{\subsection{Bew: #1}}{\hfill$\square$}
\newcommand{\Bew}[2]{\begin{bew}{#1}#2\end{bew}}
\newcommand{\enph}{F: V \to V \textrm{ Endomorphismus}}

\newcommand{\im}{\tx{im}}
\newcommand{\spa}{\tx{span}}
\newcommand{\adj}{\tx{adj}}
\newcommand{\grad}{\tx{grad}}
\newcommand{\ord}{\tx{ord}}

\newcommand{\basis}[3]{\{#1_{#2}, \dots, #1_{#3}\}}
\newcommand{\ska}[2]{\langle #1 , #2 \rangle}  % scalar product of input 1, and 2 can also be used for braket notation
\newcommand{\dmat}[3]{\begin{pmatrix} #1_{#2}&&\\ &\ddots& \\ && #1_{#3} \end{pmatrix}}


%Ex:
\newcommand{\kq}{\frac{1}{4\pi\epsilon_0}}  % writes out the whole constant k from electrostatic
\newcommand{\kqq}{\frac{\mu_0}{4\pi}}  % writes out the constant from magnetostatic
\newcommand{\uind}{U_{\tx{ind}}}
\newcommand{\folie}[1]{\color{gray}[Folie: #1]\color{black}}  % used to tell the reader that there was multimedia content during a lecture
\newcommand{\versuch}[1]{\color{red!50!black} \textbf{Versuch:} \color{black} \textbf{#1}\\ }  % used to tell the reader that there was a live experiment

\newcommand{\mau}{$\buildrel \mathcal{O} \over{\textbf{.}}$}  % the extreme Waldmann exclamation mark recreated in latex by Markus


% Lab commands:
\newcommand\mean{\begin{equation}
\frac{\sum_{i=1}^n x_i}{n}\label{mean}
\end{equation}}  % shortcut for the standard Mean function

\newcommand\meanstd{\begin{equation}
s_x=\sqrt{\frac{1}{n-1}\sum_{i=1}^n(x_i-\overline{x})^2}\label{meanstd}
\end{equation}}  % shortcut for the standard derivative mean function

\newcommand\prodquo{\begin{equation}\left\vert\frac{\Delta z}{z}\right\vert=\sqrt{\left(a\frac{\Delta x}{x}\right)^2+\left(b\frac{\Delta y}{y}\right)^2+\ldots}\textrm{ f\"ur }z=x^a\ y^b\ldots\end{equation}}

\newcommand\tfuncd{\begin{equation}
t=\frac{\vert x_n-y_n\vert}{\sqrt{x_s^2+y_s^2}}
\end{equation}}

\newcommand\tfunc{\begin{equation}
t=\frac{\vert x-y_0\vert}{u_x}
\end{equation}}


% ANDREZ
%\newcommand{\summ}[2]{\sum_{#1}^{#2}}
%\newcommand{\intt}[2]{\int_{#1}^{#2}}
\newcommand{\lcom}[1]{\color{MidnightBlue}#1\color{black}}  % used to indicate to the reader that this content was transcribed from the lecture
\newcommand{\bei}{\emph{Beispiel:}}
\newcommand{\bem}{\emph{Bemerkung:}}


% Boxen:

\tcbuselibrary{theorems}

% mahlt eine box nur um den text mit titel
\newtcbox{\fribox}[1]{nobeforeafter,colback=white,colframe=red!75!black,fonttitle=\bfseries,title=#1,sharp corners,tcbox raise base}

% mahlt eine große box um alles mit titel
\newcommand{\frbox}[2]{\begin{tcolorbox}[colback=white,colframe=red!75!black,fonttitle=\bfseries,title=#1]#2\end{tcolorbox}}

% mahlt eine box nur um den text
\newtcbox{\ribox}{nobeforeafter,colback=white,colframe=red!75!black,sharp corners,tcbox raise base}

% mahlt eine große box um alles was drinnen ist
\newcommand{\rbox}[1]{\begin{tcolorbox}[colback=white,colframe=red!75!black]#1\end{tcolorbox}}

% mahlt eine box um mathe innerhalb mathmode
\newcommand{\rmbox}[1]{\tcboxmath[colback=white,colframe=red!75!black]{#1}}

% super box (looks like regular boxed but wraps around anything)
\newenvironment{supbox}{\begin{tcolorbox}[colback=white,colframe=black,sharp corners,boxrule=.5pt]}{\end{tcolorbox}}

% the Big Black Box, can be used to separate examples or review material from the main text (used for Wiederholung)
\newcommand{\bbb}[2]{\begin{tcolorbox}[colback=white,colframe=black,fonttitle=\bfseries,title=#1,sharp corners,tcbox raise base]#2\end{tcolorbox}}

% array type box with title
\newenvironment{zebox}[1]{\begin{array}{|c|}
		\multicolumn{1}{l}{\tx{#1}} \\
		\hline
		\displaystyle
	}{\\ \hline
\end{array}}

% arrow list
\newlist{arrowlist}{itemize}{1}
\setlist[arrowlist]{label=$\Rightarrow$}


% optional:

\renewcommand{\vec}[1]{\bm{#1}}

% changed because of preferred looks
\renewcommand{\epsilon}{\varepsilon}
\renewcommand{\paragraph}[1]{\subsubsection{#1}}  % changed oddly behaving paragraphs to simple subsubsections which are not numbered nor in the toc

% nur in Theo benutzt !!!
% \renewcommand{\Phi}{\varPhi}

% evtl:
% \renewcommand{\boxed}{\rmbox}


% Tikz definitions:

\def\centerarc[#1](#2)(#3:#4:#5)% Syntax: [draw options] (center) (initial angle:final angle:radius)
{ \draw[#1] ($(#2)+({#5*cos(#3)},{#5*sin(#3)})$) arc (#3:#4:#5); }

\def\checkmark{\tikz\fill[scale=0.4](0,.35) -- (.25,0) -- (1,.7) -- (.25,.15) -- cycle;}  % checkmark used for proofs

\tikzset{
	annotated cuboid/.pic={
		\tikzset{%
			every edge quotes/.append style={midway, auto},
			/cuboid/.cd,
			#1
		}
		\draw [every edge/.append style={pic actions, densely dashed, opacity=.5}, pic actions]
		(0,0,0) coordinate (o) -- ++(-\cubescale*\cubex,0,0) coordinate (a) -- ++(0,-\cubescale*\cubey,0) coordinate (b) edge coordinate [pos=1] (g) ++(0,0,-\cubescale*\cubez)  -- ++(\cubescale*\cubex,0,0) coordinate (c) -- cycle
		(o) -- ++(0,0,-\cubescale*\cubez) coordinate (d) -- ++(0,-\cubescale*\cubey,0) coordinate (e) edge (g) -- (c) -- cycle
		(o) -- (a) -- ++(0,0,-\cubescale*\cubez) coordinate (f) edge (g) -- (d) -- cycle;
		\path [every edge/.append style={pic actions, |-|}]
		%(b) +(0,-5pt) coordinate (b1) edge ["\cubex \cubeunits"'] (b1 -| c)
		%(b) +(-5pt,0) coordinate (b2) edge ["\cubey \cubeunits"] (b2 |- a)
		%(c) +(3.5pt,-3.5pt) coordinate (c2) edge ["\cubez \cubeunits"'] ([xshift=3.5pt,yshift=-3.5pt]e)
		;
	},
	/cuboid/.search also={/tikz},
	/cuboid/.cd,
	width/.store in=\cubex,
	height/.store in=\cubey,
	depth/.store in=\cubez,
	units/.store in=\cubeunits,
	scale/.store in=\cubescale,
	width=10,
	height=10,
	depth=10,
	units=cm,
	scale=.1,
}


% other settings
\hbadness=99999  % removes unnecessary hbadness warnings

% the following are used to circumvent Roman numerals in the toc from running out of space
%\addtolength{\cftchapnumwidth}{10pt}
%\addtolength{\cftsecnumwidth}{10pt}
%\addtolength{\cftsubsecnumwidth}{10pt}
%\renewcommand{\thechapter}{\Roman{chapter}}  % just don't do this :)

% coloring
% for working at night
%\pagecolor{darkgray}
%\color{white}

\begin{document}

\title{
	{\Huge Experimentalphysik IV}\\[1em]
	{\huge Atom-, Molekül- und Festkörperphysik}\\[1em]
	{\Large Vorlesung von Prof. Dr. Giuseppe Sansone im Sommersemester 2019}}
\author{Markus Österle \hspace{5pt} Damian Lanzenstiel}
\date{ \today}
\maketitle
\tableofcontents

% Document

\setcounter{chapter}{-1}

\chapter{Einführung}

\section{Wichtige Infos}

\subsection{Programm}

\begin{itemize}
	\item Atomphysik
	\item Molekülphysik
	\item Festkörperphysik
\end{itemize}

\subsection{Übungen}

\begin{itemize}
	\item Anfang ab dem 06.05. - 10.05.
	\item Übungsblatt 0 zum Einstieg wird nicht bewertet
\end{itemize}

\subsection{Literatur}

\begin{itemize}
	\item Demtröder, Ex 3
	\item Haken-Wolf, Atom und Quantenphysik
	\item Christopher J. Foot, Atomic Physics
	\item Messiah Dover, Quantum Mechanics
	\item B.H. Bransden, Physics of atoms and molecules
\end{itemize}

\section{Leistungen}

Studienleistung: 50 \% der Punkte aller Übungsblätter (kein Kriterium für die Zulassung zur Prüfung)\\[5pt]
Prüfungsleistung: 50 \% der Punkte und schriftliche Prüfung.\\[10pt]
\textbf{Termine}:\\
Klausur 25.07.19 12-14 Uhr\\



\chapter{Quantenmechanik}

\section{Grundlagen}

\begin{enumerate}[i)]
	\item \textbf{Wellenfunktion} $ \Psi(\vec{r}) $ $ \qquad $ mit $ \vec{r} $ der \textbf{Ortsdarstellung}
	\begin{equation*}
	P = \int_{V} \left|\Psi ( \vec{r})\right|^2 \dd \vec{r} \quad \tx{Wahrscheinlichkeit} \qquad \quad \int_{V} \left|\Psi(\vec{r})\right|^2 \dd \vec{r} = 1 \quad \tx{Normierungsbedingung}
	\end{equation*}
	\item \textbf{Operatoren} $ \hat{O} $
	\begin{equation*}
	\hat{O} \Psi(\vec{r}) = \Psi'(\vec{r})
	\end{equation*}
	Korrespondenzprinzip
	\begin{align*}
	\vec{r} &= \hat{\vec{r}} \qquad \hat{\vec{r}} \Psi(\vec{r}) = \Psi'(\vec{r}) = \vec{r} \Psi(\vec{r}) \qquad \quad ; \quad \, \rmbox{\hat{x} \Psi(x) = x \Psi(x) = \Psi'(x)} \\
	\vec{p} &= \hat{\vec{p}} \qquad \hat{\vec{p}} \Psi(\vec{r}) = \Psi'(\vec{r}) = - i \hbar \vabla \Psi(\vec{r}) \quad ; \quad \rmbox{\hat{p}_{x} \Psi(x) = - i \hbar \prd{}{x} \Psi(x) = \Psi'(x) }
	\end{align*}
	$ \hat{\ham} = $ Hamiltonian oder Hamilton-Operator. $ \hat{\ham} = \hat{\ham}(\hat{\vec{r}} , \hat{\vec{p}}) $
	\begin{equation*}
	\hat{\ham} = \frac{\hat{p^2}}{2m} + V(\vec{r}) = - \frac{\hbar^2}{2m} \vabla^2 + V(\vec{r})
	\end{equation*}
	\item \textbf{Zeitabhängige Schrödinger Gleichung}
	\begin{equation*}
	\rmbox{i\hbar \prt{}{t} \Psi(\vec{r}, t) = \hat{\ham}  \Psi(\vec{r}, t)}
	\end{equation*}
	Die klassische Energie sieht so aus:
	\begin{equation*}
	E = \frac{p^2}{2m} + V(\vec{r})
	\end{equation*}
	In der QM dann folgendermaßen:
	\begin{equation*}
	\hat{\ham} = \frac{\hat{p}^2}{2m} + V(\vec{r})
	\end{equation*}
	Die \textbf{Zeitunabhängige Schrödinger Gleichung} sieht wie folgt aus:
	\begin{equation*}
	\rmbox{\hat{\ham} \Psi(\vec{r}) = E \Psi(\vec{r})}
	\end{equation*}
	Diese Gleichung ist eine Eingenwertgleichung. Der Hamilton Operator liefert also den Energie-Eingenwert $ E $ und die Eigenzustände $ \Psi(\vec{r}) $.
	
	\subsection*{Stationäre Zustände}
	
	Jeder messbaren Physikalische Größe ist ein Operator $ \hat{O} $ zugeordnet. Bei einer physikalischen Messung wir der \textbf{Erwartungswert} gemessen: $ \langle \hat{O} \rangle = \langle \Psi | \hat{O} | \Psi \rangle $.
	\begin{equation*}
	\langle \hat{O} \rangle = \langle \Psi(\vec{r}) | \hat{O} | \Psi(\vec{r}) \rangle = \int \Psi^*(\vec{r}) \ \ub{ \hat{O} \ \ \Psi(\vec{r}) }_{\Psi'(\vec{r})} \dd \vec{r}
	\end{equation*}
	\begin{equation*}
	\langle \hat{O}(t) \rangle = \langle \Psi(\vec{r},t) | \hat{O} | \Psi(\vec{r},t) \rangle = \int \Psi^*(\vec{r},t) \hat{O} \Psi(\vec{r},t) \dd \vec{r}
	\end{equation*}
	Diese Gleichung können wir wie folgt umformen:
	\begin{equation*}
	\Psi(\vec{r},t) = \Psi(\vec{r},t = 0) \ub{e^{- \nicefrac{i E t}{\hbar}}}_{\tx{Phasenfaktor}}
	\end{equation*}
	\begin{align*}
	\langle \hat{O} \rangle &= \int \Psi^*(\vec{r}, t = 0) \cancel{e^{\nicefrac{i E t}{\hbar}}} \hat{O} \Psi(\vec{r}, t = 0) \cancel{e^{- \nicefrac{i E t}{\hbar}}} \dd \vec{r} \\
	&= \int \Psi^*(\vec{r}, t = 0) \hat{O} \Psi(\vec{r} , t = 0) \dd \vec{r} \overset{*}{=} \langle \hat{O}(t = 0) \rangle
	\end{align*}
	$ *: $ wenn $ \hat{O} $ nicht Zeitabhängig ist.\\[5pt]
	\textbf{Stationäre Zustände}
	\begin{equation*}
	i \hbar \prt{}{t} \Psi(\vec{r}, t) = \hat{\ham} \Psi(\vec{r}, t) \qquad \tx{mit} \qquad \Psi(\vec{r},t) e^{- \nicefrac{i E t}{\hbar}} \Psi(\vec{r} ,t = 0)
	\end{equation*}
	\begin{equation*}
	\frac{\partial \Psi}{\Psi} = - i \frac{\hat{\ham}}{\hbar} \partial t
	\end{equation*}
	Lösung der DGL mittels Variablen-Trennung
	\begin{equation*}
	\ln\left[\frac{\Psi(\vec{r},t)}{\Psi(\vec{r}, t = 0)}\right] = - \frac{i \hat{\ham} t}{\hbar} \quad \Rightarrow \quad \Psi(\vec{r}, t) = e^{- \nicefrac{i \hat{\ham} t}{\hbar}} \Psi(\vec{r}, t = 0)
	\end{equation*}
	\begin{equation*}
	\hat{\ham} \Psi(\vec{r} , t = 0) = E \Psi(\vec{r}, t = 0)
	\end{equation*}
	Taylor Entwicklung:
	\begin{equation*}
	e^{x} = e^{- \nicefrac{i \hat{\ham} t}{\hbar}} = a \left(\hat{\ham}\right)^0 + b \left(\hat{\ham}\right)^1 + c \left(\hat{\ham}\right)^2 + \dots
	\end{equation*}
	\begin{align*}
	e^{- \nicefrac{i \hat{\ham} t}{\hbar}} \Psi(\vec{r}, t = 0)&= a \Psi(\vec{r}, t = 0) + b \hat{\ham} \Psi(\vec{r}, t = 0) + c \hat{\ham} \cdot \hat{\ham} \Psi(\vec{r}, t = 0) + \dots \\
	&= a \Psi(\vec{r}, t = 0) + b E \Psi(\vec{r}, t = 0) + c E^2 \Psi(\vec{r}, t = 0) + \dots \\
	&= (a + bE + c E^2 + \dots) \cdot \Psi(\vec{r}, t = 0) \\
	&= e^{- \nicefrac{i E t}{\hbar}} \Psi(\vec{r}, t = 0)
	\end{align*}
	\lcom{Wir können einen Operator in der $ e $-Funktion schreiben, da diese mit der Taylorentwicklung als Reihe entwickelt werden kann.}
	\item \textbf{Spin} (Elektronen)
	\begin{enumerate}[$ \Rightarrow $]
		\item Wasserstoffatom (Stern-Gerlach)
		\item Helium (Pauli Prinzip)
	\end{enumerate}
	\newpage
	\item \textbf{Quantensysteme}
	\begin{itemize}
		\item Freies Teilchen, Potentialstufe (Tunneln)
		\item Harmonischer Oszillator $ \Rightarrow $ Molekülphysik
		\item Coulomb Potential $ \Rightarrow $ Wasserstoffatom
	\end{itemize}
	\begin{figure}[ht]
		\centering
		\begin{tikzpicture}[scale=0.8]
		\draw[thick,blue] (-3.5,.025) -| (-1.5,2) -- (1.5,2) |- (3.5,.025);
		\draw[thick,black!30!green,->] (-3,1) -- node[below] {Teilchen} ++(1,0);
		\draw[->] (-3.75,0) -- (3.75,0) node[anchor=north west] {$ x $};
		\draw[->] (0,-0.5) -- (0,2.5) node[anchor=south east] {\color{blue}$ V(x) $};
		\draw (-1.5,.1) -- ++(0,-.2) node[below] {$ -a $};
%		\draw (1.5,.1) -- ++(0,-.2) node[below] {$ +a $};
%		\node[draw=black,circle, minimum size=.8cm,red] at (-2.5,-1) {I};
%		\node[draw=black,circle, minimum size=.8cm,red] at (0,-1) {II};
%		\node[draw=black,circle, minimum size=.8cm,red] at (2.5,-1) {III};
		\draw[<->,blue] (.75,.1) -- node[right] {$ V_0 $} ++(0,1.8); 
		\end{tikzpicture}
		\label{Potentialbarriere}
		\caption{Darstellung einer Potentialbarriere. Beispiel für den Tunneleffekt eines hindurchfliegenden Teilchens, das eigentlich weniger Energie hat als klassisch nötig wäre um die Barriere zu überwinden. Dieses Bild wurde mit dem \LaTeX Paket Tikz erstellt.}
	\end{figure}
	\FloatBarrier
	\item \textbf{Kommutatoren}
	\begin{align*}
	\hat{x}&: \vec{p} = \vec{p}_x \qquad \qquad \left[\hat{x}, \hat{p}\right] = \hat{x} \hat{p} - \hat{p} \hat{x} \\
	\hat{A}; \hat{B}&: \qquad \qquad \qquad \ \left[\hat{A}, \hat{B}\right] = \hat{A} \hat{B} - \hat{B} \hat{A}
	\end{align*}
	Wellenfunktion $ \Psi(x) \qquad \qquad \left[\hat{x}, \hat{p}\right] \Psi(x) = \Psi'(x) $
	\begin{equation*}
	\left[\hat{x}, \hat{p}\right] = \hat{x} \hat{p} - \hat{p} \hat{x} = x \left(- i \hbar \prd{}{x}\right) - \left(- i \hbar \prd{}{x}\right) x
	\end{equation*}
	\begin{align*}
	\left[\hat{x}, \hat{p}\right] \Psi(x) &= \left\{ x \left(- i \hbar \prd{}{x}\right) - \left(- i \hbar \prd{}{x}\right) x \right\} \Psi(x) \\
	&= x \left(- i \hbar \prd{}{x} \Psi(x) \right) - \left(- i \hbar \prd{}{x}\right) x \Psi(x) \\
	&= - i \hbar x \prd{\Psi}{x} + i \hbar \prd{}{x} ( x \Psi(x)) \\
	&= \cancel{ - i \hbar x \prd{\Psi}{x}} \cancel{+ i \hbar x \prd{\Psi}{x}} + i \hbar \Psi(x) \ub{\prd{x}{x}}_{=1} \\
	&= i \hbar \Psi(x) = \Psi'(x)
	\end{align*}
	\begin{equation*}
	\rmbox{\left[\hat{x}, \hat{p}\right] = i \hbar} \quad \Rightarrow \quad \tx{die zwei Operatore vertauschen nicht !!!}
	\end{equation*}
	
	\subsection*{Eigenschaft Kommutator}
	
	$ \hat{A} $; $ \hat{B} $
	\begin{equation*}
	\Delta A \cdot \Delta B \ge \frac{1}{2} \left| \langle \left[\hat{A}, \hat{B}\right] \rangle \right|
	\end{equation*}
	$ \left[\hat{A}, \hat{B}\right] $ Operator $ \Rightarrow $
	\begin{align*}
	\langle \left[\hat{A}, \hat{B}\right] \rangle &= \langle \hat{A} \hat{B} - \hat{B} \hat{A} \rangle \\
	&= \langle \Psi | \hat{A} \hat{B} - \hat{B} \hat{A} | \Psi \rangle \\
	&= \int \Psi^* \left(\hat{A} \hat{B} - \hat{B} \hat{A}\right) \Psi \dd \vec{r}
	\end{align*}
	$ \Delta A $, $ \Delta B $ Standardabweichung
	\begin{equation*}
	\sigma_x = P(x) \quad \sigma_x = \left[ \int (x - \mu)^2 P(x) \dd x \right]^{1/2}  \qquad \mu = \int x P(x) \dd x
	\end{equation*}
	
	% T Gaußglocke
	
	\begin{figure}
		\centering
		\begin{tikzpicture}
			\draw[->] (0,-0.2) -- (0,2.5) node[anchor=south east] {$ P(x) $};
			\draw[->] (-0.2,0) -- (7,0) node[anchor=north west] {$ x $};
			\draw[domain=0:7, samples=80, thick] plot (\x, {2*exp(-(\x - 3.5)^(2)*0.6});
			\coordinate (o) at (3.5,1);
			\coordinate (d) at (1.1,0);
			\draw[ultra thick, <->, blue, arrows = {Stealth}-{Stealth}] ($ (o) + (d) $) -- ($ (o) - (d) $);
		\end{tikzpicture}
		\label{Gausverteilung}
		\caption{Die Wahrscheinlichkeitsverteilung einer Gaußkurve. Der Pfeil soll die Standartabweichung darstellen. Dieses Bild wurde mit dem \LaTeX Paket Tikz erstellt.}
	\end{figure}
	
	$ \hat{A} = \hat{x} $, $ \hat{B} = \hat{p} $. $ \left[\hat{x}, \hat{p}\right] = i \hbar $
	\begin{equation*}
	\Delta A \cdot \Delta B \ge \frac{1}{2} \left| \langle \left[\hat{A}, \hat{B}\right] \rangle \right|
	\end{equation*}
	\begin{equation*}
	\Delta x \cdot \Delta p \ge \frac{1}{2} \left| i \hbar \right| = \frac{\hbar}{2}
	\end{equation*}	
\end{enumerate}

\subsection*{Morgen:}

Operatoren die vertauschen: Drehimpulsoperator $ \vec{l} $ mit den Komponenten $ l_x, l_y, l_z $ und $ l^2 $. Es gilt $ \left[l^2, l_z\right] = 0 $
\begin{equation*}
\Delta l^2 \cdot \Delta l_z \ge 0
\end{equation*}
Man kann also Zustände finden, bei denen $ \Delta l^2 = 0 $; $ \Delta l_z = 0 $ sind. Diese Zustände können im Prinzip existieren und verletzen die Unschärferelation nicht! Diese Zustände sind dann gleichzeitig Eigenzustände von $ l^2 $ und $ l_z $.

\subsubsection*{Exkurs: Varianz und Standardabweichung in der Quantenmechanik}

Wellenfunktion $ \Psi(x) $ mit Wahrscheinlichkeit $ P(x) = |\Psi(x)|^2 $
\begin{align*}
\mu &= \int x P(x) \dd x = \int x | \Psi(x) | ^2 \dd x \\
\sigma   &= \int x^2 P(x) \dd x = \int x^2 | \Psi(x) | ^2 \dd x
\end{align*}
Die Varianz ist definiert als:
\begin{align*}
\sigma^2 = \int (x - \mu)^2 P(x) \dd x &= \int (x^2 + \mu^2 - 2 \mu x) P(x) \dd x \\
&= \int x^2 P(x) \dd x + \mu^2 \int P(x) \dd x - 2 \mu \int x P(x) \dd x\\
&= \int x^2 P(x) \dd x + \mu^2 - 2 \mu \mu = \int x^2 P(x) \dd x - \mu^2 \\
&= \langle x ^2 \rangle - \langle x \rangle ^2
\end{align*}

% 26.04.19

\subsection*{Programm Heute}

\begin{itemize}
	\item Drehimpulsoperator
	\item Kugelflächenfunktionen (Wasserstoffatom)
	\item Vektormodell (klassische Darstellung)\\
	\lcom{Macht es leichter z.B. die Wechselwirkung zwischen Drehimpulsoperator und Magnetfeld zu verstehen. Dieses klassische Modell macht voraussagen über die QM.}
	\item Experimente (Spektrum des Wasserstoffatoms)
	\item Schrödinger Gleichung des Wasserstoffatoms
\end{itemize}

\section{Drehimpulsoperator}


\begin{minipage}{.5\linewidth}
	\begin{equation*}
	\vec{l} = \vec{r} \times \vec{p} = \vec{r} \times m \vec{v}
	\end{equation*}
	\begin{align*}
	\vec{r} & \Rightarrow \hat{\vec{r}} = \vec{r} \\
	\vec{p} & \Rightarrow \hat{ \vec{p}} = - i \hbar \vabla \\
	\vec{l} & \Rightarrow \hat{\vec{l}} = \vec{r} \times (- i \hbar \vabla) = - i \hbar \vec{r} \times \vabla
	\end{align*}
\end{minipage}%
\begin{minipage}{.5\linewidth}
	\flushright
	%t1:
	\begin{tikzpicture}[scale=.6]
		\draw[thick, ->] (0,0) -- (0,3);
		\draw (0,0) ellipse(2cm and 1cm);
		\coordinate (a) at (0,-1);
		\node[right] at (0,2) {$\vec{\mu}$};
		\node[anchor=north west] at (a) {$q,m$};
		\draw[very thick,->] (0,0) -- node[left] {$ \vec{r} $} (a);
		\draw[thick,->] (a) -- (2.5,-1) node[right] {$\vec{v}$};
	\end{tikzpicture}
\end{minipage}%
\\[5pt]
\begin{equation*}
\hat{\vec{l}} = - i \hbar \begin{pmatrix}
\vec{u}_x & \vec{u}_y & \vec{u}_x \\
x & y & z \\
\prt{}{x} & \prt{}{y} & \prt{}{z}
\end{pmatrix} = - i \hbar
\end{equation*}

%T2
\begin{tikzpicture}[scale=0.6]
	\draw[->] (0,0) -- (3,0) node[anchor=north west] {$ y $};
	\draw[->] (0,0) -- (0,3) node[anchor=south east] {$ z $};
	\draw[->] (0,0) -- (-2,-2) node[anchor=north east] {$ x $};
	\draw[red, thick, ->] (0.5,0.2) -- node[above] {$ \vec{u_y} $} ++(1,0);
	\draw[red, thick, ->] (-0.2,0.5) -- node[left] {$ \vec{u_z} $} ++(0,1);
	\draw[red, thick, ->] (0.0,-0.3) -- node[anchor=north west] {$ \vec{u_x} $} ++(-0.8,-0.8);
\end{tikzpicture}

\begin{align*}
l_x &= - i \hbar \left\{ y \prt{}{z} - z \prt{}{y} \right\} \\
l_y &= - i \hbar \left\{ z \prt{}{x} - x \prt{}{z} \right\} \\
l_z &= - i \hbar \left\{ x \prt{}{y} - y \prt{}{x} \right\}
\end{align*}
\begin{align*}
\left[l_x, l_y\right] = l_x l_y - l_y l_x \neq 0
\end{align*}

\subsection*{Vertauschungsregeln}

\begin{align*}
\left[l_x, l_y\right] &= i \hbar l_z \\
\left[l_y, l_z\right] &= i \hbar l_x \\
\left[l_z, l_x\right] &= i \hbar l_y
\end{align*}
Das Betragsquadrat berechnet sich wie folgt: $ l^2 = l_x^2 + l_y^2 l_z^2 $

\subsection*{Vertauschungregeln}

\begin{equation*}
\left[l^2, l_x\right] = \left[l^2, l_y\right] = \left[l^2, l_z\right] = 0
\end{equation*}
Wir werden bevorzugt $ l_z $ verwenden.\\[10pt]
\noindent
Die Eigenzustände von $ l_z $
\begin{equation*}
l_z = - i \hbar \left\{ x \prt{}{y} - y \prt{}{x} \right\} = - i \hbar \prt{}{\varphi}
\end{equation*}

%T3

\noindent
$ l_z \Rightarrow $ Drehung um die $ z $-Achse\\[10pt]
\noindent
Wir suchen die Operatoren $ \Phi(\varphi) $. Hierzu stellen wir eine Eigenwertgleichung auf und lösen diese.
\begin{equation*}
l_z \Phi(\varphi) = m \hbar \Phi(\varphi) \quad \Rightarrow \quad - i \cancel{\hbar} \prt{}{\varphi} \Phi(\varphi) = m \cancel{\hbar} \Phi(\varphi) \quad \Rightarrow \quad \frac{\partial \Phi}{\Phi} = i m \partial \varphi
\end{equation*}
\begin{equation*}
\int \frac{\partial \Phi}{\Phi} = \int i m \partial \varphi \quad \Rightarrow \quad \Phi(\varphi) = a e^{i m \varphi}
\end{equation*}
Aufgrund der Definition von $ \varphi $ erwarten wir, dass unsere Funktion bei den Winkeln $ \varphi_0 $ und $ \varphi_0 + n \cdot 2 \pi $ ($ n \in \mathbb{Z} $) gleich sind. $ \Phi(\varphi_0) = \Phi(\varphi_0 + 2 \pi) $
\begin{equation*}
\Rightarrow \quad \cancel{a} \cancel{e^{i m \varphi_0}} = \cancel{a} \cancel{e^{i m \varphi_0}} e^{i m 2 \pi} \quad \Rightarrow \quad e^{i m 2 \pi} = 1
\end{equation*}
\begin{align*}
m &= 0 \\
m &= 1 \quad \Rightarrow \quad e^{i 2 \pi} = 1 ! \\
m &= 2 \quad \Rightarrow \quad e^{i 4 \pi} = 1 ! \\
m &= -1 \quad \Rightarrow \quad e^{-i 2 \pi} = 1 ! \\
m &= -2 \quad \Rightarrow \quad e^{-i 4 \pi} = 1 ! \\
\end{align*}
\begin{equation*}
m = 0, \pm 1, \pm 2, \pm 3, \dots
\end{equation*}
\begin{equation*}
\rmbox{m = \tx{Magnetische Quantenzahl}}
\end{equation*}
$ \Rightarrow $ Zeemann Effekt
\begin{equation*}
l_x \Phi(\varphi) = m \hbar \Phi_m(\varphi) \qquad \qquad \Phi_m(\varphi) = a e^{i m \varphi}
\end{equation*}
\begin{equation*}
\int_{0}^{2 \pi} \dd \varphi \Phi_{m}^{*}(\varphi) \Phi_m(\varphi) = 1 \quad \Rightarrow \quad a = \frac{1}{\sqrt{2 \pi}}
\end{equation*}
\begin{equation*}
\rmbox{\Phi_m(\varphi) = \frac{1}{\sqrt{2 \pi}} e^{i m \varphi}}
\end{equation*}

\subsubsection*{Eigenzustände \texorpdfstring{$ l^2 $}{l2}}

\begin{equation*}
\left\{ \begin{array}{c}
l^2 \mathcal{Y}_{l,m} (\theta, \varphi) = l(l + 1) \hbar^2 \mathcal{Y}_{l,m} (\theta, \varphi) \\
l_z \mathcal{Y}_{l,m} (\theta, \varphi) = m \hbar \mathcal{Y}_{l,m} (\theta, \varphi)
\end{array} \right.
\end{equation*}
$ \hat{l^2} = l^2 $, $ \hat{l_z} = l_z $; beide Ausdrücke sind Operatoren, auch wenn sie ohne Dach geschrieben werden.\\[10pt]
Operatoren $ \hat{A} \rho(\vec{r}) = a \rho (\vec{r}) \quad \Rightarrow $ Eigenzustände und Eigenwerte.\\[10pt]
$ m =  $ magnetische Quantenzahl\\
$ l =  $ Drehimpuls Quantenzahl
\begin{equation*}
\mathcal{Y}_{l,m} (\theta, \varphi) \propto e^{i m \varphi} P_{l}^{m}(\cos(\theta)) \cdot a
\end{equation*}
$ P_{l}^{m} $ sind die \textbf{Legendre Polynome}.\\[10pt]
Wir haben bereits gesehen, dass $ m = 0, \pm 1, \pm 2, \dots \in \mathbb{Z} $ und $ l = 0, 1, 2, \dots \in \mathbb{N} $ sein müssen. Es gilt $ - l \le m \le l $.\par
Also $ m = - l, m = - l + 1, m = - l + 2, \dots, m = 0, \dots, m = l - 2, m = l - 1, m = l  $
\begin{equation*}
\int \dd \Omega \mathcal{Y}_{l, m}^{*}(\theta, \varphi) \mathcal{Y}_{l,m} (\theta, \varphi) = \delta_{l l'} \delta_{m m'} \qquad \qquad \dd \Omega = \sin \theta \dd \theta \dd \varphi
\end{equation*}
\begin{equation*}
\begin{array}{c}
l' = l \\
m' = m
\end{array} \quad \Rightarrow \quad \int \dd \Omega \left|\mathcal{Y}_{l,m} (\theta, \varphi)\right|^2 = 1
\end{equation*}
\begin{equation*}
l' = l \quad \Rightarrow \quad \delta_{ll'} = 0 \quad \Rightarrow \quad \int\dd \Omega \mathcal{Y}_{0,m}^{*} (\theta, \varphi) \mathcal{Y}_{0,m} (\theta, \varphi) = 0
\end{equation*}

\subsection*{Kugelflächenfunktionen}

$ l=0 $, $ m=0 \quad \Rightarrow \quad $ 
$$ \mathcal{Y}_{0,0} (\theta, \varphi) = \frac{1}{\sqrt{4 \pi}} $$
$ l=1 $, $ m=-1 $
\begin{equation*}
\mathcal{Y}_{1, -1} (\theta, \varphi) = \sqrt{\frac{3}{8 \pi}} \ub{\sin \theta}_{P_{l}^{m}(\cos \theta)} e^{- i \varphi}
\end{equation*}
$ l=1 $, $ m=0 $
\begin{equation*}
\mathcal{Y}_{1, 0} (\theta, \varphi) = \sqrt{\frac{3}{4 \pi}} \cos \theta
\end{equation*}
$ l=1 $, $ m=1 $
\begin{equation*}
\mathcal{Y}_{1, 0} (\theta, \varphi) = \sqrt{\frac{3}{8 \pi}} \cos \theta e^{i \varphi}
\end{equation*}
\folie{Betragsquarat der Kugelflächenfunktionen} (Darstellung der Elektronen-Orbitale)\\[10pt]
$ \mathcal{Y}_{l,m}(\theta, \varphi) $
\begin{align*}
l=0 \quad &\Rightarrow \quad b-\tx{Obrital} \\
l=1 \quad &\Rightarrow \quad p-\tx{Obrital} \\
l=2 \quad &\Rightarrow \quad d-\tx{Obrital} \\
l=3 \quad &\Rightarrow \quad f-\tx{Obrital}
\end{align*}

\section{Vektormodell}

Die Kugelflächenfunktionen sind die Eigenzustände von $ l^2 $ und $ l_z $ und liefern die Eigenwerte $ l(l+1) \hbar^2 $ und $ m \hbar $.\\[5pt]
Die Länge von $ \vec{l} $ ist $ \sqrt{l(l+1) \hbar^2} $, die von der $ z $-Komponente $ l_z $ ist $ m \hbar $.

%T4
\begin{figure}[h]
\begin{tikzpicture}[decoration={brace,amplitude=1.5ex}]
\draw[->] (0,0) -- (0,3) node[above] {$z$};
\draw[->] (0,0) -- (3,0) node[right] {$y$};
\draw[->] (0,0) -- (-1.2,-2) node[below] {$x$};
\draw[->] (0,0) -- (2.7,1.2);
\draw[dashed] (2.5,1.1) -- (2.5,0);
\draw[decorate]  (2.55,1.1) -- (2.55,0);
\node[right] at (2.7,0.6) {$m*\hbar$};
\draw[decorate] (0,0) -- (2.5,1.13);
\node[rotate=22.93] at (1,1) {$\sqrt{l(l+1)\hbar^2}$};
\end{tikzpicture}
\centering
\caption{Grafische Darstellung des Vektormodels}
\end{figure}
Klassisch wissen wir $ z $-Komponente Länge $ |\vec{l}| $ und müssen für die anderen beiden Komponenten zurück zur QM.

\begin{equation*}
\langle l_x \rangle = \langle \mathcal{Y}_{l,m}(\theta, \varphi) | l_x | \mathcal{Y}_{l,m}(\theta, \varphi) \rangle = \int \mathcal{Y}_{l,m}^{*}(\theta, \varphi) l_x \mathcal{Y}_{l,m}(\theta, \varphi) \overset{*}{=} 0
\end{equation*}
$ (*) $ kann mathematisch gezeigt werden, ist aber nicht Teil der Vorlesung.\\[10pt]
Das selbe gilt auch für $ l_y $: $ \langle l_y \rangle = 0 $
%T5
\begin{minipage}[h]{0.45\textwidth}
\begin{tikzpicture}[decoration={brace,amplitude=10pt}]
\draw[->,thick] (0,0) -- (0,3.5) node[above] {$z$};
\draw[->,thick] (0,0) -- (3.5,0) node[right] {$y$};
\draw[->,thick] (0,0) -- (-1.4,-2.2) node[below] {$x$};
\draw[decorate] (0,0) --  (0,2);
\node[left] at (-0.25,1.1) {$m\hbar$};
%\draw[dashed] (0,2) -- (2.5,2);
\draw[->] (0,0) -- (2.55,2);
\draw[decorate] (2.5,2) -- (0,0);
\node at (2.3,0.5) {$\sqrt{l(l+1)\hbar^2}$};
\draw[->] (0,0) -- (-2.55,2);
%\draw[decorate] (0,0) -- (-2.5,2);
%\node[rotate=-38.6] at (-1.7,0.5) {$\sqrt{l(l+1)\hbar^2}$};
\draw[dashed,red] (0,1.9) ellipse (2.55 and 1);
\end{tikzpicture}\\
\captionof{figure}{Intuitives Model wobei die rote Kreisbahn den Abstand $\sqrt{l(l+1)\hbar^2}$ zum Mittelpunkt hat und parallel zur $x,y$-Achse liegt}
\end{minipage}
\begin{minipage}[h]{0.35\textwidth}
	\begin{align*}
	| \vec{l} | &= \sqrt{ l ( l+1) \hbar^2} \\
	l_z &= m \hbar \\
	\langle l_x \rangle &= 0 \\
	\langle l_y \rangle &= 0
	\end{align*}
\end{minipage}
\newpage
\textbf{Beispiel:}\\
$ l=2 $, $ m = -2, -1, 0, 1, 2 $\\[5pt]
$ \Rightarrow |\vec{l}| = \sqrt{6}\hbar $

%T6
\begin{minipage}[t]{0.5\textwidth}
\begin{tikzpicture}[decoration={brace,amplitude=10pt}]
\draw[->,thick] (0,-2.5) -- (0,2.5) node[above] {$z$};
%\centerarc[](0,0)(0:360:2);
\draw[->] (0,0) -- (2,0);
\draw[->] (0,0) -- (-2,0);
\node at (2.6,0) {$m=0$};

\draw[->] (0,0) -- (1.72,1);
\draw[->] (0,0) -- (-1.72,1);
\draw[dashed,red] (0,1) ellipse (1.72 and 0.3);
\node at (1.6,2) {$m=2$};
	
\draw[->] (0,0) -- (0.8,1.83);
\draw[->] (0,0) -- (-0.8,1.83);
\draw[dashed,red] (0,1.83) ellipse (0.8 and 0.1);
\node at (2.35,1) {$m=1$};
	
\draw[->] (0,0) -- (1.72,-1);
\draw[->] (0,0) -- (-1.72,-1);
\draw[dashed,red] (0,-1) ellipse (1.72 and 0.3);
\node at (2.5,-1) {$m=-1$};
	
\draw[->] (0,0) -- (0.8,-1.83);
\draw[->] (0,0) -- (-0.8,-1.83);
\draw[dashed,red] (0,-1.83) ellipse (0.8 and 0.1);
\node at (1.7,-2) {$m=-2$};
\end{tikzpicture}
\centering
\captionof{figure}{Präzession um die $ z $-Achse hängt von der Quantenzahl $ m $ ab.}
\end{minipage}
%T7
\begin{minipage}[t]{0.5\textwidth}
\begin{tikzpicture}
\draw[->] (0,0) -- (0,2) node[above] {$\vec{B}$};
\draw[->] (0,0) -- (2,0) node[right] {$\vec{l}$};
\draw[->] (0,0) -- (1.3,1.2) node[right] {$\vec{l}$};
\draw[->] (0,0) -- (-0.8,1.6);
\end{tikzpicture}
\centering
\captionof{figure}{Zeeman Effekt}
\end{minipage}

\section{Experimente: Wasserstoffatom Spektrum}

Präsentation auf Folien: Wasserstoffatom\\[5pt]
\folie{Balmer Series: Wasserstoffatom}\\
Es gibt verschiedene Zustände im Atom. Man misst das Licht, dass von diesem Atom emittiert wird mit einem Spektrometer. Man erhält Spektrallinien (zunächst einmal die \textbf{Balmer Serie}). Beispielsweise die schwarzen Absorptionslinien im Sonnenspektrum oder diskrete Emissionslinien im Wasserstoffspektrum. Die Linien befinden sich im sichtbaren Spektrum und im nahen UV Die Position der Linien führte auf die \textbf{Balmer Gleichung}:
\begin{equation*}
\lambda = B \left(\frac{m^2}{m^2 - 4}\right)
\end{equation*}
\folie{Lyman Series: Wasserstoffatom}
Später wurde dann die \textbf{Lyman Serie}. Diese Linien sind eher im UV Bereich zu finden. Auch für ihre Positionen konnte eine Gleichung aufgestellt werden.
\begin{equation*}
\lambda = \frac{1}{R_H} \left(\frac{m^2}{m^2 - 1}\right)
\end{equation*}
\folie{Bohrsches Atommodell}\\
Im Bohrschen Atommodell geht man von festen Umlaufbahnen der Elektronen um den Atomkern aus. Bohr hat eine Quantisierung des Drehimpulses eingeführt als die Stationären Zustände der De-Broglie Wellenlänge der Elektronen.\\
Mit der Quantisierung der Umlaufbahnen kommt man zu Schlussfolgerung, dass die Energie der Elektronen nicht beliebig sondern diskret ist. Durch die Energie der emittierten Photonen konnte die \textbf{Rydberg-Formel} aufgestellt werden, die die Wellenlängen der Balmer- und der Lyman-Serie beschreibt.
\begin{equation*}
\lambda = \frac{h c}{R_{y}} \left(\frac{m^2 n^2}{m^2 - n^2}\right)
\end{equation*}
\folie{Rydberg Saal in Lund}\\
Bild der Originalen Gleichung von Rydman an einer Wand verewigt.\\[10pt]
Wie man an der allgemeineren Rydberg-Formel kann man erkennen, dass die Balmer Serie ein Spezialfall für $ n = 2 $ und die Lyman-Serie ein Spezialfall für $ n = 1 $ ist.\\
\folie{Übergänge zwischen stationären Zuständen}\\
Alle weiteren bekannten Serien wie: Balmer, Lyman, Paschen, Brackett, und Pfund.

\subsection*{Morgen}

\begin{itemize}
	\item Energie des Wasserstoffatoms
	\item Korrespondenzprinzip: ($ \vec{r} \to \vec{r} $, $ \vec{p} \to - i \hbar \vabla $) Zeitunabhängige Schrödingergleichung
	\item Energiezustände und Eigenwerte
	\item Drehimpulsoperator $ \vec{l} $ und Kugelflächenfunktionen $ \mathcal{Y}_{l,m}(\theta, \varphi) $
\end{itemize}

% 2.05.19

\subsection*{Lach und Sachgeschichten, heute mit:}
\begin{itemize}
	\item Wasserstoffatom mit Schrödingergleichung
	\begin{itemize}
		\item Wellenfunktion (Quantenzahlen)
		\item Energieniveaus (Entartung)
	\end{itemize}
	\item Spektroskopie
	\begin{itemize}
		\item Wasserstoffatom
		\item Spektrometer
		\item Balmar-Serie
	\end{itemize}
	\item Zeeman Effekt
	\item und natürlich mit der Maus
\end{itemize}
%T1
\hft\\
Energie des Wasserstoffatoms:
$$E=\frac{p_k^2}{2M}+\frac{p_e}{2m_e}+V(r)=\ub{\frac{p_k^2}{2M}}_\text{Kern}+\ub{\frac{p_e^2}{2m_e}}_\text{Elektron}-\ub{\frac{e^2}{4\pi\epsilon_0}\frac{1}{r}}_\text{Coulomb}$$
$$\vec{R_k}=\vec{R_{cM}}+\vec{r_k}$$
$$\vec{R_1}=\vec{R_{cM}}+\vec{r_1}$$
Energie des Wasserstoffatoms im Schwerpunktsystem:
$$E=\frac{p^2_{cM}}{2M_{tot}}+\frac{p^2}{2\mu}-\frac{e^2}{4\pi\epsilon_0}\frac{1}{r}$$
\begin{itemize}
	\item $M_{tot}=M+m$
	\item $\vec{m_{cM}}=(m_e+M)\vec{v_{cM}}$ Impuls des Schwerpunktes
	\item $\mu=\frac{m_eM}{m_e+M}=\frac{m_e}{\frac{m_e}{M}+1}\approx m_e$ Reduzierte Masse mit $m_e \ll M$
	\item $\vec{p}=\frac{1}{M+m_e}[m_e\vec{p}_k-M\vec{p_1}]$ relatives Moment
\end{itemize}

Hamiltonoperator lässt sich in ähnlicher Weise auseinander ziehen.
$$E_{cM}=\frac{p_{cm}^2}{2M_{tot}} \qquad \qquad E_0=\frac{p^2}{2\mu}-\frac{e^2}{4\pi \epsilon_0}\frac{1}{r}$$
$$E\Rightarrow \hat{H}\equiv H$$
$$H=H_{cM}+H_0$$
\begin{align*}
&H_{cm}=\frac{p^2_{cM}}{2M_{tot}}\overset{*}{=}-\frac{\hbar^2}{2M_{tot}}\nabla^2_{cM}\\
&H_0=\frac{p^2}{2\mu}-\frac{e^2}{4\pi \epsilon_0}\frac{1}{r}=-\frac{\hbar^2}{2\mu}\nabla^2-\frac{e^2}{4\pi \epsilon_0}\frac{1}{r}
\end{align*}
$(*)$ $p_{cM}$ als Operator da $H$ ein Operator ist: $ p_{cM}=-i\hbar \nabla_{cM} $
Eigenzustände für $H_{cM}$ und $H_0$:\\
\ \\
$H_{cm}$ ist der Hamilton für ein freies Teilchen
$$\overset{\text{Schrödinger GL}}{\Rightarrow} H_{cM}*\Psi_{cM}=E_{cM}\Psi_{cM}\Rightarrow\frac{-\hbar^2}{2M_{tot}}\nabla^2\Psi_{cM}=E_{cM}\Psi_{cM}$$
Für die Wellengleichung eines freien Teilchens gilt:
$$\Psi_{cM}(\vec{R_{cM}})=Ce^{i(\vec{p_{cM}}\vec{R_{cM}})}\frac{1}{\hbar}$$
$$\Rightarrow -\frac{\hbar^2}{2M_{tot}}\nabla^2\Psi_{cM}=-\frac{\cancel{\hbar^2}}{2M_{tot}}\frac{-\vec{p_{cM}^2}}{\cancel{\hbar^2}}\cancel{\Psi_{cM}}=E_{cM}\cancel{\Psi_{cM}}$$
$$\Rightarrow \frac{\vec{p}^2_{cM}}{2M_{tot}}=E_{cM}$$
Schwerpunkt:
$$H_0=\frac{p^2}{2\mu}-\frac{e^2}{4\pi\epsilon_0}\frac{1}{r}\overset{\mu=m_e}{\approx}\frac{p^2}{2m_e}-\frac{e^2}{4\pi\epsilon_0}$$
Relative Bewegung des Wasserstoffs:
$$\Rightarrow \rmbox{-\frac{\hbar}{2m_e}\nabla^2\Psi(\vec{r})-\frac{e^2}{4\pi\epsilon_0}\frac{1}{r}\Psi(\vec{r})=E_0\Psi(\vec{r})}$$
Lösen der Gleichung mit Kugelkoordinaten:
$$\nabla^2=\frac{1}{r^2}\frac{\partial}{\partial r}(r^2\frac{\partial}{\partial r})-\frac{l^2}{\hbar^2r^2}$$
$l^2=$ Drehimpulsoperator deshalb verwenden wir Kugelflächenfunktionen da wir dort diesen auch vorfinden.\\
Aufspalten der Wellenfunktion in einen Radial und einen Winkelanteil:
$$\Psi(r)=\ub{R_{E_0,l}(r)}_{\text{Radialteil}}\text{ }\ub{\ssf_{l,m}(\theta,\phi)}_{\text{Winkelanteil}}$$
$$\Rightarrow -\frac{\hbar^2}{2m_e}\left[\frac{1}{r^2}\frac{\partial}{\partial r}(r^2\frac{\partial}{\partial r})-\frac{l(l+1)\hbar^2}{\hbar^2r^2}\right]R_{E_0,l}(r)-\frac{e^2}{4\pi\epsilon_0}\frac{1}{r}R_{E_0,l}(r)=E_0R_{E_0,l}(r)$$
$$l^2\ssf_{l,m}(\theta,\phi)=l(l+1)\hbar^2\ssf_{l,m}(\theta\phi)$$
$$U_{E_0,l}(r)=rR{E_0,l}(r)$$
\begin{equation}
\frac{d^2U_{E_0,l}(r)}{dr^2}+\frac{2m_e}{\hbar}\left[E_0-V_{eff}(r)\right]U_{E_0,l}(r)=0
\end{equation}
$$V_{eff}(r)=\ub{-\frac{l^2}{4\pi\epsilon_0}\frac{1}{r}}_{\text{Coulomb Potential}}+\ub{\frac{l(l+1)\hbar^2}{2m_er^2}}{\text{Zentrifugalpotential}}$$
\folie{Zentrifugalpotential}\\
Durch lösen von (1.1) erhält man die Hauptquantenzahlen $n$\\
$$\Psi_{l,n,m}(r)=R_{n,l}(r)\ssf_{l,m}(\theta,\phi)$$
Dies sind die drei Quantenzahlen die die Wellenfunktion eines Wasserstoffatoms beschreiben.
\begin{itemize}
\item $n=1,2,3,4,...$
\item $0\leq l \leq (n-1)$
\item $-l\leq m \leq l$
\end{itemize}
\subsection*{Beispiel}
$$n=2\Rightarrow l=0 \rightarrow m=0$$
$$\Rightarrow l=1 \rightarrow m=1,m=0,m=-1$$
Energie hängt im Wasserstoffatom nur von der Hauptquantenzahl ab, im Gegensatz zur Wellenfunktion die von drei abhängig ist.\\
$$E_n=-\frac{1}{2n^2}\frac{e^2}{4\pi\epsilon_0}\frac{m_e}{\hbar^2}$$
Radialer Anteil der Wellenfunktion mit $n=1\Rightarrow l=0$\\
$$R_{10}=2(\frac{1}{a_0})^{\frac{3}{2}}e^{-\frac{r}{a_0}} \qquad \qquad \ub{a_0=\frac{4\pi \epsilon_0 \hbar^2}{m_el^2}}_{\text{Bohrradius (erste Umlaufbahn)}}$$
\folie{Radialer Anteil der Wellenfunktion: $R_{n,l}(r)$}

\begin{equation*}
\left|\psi_{nkm}(\vec{r})\right|^2  \quad \Rightarrow \quad r^2 \dd r \sin \theta \dd \theta \dd \varphi
\end{equation*}
\begin{align*}
\int \left|\psi_{nlm}(\vec{r})\right|^2 \dd \vec{r} &= \int_{r_0}^{r_1} \dd r \int_{\Omega} \sin \theta \dd \theta \dd \varphi \left|\psi_{nlm}(\vec{r})\right|^2 \\
&= \int_{r_0}^{r_1} r^2 \dd r \int_{\Omega} \sin \theta \dd \theta \dd \varphi \left|R_{nl}(\vec{r})\right|^2 \cdot \left|\mathcal{Y}_{lm}(\theta, \varphi)\right|^2 \\
&= \int_{r_0}^{r_1} r^2 \left|R_{nl}(\vec{r})\right|^2  \dd r \ub{\int_{\Omega} \sin \theta \left|\mathcal{Y}_{lm}(\theta, \varphi)\right|^2 \dd \theta \dd \varphi}_{\Omega = 4 \pi} \\
&= \int_{r_0}^{r_1} r^2 \left|R_{nl}(\vec{r})\right|^2 \dd r
\end{align*}

\subsection{Spektrometer}

%T2
\hft
\folie{Spektrometer}

NIST Database für Spektrallinien von Atomen und Molekülen. \texttt{https://www.nist.gov/pml/atomic-spectra-database} und \verb|https://physics.nist.gov/PhysRefData/ASD/lines_form.html|.\\[10pt]
\noindent
Balmer Linie
\begin{align*}
n = 3 \quad &\Rightarrow \quad n = 2 ! \\
n = 4 \quad &\Rightarrow \quad n = 2 ! \\
n = 5 \quad &\Rightarrow \quad n = 2 !? \\
n = 6 \quad &\Rightarrow \quad n = 2 !?? \\
\end{align*}
\folie{Transmission  von $ SiO_2 $ bei verschiedenen Wellenlängen}\\
\folie{Extreme ultraviolet (XUV) Spektrometer}


%\bibliographystyle{plain}
%\bibliography{literature}
%\addcontentsline{toc}{section}{Literatur}

\end{document}
