% packages
\PassOptionsToPackage{dvipsnames}{xcolor}  % needed to get colors working in certain environments
\documentclass[titlepage,11pt,a4paper,ngerman]{report}
\usepackage[utf8]{inputenc}
\usepackage[T1]{fontenc}
\usepackage[german]{babel}
\usepackage{graphicx}
\usepackage{wrapfig}  % used for wrapping a figure alternative to using minipages
\usepackage{amsmath}
\usepackage{amsfonts}
\usepackage{amssymb}
\usepackage[hidelinks]{hyperref}  % removes coloring for links
\usepackage{cleveref}
\usepackage{tikz}
\usepackage{tikz-cd}
\usepackage{nicefrac}  % adds \nicefrac alternative to regular \frac
\usepackage{mathtools}
\usepackage{enumerate}
\usepackage{cancel}  % adds \cancel which adds strikethrough
\usepackage{tocloft}
\usepackage{tcolorbox}
\usepackage{bm}  % adds \bm to make symbols bold, replaces outdated \boldsymbol
\usepackage[shortlabels]{enumitem}
\usepackage{placeins}
\usepackage{booktabs}
\usepackage{wasysym}
\usepackage{capt-of}

\usepackage[margin=1in]{geometry}  % changes the margins on all pages
\usepackage{url}

%SI-unix
%\usepackage{array}
\usepackage[per=slash,
            decimalsymbol=comma,
			loctolang={DE:ngerman,UK:english},
			]{siunitx}	
\sisetup{locale = DE}

\usetikzlibrary{calc}
\usetikzlibrary{decorations.pathmorphing,patterns}
\usetikzlibrary{arrows}
\usetikzlibrary{decorations.pathreplacing}
%\usetikzlibrary{snakes}

% Andrez:
%\usepackage{epigraph}  % adds \epigraph used to add fancy quotes to the beginning of chapters
%\usepackage{fancyhdr}
%\setlength{\parskip}{1em}
%\setlength{\headheight}{35pt}
%\setlength\epigraphwidth{.8\textwidth}
% alt math font
%\usepackage{eulervm}  % switches to alternate math font, usually requires extra download


%Environments und Newcommands:


% general commands

% zu zeigen symbol
\newcommand{\zz}{\fontfamily{cmss} \selectfont{Z\kern-.61em\raise-0.7ex\hbox{Z}:}}
% build over
\newcommand{\bov}[2]{\buildrel{#2} \over{#1}}
% better looking := (defined as)
\newcommand*{\defeq}{\mathrel{\vcenter{\baselineskip0.5ex \lineskiplimit0pt \hbox{\scriptsize.}\hbox{\scriptsize.}}}=}
\newcommand*{\eqdef}{=\mathrel{\vcenter{\baselineskip0.5ex \lineskiplimit0pt \hbox{\scriptsize.}\hbox{\scriptsize.}}}}

% integral differential d
\newcommand{\dif}{\mathop{}\!\mathrm{d}}
\newcommand{\difi}[1]{\mathrm{d}#1\mathop{}\!}

\newcommand{\prt}[2]{\frac{\partial #1}{\partial #2}}  % used for partial derivatives, tip: input #1 can be left blank
\newcommand{\prd}[2]{\frac{\tx{d} #1}{\tx{d} #2}}  % used for absolute (standart) derivatives, tip: input #1 can be left blank

\newcommand{\dd}{\tx{d}}


%Mathe:
\newcommand{\verteq}{\rotatebox{90}{$\,=$}}  % used for commands below
\newcommand{\equalto}[2]{\underset{\scriptstyle\overset{\mkern4mu\verteq}{#2}}{#1}}  % adds equal to underneath
\newcommand{\equaltoup}[2]{\overset{\scriptstyle\underset{\mkern4mu\verteq}{#2}}{#1}}  % adds equal to above
\newcommand{\custo}[3]{\underset{\scriptstyle\overset{\mkern4mu\rotatebox{-90}{$\,#1$}}{#3}}{#2}}  % same as above but replaces equal sign with input #1
\newcommand{\custoup}[3]{\overset{\scriptstyle\underset{\mkern4mu\rotatebox{-90}{$\hspace{-3pt} #1$}}{#3}}{#2}}
\newcommand{\casess}[4]{\left\{ \begin{array}{ll} {#1} & {#2} \\ {#3} & {#4} \end{array} \right.}  % used to indicate to the reader that something is missing here


%Text:
\newcommand{\tx}[1]{\textrm{#1}}
\newcommand{\const}{\tx{const.}}

\newcommand{\ul}[1]{\underline{#1}}
\newcommand{\ol}[1]{\overline{#1}}
\newcommand{\ub}[1]{\underbrace{#1}}
\newcommand{\ob}[1]{\overbrace{#1}}

\newcommand{\hfw}{\color{RubineRed}\tx{ $\star$hier fehlt was$\star$ } \color{black}}  % used to indicate to the reader that something is missing here 
\newcommand{\hft}{\color{green!80!black}\tx{ $\star$hier fehlt eine Grafik$\star$ } \color{black}}  % used to indicate to the reader that a tikz picture is migging


%Spezielles:


%Theo:
\newcommand{\lag}{\mathcal{L}}  % used for Lagrange function
\newcommand{\ham}{\mathcal{H}}  % used for Hamiltonian function
\newcommand{\gre}{\mathcal{G}}  % used for Green's function
\newcommand{\ssf}{\mathcal{Y}}  % used for Spherical Surface Function (Kugelflächenfunktion)
\newcommand{\eofr}{\vec{E}(\vec{r})}
\newcommand{\pofr}{\Phi(\vec{r})}
\newcommand{\grr}{\mathcal G(\vec{r},\vec{r}')}
\newcommand{\vphi}{\varphi}
\newcommand{\vabla}{\vec{\nabla}}


%LA:
\newenvironment{bew}[1]{\subsection{Bew: #1}}{\hfill$\square$}
\newcommand{\Bew}[2]{\begin{bew}{#1}#2\end{bew}}
\newcommand{\enph}{F: V \to V \textrm{ Endomorphismus}}

\newcommand{\im}{\tx{im}}
\newcommand{\spa}{\tx{span}}
\newcommand{\adj}{\tx{adj}}
\newcommand{\grad}{\tx{grad}}
\newcommand{\ord}{\tx{ord}}

\newcommand{\basis}[3]{\{#1_{#2}, \dots, #1_{#3}\}}
\newcommand{\ska}[2]{\langle #1 , #2 \rangle}  % scalar product of input 1, and 2 can also be used for braket notation
\newcommand{\dmat}[3]{\begin{pmatrix} #1_{#2}&&\\ &\ddots& \\ && #1_{#3} \end{pmatrix}}


%Ex:
\newcommand{\kq}{\frac{1}{4\pi\epsilon_0}}  % writes out the whole constant k from electrostatic
\newcommand{\kqq}{\frac{\mu_0}{4\pi}}  % writes out the constant from magnetostatic
\newcommand{\uind}{U_{\tx{ind}}}
\newcommand{\folie}[1]{\color{gray}[Folie: #1]\color{black}}  % used to tell the reader that there was multimedia content during a lecture
\newcommand{\versuch}[1]{\color{red!50!black} \textbf{Versuch:} \color{black} \textbf{#1}\\ }  % used to tell the reader that there was a live experiment

\newcommand{\mau}{$\buildrel \mathcal{O} \over{\textbf{.}}$}  % the extreme Waldmann exclamation mark recreated in latex by Markus


% Lab commands:
\newcommand\mean{\begin{equation}
\frac{\sum_{i=1}^n x_i}{n}\label{mean}
\end{equation}}  % shortcut for the standard Mean function

\newcommand\meanstd{\begin{equation}
s_x=\sqrt{\frac{1}{n-1}\sum_{i=1}^n(x_i-\overline{x})^2}\label{meanstd}
\end{equation}}  % shortcut for the standard derivative mean function

\newcommand\prodquo{\begin{equation}\left\vert\frac{\Delta z}{z}\right\vert=\sqrt{\left(a\frac{\Delta x}{x}\right)^2+\left(b\frac{\Delta y}{y}\right)^2+\ldots}\textrm{ f\"ur }z=x^a\ y^b\ldots\end{equation}}

\newcommand\tfuncd{\begin{equation}
t=\frac{\vert x_n-y_n\vert}{\sqrt{x_s^2+y_s^2}}
\end{equation}}

\newcommand\tfunc{\begin{equation}
t=\frac{\vert x-y_0\vert}{u_x}
\end{equation}}


% ANDREZ
%\newcommand{\summ}[2]{\sum_{#1}^{#2}}
%\newcommand{\intt}[2]{\int_{#1}^{#2}}
\newcommand{\lcom}[1]{\color{MidnightBlue}#1\color{black}}  % used to indicate to the reader that this content was transcribed from the lecture
\newcommand{\bei}{\emph{Beispiel:}}
\newcommand{\bem}{\emph{Bemerkung:}}


% Boxen:

\tcbuselibrary{theorems}

% mahlt eine box nur um den text mit titel
\newtcbox{\fribox}[1]{nobeforeafter,colback=white,colframe=red!75!black,fonttitle=\bfseries,title=#1,sharp corners,tcbox raise base}

% mahlt eine große box um alles mit titel
\newcommand{\frbox}[2]{\begin{tcolorbox}[colback=white,colframe=red!75!black,fonttitle=\bfseries,title=#1]#2\end{tcolorbox}}

% mahlt eine box nur um den text
\newtcbox{\ribox}{nobeforeafter,colback=white,colframe=red!75!black,sharp corners,tcbox raise base}

% mahlt eine große box um alles was drinnen ist
\newcommand{\rbox}[1]{\begin{tcolorbox}[colback=white,colframe=red!75!black]#1\end{tcolorbox}}

% mahlt eine box um mathe innerhalb mathmode
\newcommand{\rmbox}[1]{\tcboxmath[colback=white,colframe=red!75!black]{#1}}

% super box (looks like regular boxed but wraps around anything)
\newenvironment{supbox}{\begin{tcolorbox}[colback=white,colframe=black,sharp corners,boxrule=.5pt]}{\end{tcolorbox}}

% the Big Black Box, can be used to separate examples or review material from the main text (used for Wiederholung)
\newcommand{\bbb}[2]{\begin{tcolorbox}[colback=white,colframe=black,fonttitle=\bfseries,title=#1,sharp corners,tcbox raise base]#2\end{tcolorbox}}

% array type box with title
\newenvironment{zebox}[1]{\begin{array}{|c|}
		\multicolumn{1}{l}{\tx{#1}} \\
		\hline
		\displaystyle
	}{\\ \hline
\end{array}}

% arrow list
\newlist{arrowlist}{itemize}{1}
\setlist[arrowlist]{label=$\Rightarrow$}


% optional:

\renewcommand{\vec}[1]{\bm{#1}}

% changed because of preferred looks
\renewcommand{\epsilon}{\varepsilon}
\renewcommand{\paragraph}[1]{\subsubsection{#1}}  % changed oddly behaving paragraphs to simple subsubsections which are not numbered nor in the toc

% nur in Theo benutzt !!!
% \renewcommand{\Phi}{\varPhi}

% evtl:
% \renewcommand{\boxed}{\rmbox}


% Tikz definitions:

\def\centerarc[#1](#2)(#3:#4:#5)% Syntax: [draw options] (center) (initial angle:final angle:radius)
{ \draw[#1] ($(#2)+({#5*cos(#3)},{#5*sin(#3)})$) arc (#3:#4:#5); }

\def\checkmark{\tikz\fill[scale=0.4](0,.35) -- (.25,0) -- (1,.7) -- (.25,.15) -- cycle;}  % checkmark used for proofs

\tikzset{
	annotated cuboid/.pic={
		\tikzset{%
			every edge quotes/.append style={midway, auto},
			/cuboid/.cd,
			#1
		}
		\draw [every edge/.append style={pic actions, densely dashed, opacity=.5}, pic actions]
		(0,0,0) coordinate (o) -- ++(-\cubescale*\cubex,0,0) coordinate (a) -- ++(0,-\cubescale*\cubey,0) coordinate (b) edge coordinate [pos=1] (g) ++(0,0,-\cubescale*\cubez)  -- ++(\cubescale*\cubex,0,0) coordinate (c) -- cycle
		(o) -- ++(0,0,-\cubescale*\cubez) coordinate (d) -- ++(0,-\cubescale*\cubey,0) coordinate (e) edge (g) -- (c) -- cycle
		(o) -- (a) -- ++(0,0,-\cubescale*\cubez) coordinate (f) edge (g) -- (d) -- cycle;
		\path [every edge/.append style={pic actions, |-|}]
		%(b) +(0,-5pt) coordinate (b1) edge ["\cubex \cubeunits"'] (b1 -| c)
		%(b) +(-5pt,0) coordinate (b2) edge ["\cubey \cubeunits"] (b2 |- a)
		%(c) +(3.5pt,-3.5pt) coordinate (c2) edge ["\cubez \cubeunits"'] ([xshift=3.5pt,yshift=-3.5pt]e)
		;
	},
	/cuboid/.search also={/tikz},
	/cuboid/.cd,
	width/.store in=\cubex,
	height/.store in=\cubey,
	depth/.store in=\cubez,
	units/.store in=\cubeunits,
	scale/.store in=\cubescale,
	width=10,
	height=10,
	depth=10,
	units=cm,
	scale=.1,
}


% other settings
\hbadness=99999  % removes unnecessary hbadness warnings

% the following are used to circumvent Roman numerals in the toc from running out of space
%\addtolength{\cftchapnumwidth}{10pt}
%\addtolength{\cftsecnumwidth}{10pt}
%\addtolength{\cftsubsecnumwidth}{10pt}
%\renewcommand{\thechapter}{\Roman{chapter}}  % just don't do this :)

% coloring
% for working at night
%\pagecolor{darkgray}
%\color{white}

\begin{document}

\title{
	{\Huge Höhere Mathematik}\\[1em]
	{\Large Vorlesung von Prof. Dr. Harald Ita im Sommersemester 2019}}
\author{Markus Österle \hspace{5pt} Damian Lanzenstiel}
\date{ \today}
\maketitle
\tableofcontents

% Document

\setcounter{chapter}{-1}

\chapter{Einführung}

\section{Wichtige Infos}

\begin{description}
	\item[e-mail] \texttt{harals.ita@physik.uni-freiburg.de}
	\item[Zimmer] 803
	\item[Homepage] \texttt{www.qft.physik.uni-freiburg/Teaching}
	\item[Tutorate] 24. Aprill ab 14:00 Einschreibungsbeginn\\
	60\% sind zum bestehen der Studienleistung erforderlich. Die Teilnahme an der Prüfung ist nicht daran gebunden und kann auch ohne bestehen mitgeschrieben werden.
\end{description}

\section{Inhalt der Vorlesung}

Die Vorlesung orientiert sich stark am Script von Prof. Dittmeier.



\part{Funktionentheorie}

Theorie der Funktionen in einer komplexen Veränderlichen

\chapter{Komplexe Zahlen}

\begin{itemize}
	\item natürliche Zahlen $ \mathbb{N} = \{1, 2, \dots\} $ mit definierten Operatoren $ + $ und $ \times $
	\item ganze Zahlen $ \mathbb{Z} = \{0, \pm 1, \pm 2, \pm 3, \dots\} $ mit den Operationen $ + $ mit Inversion und $ \times $ ohne Inversion
	\item rationale Zahlen $ \mathbb{Q} = \left\{\frac{a}{b} \big| a, b \in \mathbb{Z}, b \neq 0 \right\} $ mit den Operatoren $ + $ und $ \times $ und ihren Inversionen\\[5pt]
	$ x^2 = z $ algebraisch unvollständig, konvergente Folge, die nicht in $ \mathbb{Q} $ liegenden Limes hat (Cauchy Folge \footnote{Mit einer Cauchy Folge kann gezeigt werden, dass eine Folge konvergiert, ohne dass der Limes bekannt ist.}).
	\item reelle Zahlen $ \mathbb{R} = \mathbb{Q} \cup \{ \tx{irrationale Zahlen} \} $. Vollständiger Körper \footnote{Bei einem vollständigen Körper liegen die Grenzwerte aller konvergenter Folgen wieder in dem Körper.} aber algebraisch nicht abgeschlossen.\\[5pt]
	$ x^2 = -1 $ nicht lösbar in $ \mathbb{R} $
	\item komplexe Zahlen $ \mathbb{C} = \mathbb{R}, i $ algebraisch abgeschlossen, vollständiger Körper\\[5pt]
	konstuktion über imaginäre Einheit $ i $ mit $ (i)^2 = (-1) $, Euler 1777
\end{itemize}

\subsection*{Def: komplexe Zahlen}

\begin{enumerate}[a)]
	\item komplexe Zahl $ z $ ist ein Zahlenpaar $ z = (x,y) $ mit $ x, y \in \mathbb{R} $. $ x $ ist der Realteil von $ z $ mit $ \Re(z) = x $ und $ y $ der Imaginärteil von $ z $ mit $ \Im(z) = y $.\\[10pt]
	Definieren wir zwei komplexe Zahlen $ z_1 = (x_1, y_1), z_2 = (x_2, y_2) $, so ist:\\
	die \textbf{Addition} definiert als:\\
	$$ z_1 + z_2 = (x_1 + x_2, y_1 + y_2) $$
	die \textbf{Multiplikation} definiert als:\\
	$$ z_1 \cdot z_2 = (x_1 x_2 + y_1 y_2, x_1 y_2 + x_2 y_1) $$
	\item Das Symbol der Menge der komplexen Zahlen ist $ \mathbb{C} $.
	\begin{equation*}
	\ol{\mathbb{C}} = \mathbb{C} \cup \{-\infty\}
	\end{equation*}
	\item Kurzschreibweise: $ i = (0, 1) $; $ z = (x,y) = x + i \cdot y $
	\item komplex konjugierte Zahl
	$$ z = (x,y) = x + i y \to \ol{z} = (x, -y) = x -  i y $$
	\item Betrag einer komplexen Zahl
	$$ |z| = \sqrt{z \cdot \ol{z}} = \sqrt{x^2 + y^2} $$
	\item Polardarstellung
	\begin{equation*}
	z = (r \cos \varphi, r \sin \varphi) = r \cos \varphi + i \cdot r \sin \varphi
	\end{equation*}
	\begin{equation*}
	\varphi \in (- \pi , \pi] \qquad r \in \mathbb{R}^+
	\end{equation*}
	$ r $ ist der \textbf{Betrag} von $ z $: $ r  = |z| $. $ \varphi $ ist das \textbf{Argument} von $ z $: $ \varphi = \tx{arg}(z) $
	
	% T1
\end{enumerate}

\section{Satz: Rechenregeln im \texorpdfstring{$ \mathbb{C} $}{C}}

für $ z_i \in \mathbb{C} $ gilt:
\begin{equation*}
\ol{z_1 + z_2} = \ol{z_1} + \ol{z_2} \qquad \qquad \ol{z_1 \cdot z_2} = \ol{z_1} \cdot \ol{z_2} \qquad \qquad \ol{\ol{z_1}} = z_1
\end{equation*}
\begin{equation*}
\Re(z) = \frac{1}{2} ( z + \ol{z}) \qquad \qquad \Im(z) = \frac{1}{2i} ( z - \ol{z})
\end{equation*}
\begin{equation*}
| z_1 z_2 | = | z_1 | | z_2 | \qquad | \ol{z} | = | z |
\end{equation*}
\begin{equation*}
| z | \ge 0 \quad \tx{und} \quad | z | = 0 \quad \Rightarrow \quad z = (0,0) = 0 + i 0 = 0
\end{equation*}
\begin{equation*}
| z_1 | + | z_2 | \ge | z_1 + z_2 | \ge | z_1 | - | z_2 |
\end{equation*}

%T2 

\subsection*{Gaußsche Zahlenebene}

$ \mathbb{C} $ bildet einen 2-dimensionale Vektorraum wie $ \mathbb{R}^2 $. Es gibt also eine gemeinsame Struktur mit dem $ \mathbb{R}^2 $, dennoch ist $ \mathbb{C} $ eine Erweiterung.
\begin{enumerate}[a)]
	\item Vektoraddition, Multiplikation mt reeller zahl, Länge und Abstandsbegriff. 
	
	% T3
	
	% T4
	
	\item Multiplikation komplexer Zahlen $ \to $ Darstellung in Polarform
	\begin{align*}
	z_1 z_2 &= ( r_1 \cos \varphi_1 , r_1 \sin \varphi_1) \cdot (r_2 \cos \varphi_2 , r_2 \sin \varphi_2) \\
	& = r_1 r_2 \cdot (\cos \varphi_1 \cos \varphi_2 - \sin \varphi_1 \sin \varphi_2 , \cos \varphi_1 \sin \varphi_2 + \cos \varphi_2 \sin \varphi_1) \\
	& = r_1 r_2 \cdot (\cos(\varphi_1 + \varphi_2) , \sin(\varphi_1 + \varphi_2))
	\end{align*}
	$ \Rightarrow $ Beträge multiplizieren, Argumente addieren\\[5pt]
	Kehrbruch einer komplexen Zahl
	\begin{align*}
	\frac{1}{z} = \frac{\ol{z}}{z \ol{z}} & = \frac{1}{r^2} (r \cos \varphi, - r \sin \varphi) \\
	& = \frac{1}{r} ( \cos(- \varphi), \sin(- \varphi)
	\end{align*}
	mit $ r' = \frac{1}{r} $, $ \varphi' = - \varphi $
	
	% T5
	
	% T6
\end{enumerate}

\subsection*{Riemannsche Sphäre}

Kompaktifizierung der komplexen Zahlen Ebene $ \mathbb{C} $ durch stereographische Projektion: $ \mathbb{\hat{C}} = \ol{\mathbb{C}} = \mathbb{C} \cup \{ -\infty \} $.\par
Es wird also ein Punkt im unendlichen zu $ \mathbb{C} $ hinzugefügt.

% T7

% T8

\noindent
$ N = (0,0,1) $\\
Sphäre mit Radius $ R = 1 $, um Koordinatenuhrsprung in $ \mathbb{R}^3 $. $ \mathbb{C} $ wird identifiziert mit der $ (x,y) $-Ebene.
\begin{description}
	\item[stereographische Projektion] = Zuordnung von Punkten auf Sphäre mit Punkten in $ (x,y) $-Ebene.
\end{description}
Vorschrift: Gerade durch Punkt $ (x_{\Re}, y_{\Im}, 0) $ und den Nordpol $ N $. Durchstoßpunkt = projezierter Punkt auf Sphäre. Bildpunkte: $ \vec{w}(z) $.

\subsection*{Def: Chordaler Abstand}

$ \chi(z_1, z_2) = $ ,, Abstand der Bilder $ \vec{w}_1 = \vec{w}(z_1) $, $ \vec{w}_i = \vec{w}(z_i) $ unter stereographischen Projektion im $ \mathbb{R}^3 $.``
\begin{equation*}
\chi (z_1, z_2) = |\vec{w}(z_1) - \vec{w}(z_2)|
\end{equation*}

% T9

% T10

%vorlesung 26.04.19

\subsection*{Def: Metrik}

(Topologie, Stetigkeit, Limes)\\[10pt]
\noindent
\textbf{Abstandsfunktion:} $ d(\cdot , \cdot) $ auf Menge $ M: \ M \times M \to \mathbb{R}^+ $
\begin{enumerate}[(a)]
	\item $ d(z_1, z_2) \ge 0 \ \forall z_i \in M  \ (\mathbb{C})$ sowie $ d(z_1, z_2) ) 0 \Leftrightarrow z_1 = z_2 $
	\item $ d(z_1, z_2) \le d(z_1, z_3) - d(z_3, z_2) \ \forall z_i \in M $
\end{enumerate}
\emph{Beispiele:}
\begin{enumerate}[(i)]
	\item $ \mathbb{C}: d(z_1, z_2) = | z_1 - z_2 | $
	\item $ \ol{\mathbb{C}} : \chi(z_1, z_2) $ abgeleitet von Abstandsfunktion im $ \mathbb{R}^3 $ $ d(\vec{x}_1 , \vec{x}_2) = \sqrt{(\vec{x}_1 - \vec{x}_2)^2} $
\end{enumerate}

\subsection*{Def: Metrischer Raum}

$$ \tx{Metrischer Raum = Menge + Metrik} $$

\subsection{Polarform und komplexe Wurzeln}

\begin{enumerate}[a)]
	\item Formel von Moivre ($ \to $ Übungen)
	\begin{equation*}
	(\cos \varphi + i \sin \varphi)^n = \cos (n \varphi) + i \sin (n \varphi)
	\end{equation*}
	\item Die $ n $-te Wurzeln $ \xi_n, \circ_n^2, \xi_n^3, \dots, \xi_n^n $ mit
	\begin{equation*}
	\xi_n = \cos \left(\frac{2 \pi}{n}\right) + i \sin \left(\frac{2 \pi }{n}\right)
	\end{equation*}
	\begin{align*}
	\xi_n^n = \left(\cos \left(\frac{2 \pi}{n}\right) + i \sin \left(\frac{2 pi}{n}\right)\right)^2 = \cos \left(\frac{2 \pi n}{n}\right) + i \sin \left(\frac{2 \pi n}{n}\right) = 1
	\end{align*}
	\begin{align*}
	\left(\xi_n^k\right)^n = \xi_n^{kn} &= \cos \left(\frac{2 \pi k n}{n}\right) + i \sin \left(\frac{2 \pi k n}{n}\right) \\
	&= 1 \quad  \qquad \qquad + \qquad  \quad i 0 \qquad = 1
	\end{align*}
	Wurzeln lösen $ z^k = 1 $\\
	Für $ n = 2 $ und $ z^2 = 1 $ gibt es die Lösungen $ z = \pm 1 $
	\begin{equation*}
	\xi_2 0 \cos \left(\frac{2 \pi}{2}\right) + i \sin \left(\frac{2 \pi}{2}\right) = (-1)
	\end{equation*}
	$ \xi_2^2 = 1 $
	
	%T1
	
	\item Verallgemeinerung $ z^n = a \Rightarrow $,,$ \sqrt[n]{a} $``	
	\begin{align*}
	z_k &= \sqrt[n]{|a|} \left(\cos \left(\frac{\alpha}{n}\right) + i \sin \left(\frac{\alpha}{n}\right)\right) \xi_k^k \qquad \qquad \alpha = \tx{arg}(a), k = 0, \dots, n -1\\
	z_k^n &= \left(\sqrt[n]{|a|}\right)^n \left(\cos \left(\frac{\alpha}{n} \cdot n\right) + i \sin \left(\frac{\alpha}{n} \cdot n\right)\right) \left(\xi_{n^{k}}\right)^{n}\\
	&= | a |\left( \cos \alpha + i \sin \alpha \right)\cdot 1 = a
	\end{align*}
\end{enumerate}

\subsection*{Nützliche begriffe}

Kreisscheiben: $ K_R(z_0) = \left\{ z_0 \in \mathbb{C}, R \in \mathbb{R}^+ : | z - z_0 | < R \right\} $\\
Kreislinie: $ C_{R}(z_0) = \left\{ z_0 \in \mathbb{C}, R \in \mathbb{R}^+ : | z - z_0 | = R \right\} $

%T2 

\section{Folgen und Reihen}

Motivation: $ \sum_{n=0}^{\infty} a_n = 0 $ mit $ a_n \in \mathbb{C} $\\[5pt]
Partialsummen, Folge der Partialsummen $ (s_0, s_1, \dots, s_m) $
\begin{equation*}
s_m = \sum_{n=0}^{m} a_n
\end{equation*}
Konvergenz der unendlichen Reihe $ \Leftrightarrow $ Konvergenz der Folge ihrer Partialsummen $ (s_n)_{n = 1, \infty} $

\subsection*{Def: Folge}

Eine Folge ist eine geordnete Menge von Zahlen $ (a_1, a_2, \dots) $, die Zuordnung von $ \mathbb{N} \to \mathbb{R} $ oder $ \mathbb{N} \to \mathbb{C} $ darstellt.

\subsection*{Def: Konvergenz einer Folge}

Eine Folge Konvergiert gegen den Grenzwert $ a $, wenn es zu jedem $ \epsilon > 0 $, $ \epsilon \in \mathbb{R} $ einen Index $ N(\epsilon) $ gibt, so dass $ \forall n \ge N(\epsilon), n \in \mathbb{N} $  gilt $ | a_{n} - a | < \epsilon $

\subsection*{Satz: Konvergenzkriterium von Cauchy}

Eine Folge $ (a_n) $ ist genau dann konvergent, wenn es zu jedem $ \epsilon > 0 $ ein $ N(\epsilon) $ gibt, so dass $ |a_n - a_m| \le \epsilon $ für $ n,m \ge N(\epsilon) $

\subsection*{Satz: Rechenregeln zu Limite}

$ (z_n), (w_n) $ seinen Folgen, die in $ \mathbb{C} $ konvergieren, dann konvergieren auch die Folgen: $ (z_n + w_n) $, $ (z_n \cdot w_n) $ und $ (\nicefrac{z_n}{w_n}) $ hier muss $ w_n \neq 0 $ geordert werden.\\[10pt]
\noindent
Schreibweise:
\begin{equation*}
\lim_{n \to \infty} = z , \lim_{n \to \infty} w_n = w
\end{equation*}
\begin{align*}
\lim_{n \to \infty} (z_n + w_n) &= \lim_{n \to \infty} z_n + \lim_{n \to \infty} w_n = z + w \\
\lim_{n \to \infty} (z_n \cdot w_n) &= \left(\lim_{n \to \infty} z_n  \right)\cdot \left( \lim_{n \to \infty} \right) \\
\lim_{n \to \infty} \left(z_n / w_n\right) &= \left(\lim_{n \to \infty} z_n\right) \bigg/ \left(\lim_{n \to \infty} w_n\right)
\end{align*}

\subsection*{Def: unendliche Reihen}

Die Reihe $ \displaystyle \sum_{n=0}^{\infty} a_n $ mit $ a_n \in \mathbb{C} $ ist definiert als Grenzwert der Folge $ \displaystyle (s_n)_{n=0,\infty} $ der Partialsummen $ \displaystyle s_n = \sum_{k = 0}^{n} a_k $ \ .

\subsection*{Def: Absolute Konvergenz}

$ \Big( $$ \sum a_n \to \sum |a_n| \overset{?}{=} \tx{konvergenz ?} $ ($ |a_n| \in \mathbb{R}^+ $) $ \Big) $\\[10pt]
Eine Reihe $ \displaystyle \sum_{n = 0}^{\infty} a_n $ heißt absolut konvergent, falls $ \displaystyle \sum_{n = 0}^{\infty} |a_n| $ konvergiert.

\subsection*{Satz: Jede absolut konvergente Reihe konvergiert}

\subsubsection{Beweis:}

\begin{align*}
|s_m - s_n| &= \bigg| \sum_{\substack{k = n + 1 \\ n < m}}^{m} a_k \bigg| \le \sum_{k = n+1}^{m} |a_k|  \quad \leftarrow \tx{Differenz von Partialsummen von} \sum_{k = 0}^{\infty} |a_k| \\
&= | \hat{s}_m - \hat{s}_n | \quad \leftarrow \tx{Partialsummen von} \uparrow
\end{align*}
\begin{equation*}
\hat{s}_n = \sum_{k=0}^{n}|a_k|
\end{equation*}
absolute Konvergenz: $ \forall \epsilon > 0 \ \exists n_0 $ sodass $ | \hat{s}_m - \hat{s}_n| < \epsilon $ für $ m,n \ge n_0 $.\\[10pt]
$ \Rightarrow $ Konvergenz von $ \sum_{k=n+1}^{m} a_k $ \hfw

\subsubsection*{\emph{Bemerkung}}

Nicht jede konvergente Reihe ist absolut konvergent. Zum Beispiel:
\begin{equation*}
\sum_{n=1}^{\infty} \frac{1}{n} \quad \tx{divergiert}, \quad \sum_{n=1}^{\infty} \frac{(-1)^n}{n} \quad \tx{konvergiert}
\end{equation*}
\begin{equation*}
s_{2n} - s_m = \ub{\frac{1}{n + 1} + \frac{1}{n+2} + \dots + \frac{1}{2m}}_{m} > \ub{\frac{1}{2m} + \frac{1}{2m} + \dots + \frac{1}{2m}}_{m} = \frac{m}{2m} = \frac{1}{2}
\end{equation*}
$ \rightarrow $ Cauchy Kriterium nicht erfüllt.

\subsection*{Satz: Die Reihe \texorpdfstring{$ \displaystyle \sum_{n=0}^{\infty} a_n $}{summe} konvergiert (absolut)}

\begin{equation*}
\Leftrightarrow \quad \sum_{n=0}^{\infty} \Re(a_n) \quad \tx{und} \quad \sum_{n=0}^{\infty} \Im(a_n) \quad \tx{konvergieren (absolut)}
\end{equation*}
Dabei gilt:
\begin{equation*}
\sum_{n=0}^{\infty} a_n = \sum_{n=0}^{\infty} \Re(a_n) + i \sum_{n=0}^{\infty} \Im(a_n)
\end{equation*}

\subsubsection{Beweis:}

Konvergenz: Aussage über Folge $ s_n, \Re(s_n), \Im(s_n) $
\begin{equation*}
s_n = \sum_{k=0}^{n} a_k \qquad , \qquad \hat{s}_{n}^{\Re} = \sum_{k=0}^{n} \Re(a_k) = \Re\left(\sum_{k=0}^{n} a_k\right) = \Re(s_n)
\end{equation*}
außerdem gilt auch:
\begin{equation*}
\hat{s}_{n}^{\Im} = \Im(s_n)
\end{equation*}
\begin{enumerate}[a)]
	\item $ |s_n - s_m| < \epsilon $ für $ m,n \ge n_0 $
	\begin{equation*}
	\epsilon > | s_n - s_m | = | \Re(s_n - s_m) + i \Im(s_n - s_m) | \ge | \Re(s_m) - \Re(s_n) |
	\end{equation*}
	Hierbei sind $ \Re(s_m) $ Partialsummen von $ \displaystyle \sum_{n=0}^{\infty} \Re(a_n) $\\
	$ \Rightarrow $ Konvergenz analog für $ \displaystyle \sum \Im(a_n) $
	\item $ |s_n - s_m| = | \Re(s_n - s_m) + i \Im(s_n - s_m) | \le \ub{|\Re(s_n - s_m)|}_{< \nicefrac{\epsilon}{2}} + \ub{|\Im(s_n - s_m)|}_{< \nicefrac{\epsilon}{2}} < \epsilon $ für $ m,n > n_0 $
\end{enumerate}








%\bibliographystyle{plain}
%\bibliography{literature}
%\addcontentsline{toc}{section}{Literatur}

\end{document}
