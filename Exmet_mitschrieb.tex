% packages
\PassOptionsToPackage{dvipsnames}{xcolor}  % needed to get colors working in certain environments
\documentclass[titlepage,11pt,a4paper,ngerman]{report}
\usepackage[utf8]{inputenc}
\usepackage[T1]{fontenc}
\usepackage[german]{babel}
\usepackage{graphicx}
\usepackage{wrapfig}  % used for wrapping a figure alternative to using minipages
\usepackage{amsmath}
\usepackage{amsfonts}
\usepackage{amssymb}
\usepackage[hidelinks]{hyperref}  % removes coloring for links
\usepackage{cleveref}
\usepackage{tikz}
\usepackage{tikz-cd}
\usepackage{nicefrac}  % adds \nicefrac alternative to regular \frac
\usepackage{mathtools}
\usepackage{enumerate}
\usepackage{cancel}  % adds \cancel which adds strikethrough
\usepackage{tocloft}
\usepackage{tcolorbox}
\usepackage{bm}  % adds \bm to make symbols bold, replaces outdated \boldsymbol
\usepackage[shortlabels]{enumitem}
\usepackage{placeins}
\usepackage{booktabs}
\usepackage{wasysym}
\usepackage{capt-of}

\usepackage[margin=1in]{geometry}  % changes the margins on all pages
\usepackage{url}

%SI-unix
%\usepackage{array}
\usepackage[per=slash,
            decimalsymbol=comma,
			loctolang={DE:ngerman,UK:english},
			]{siunitx}	
\sisetup{locale = DE}

\usetikzlibrary{calc}
\usetikzlibrary{decorations.pathmorphing,patterns}
\usetikzlibrary{arrows}
\usetikzlibrary{decorations.pathreplacing}
%\usetikzlibrary{snakes}

% Andrez:
%\usepackage{epigraph}  % adds \epigraph used to add fancy quotes to the beginning of chapters
%\usepackage{fancyhdr}
%\setlength{\parskip}{1em}
%\setlength{\headheight}{35pt}
%\setlength\epigraphwidth{.8\textwidth}
% alt math font
%\usepackage{eulervm}  % switches to alternate math font, usually requires extra download


%Environments und Newcommands:


% general commands

% zu zeigen symbol
\newcommand{\zz}{\fontfamily{cmss} \selectfont{Z\kern-.61em\raise-0.7ex\hbox{Z}:}}
% build over
\newcommand{\bov}[2]{\buildrel{#2} \over{#1}}
% better looking := (defined as)
\newcommand*{\defeq}{\mathrel{\vcenter{\baselineskip0.5ex \lineskiplimit0pt \hbox{\scriptsize.}\hbox{\scriptsize.}}}=}
\newcommand*{\eqdef}{=\mathrel{\vcenter{\baselineskip0.5ex \lineskiplimit0pt \hbox{\scriptsize.}\hbox{\scriptsize.}}}}

% integral differential d
\newcommand{\dif}{\mathop{}\!\mathrm{d}}
\newcommand{\difi}[1]{\mathrm{d}#1\mathop{}\!}

\newcommand{\prt}[2]{\frac{\partial #1}{\partial #2}}  % used for partial derivatives, tip: input #1 can be left blank
\newcommand{\prd}[2]{\frac{\tx{d} #1}{\tx{d} #2}}  % used for absolute (standart) derivatives, tip: input #1 can be left blank

\newcommand{\dd}{\tx{d}}


%Mathe:
\newcommand{\verteq}{\rotatebox{90}{$\,=$}}  % used for commands below
\newcommand{\equalto}[2]{\underset{\scriptstyle\overset{\mkern4mu\verteq}{#2}}{#1}}  % adds equal to underneath
\newcommand{\equaltoup}[2]{\overset{\scriptstyle\underset{\mkern4mu\verteq}{#2}}{#1}}  % adds equal to above
\newcommand{\custo}[3]{\underset{\scriptstyle\overset{\mkern4mu\rotatebox{-90}{$\,#1$}}{#3}}{#2}}  % same as above but replaces equal sign with input #1
\newcommand{\custoup}[3]{\overset{\scriptstyle\underset{\mkern4mu\rotatebox{-90}{$\hspace{-3pt} #1$}}{#3}}{#2}}
\newcommand{\casess}[4]{\left\{ \begin{array}{ll} {#1} & {#2} \\ {#3} & {#4} \end{array} \right.}  % used to indicate to the reader that something is missing here


%Text:
\newcommand{\tx}[1]{\textrm{#1}}
\newcommand{\const}{\tx{const.}}

\newcommand{\ul}[1]{\underline{#1}}
\newcommand{\ol}[1]{\overline{#1}}
\newcommand{\ub}[1]{\underbrace{#1}}
\newcommand{\ob}[1]{\overbrace{#1}}

\newcommand{\hfw}{\color{RubineRed}\tx{ $\star$hier fehlt was$\star$ } \color{black}}  % used to indicate to the reader that something is missing here 
\newcommand{\hft}{\color{green!80!black}\tx{ $\star$hier fehlt eine Grafik$\star$ } \color{black}}  % used to indicate to the reader that a tikz picture is migging


%Spezielles:


%Theo:
\newcommand{\lag}{\mathcal{L}}  % used for Lagrange function
\newcommand{\ham}{\mathcal{H}}  % used for Hamiltonian function
\newcommand{\gre}{\mathcal{G}}  % used for Green's function
\newcommand{\ssf}{\mathcal{Y}}  % used for Spherical Surface Function (Kugelflächenfunktion)
\newcommand{\eofr}{\vec{E}(\vec{r})}
\newcommand{\pofr}{\Phi(\vec{r})}
\newcommand{\grr}{\mathcal G(\vec{r},\vec{r}')}
\newcommand{\vphi}{\varphi}
\newcommand{\vabla}{\vec{\nabla}}


%LA:
\newenvironment{bew}[1]{\subsection{Bew: #1}}{\hfill$\square$}
\newcommand{\Bew}[2]{\begin{bew}{#1}#2\end{bew}}
\newcommand{\enph}{F: V \to V \textrm{ Endomorphismus}}

\newcommand{\im}{\tx{im}}
\newcommand{\spa}{\tx{span}}
\newcommand{\adj}{\tx{adj}}
\newcommand{\grad}{\tx{grad}}
\newcommand{\ord}{\tx{ord}}

\newcommand{\basis}[3]{\{#1_{#2}, \dots, #1_{#3}\}}
\newcommand{\ska}[2]{\langle #1 , #2 \rangle}  % scalar product of input 1, and 2 can also be used for braket notation
\newcommand{\dmat}[3]{\begin{pmatrix} #1_{#2}&&\\ &\ddots& \\ && #1_{#3} \end{pmatrix}}


%Ex:
\newcommand{\kq}{\frac{1}{4\pi\epsilon_0}}  % writes out the whole constant k from electrostatic
\newcommand{\kqq}{\frac{\mu_0}{4\pi}}  % writes out the constant from magnetostatic
\newcommand{\uind}{U_{\tx{ind}}}
\newcommand{\folie}[1]{\color{gray}[Folie: #1]\color{black}}  % used to tell the reader that there was multimedia content during a lecture
\newcommand{\versuch}[1]{\color{red!50!black} \textbf{Versuch:} \color{black} \textbf{#1}\\ }  % used to tell the reader that there was a live experiment

\newcommand{\mau}{$\buildrel \mathcal{O} \over{\textbf{.}}$}  % the extreme Waldmann exclamation mark recreated in latex by Markus


% Lab commands:
\newcommand\mean{\begin{equation}
\frac{\sum_{i=1}^n x_i}{n}\label{mean}
\end{equation}}  % shortcut for the standard Mean function

\newcommand\meanstd{\begin{equation}
s_x=\sqrt{\frac{1}{n-1}\sum_{i=1}^n(x_i-\overline{x})^2}\label{meanstd}
\end{equation}}  % shortcut for the standard derivative mean function

\newcommand\prodquo{\begin{equation}\left\vert\frac{\Delta z}{z}\right\vert=\sqrt{\left(a\frac{\Delta x}{x}\right)^2+\left(b\frac{\Delta y}{y}\right)^2+\ldots}\textrm{ f\"ur }z=x^a\ y^b\ldots\end{equation}}

\newcommand\tfuncd{\begin{equation}
t=\frac{\vert x_n-y_n\vert}{\sqrt{x_s^2+y_s^2}}
\end{equation}}

\newcommand\tfunc{\begin{equation}
t=\frac{\vert x-y_0\vert}{u_x}
\end{equation}}


% ANDREZ
%\newcommand{\summ}[2]{\sum_{#1}^{#2}}
%\newcommand{\intt}[2]{\int_{#1}^{#2}}
\newcommand{\lcom}[1]{\color{MidnightBlue}#1\color{black}}  % used to indicate to the reader that this content was transcribed from the lecture
\newcommand{\bei}{\emph{Beispiel:}}
\newcommand{\bem}{\emph{Bemerkung:}}


% Boxen:

\tcbuselibrary{theorems}

% mahlt eine box nur um den text mit titel
\newtcbox{\fribox}[1]{nobeforeafter,colback=white,colframe=red!75!black,fonttitle=\bfseries,title=#1,sharp corners,tcbox raise base}

% mahlt eine große box um alles mit titel
\newcommand{\frbox}[2]{\begin{tcolorbox}[colback=white,colframe=red!75!black,fonttitle=\bfseries,title=#1]#2\end{tcolorbox}}

% mahlt eine box nur um den text
\newtcbox{\ribox}{nobeforeafter,colback=white,colframe=red!75!black,sharp corners,tcbox raise base}

% mahlt eine große box um alles was drinnen ist
\newcommand{\rbox}[1]{\begin{tcolorbox}[colback=white,colframe=red!75!black]#1\end{tcolorbox}}

% mahlt eine box um mathe innerhalb mathmode
\newcommand{\rmbox}[1]{\tcboxmath[colback=white,colframe=red!75!black]{#1}}

% super box (looks like regular boxed but wraps around anything)
\newenvironment{supbox}{\begin{tcolorbox}[colback=white,colframe=black,sharp corners,boxrule=.5pt]}{\end{tcolorbox}}

% the Big Black Box, can be used to separate examples or review material from the main text (used for Wiederholung)
\newcommand{\bbb}[2]{\begin{tcolorbox}[colback=white,colframe=black,fonttitle=\bfseries,title=#1,sharp corners,tcbox raise base]#2\end{tcolorbox}}

% array type box with title
\newenvironment{zebox}[1]{\begin{array}{|c|}
		\multicolumn{1}{l}{\tx{#1}} \\
		\hline
		\displaystyle
	}{\\ \hline
\end{array}}

% arrow list
\newlist{arrowlist}{itemize}{1}
\setlist[arrowlist]{label=$\Rightarrow$}


% optional:

\renewcommand{\vec}[1]{\bm{#1}}

% changed because of preferred looks
\renewcommand{\epsilon}{\varepsilon}
\renewcommand{\paragraph}[1]{\subsubsection{#1}}  % changed oddly behaving paragraphs to simple subsubsections which are not numbered nor in the toc

% nur in Theo benutzt !!!
% \renewcommand{\Phi}{\varPhi}

% evtl:
% \renewcommand{\boxed}{\rmbox}


% Tikz definitions:

\def\centerarc[#1](#2)(#3:#4:#5)% Syntax: [draw options] (center) (initial angle:final angle:radius)
{ \draw[#1] ($(#2)+({#5*cos(#3)},{#5*sin(#3)})$) arc (#3:#4:#5); }

\def\checkmark{\tikz\fill[scale=0.4](0,.35) -- (.25,0) -- (1,.7) -- (.25,.15) -- cycle;}  % checkmark used for proofs

\tikzset{
	annotated cuboid/.pic={
		\tikzset{%
			every edge quotes/.append style={midway, auto},
			/cuboid/.cd,
			#1
		}
		\draw [every edge/.append style={pic actions, densely dashed, opacity=.5}, pic actions]
		(0,0,0) coordinate (o) -- ++(-\cubescale*\cubex,0,0) coordinate (a) -- ++(0,-\cubescale*\cubey,0) coordinate (b) edge coordinate [pos=1] (g) ++(0,0,-\cubescale*\cubez)  -- ++(\cubescale*\cubex,0,0) coordinate (c) -- cycle
		(o) -- ++(0,0,-\cubescale*\cubez) coordinate (d) -- ++(0,-\cubescale*\cubey,0) coordinate (e) edge (g) -- (c) -- cycle
		(o) -- (a) -- ++(0,0,-\cubescale*\cubez) coordinate (f) edge (g) -- (d) -- cycle;
		\path [every edge/.append style={pic actions, |-|}]
		%(b) +(0,-5pt) coordinate (b1) edge ["\cubex \cubeunits"'] (b1 -| c)
		%(b) +(-5pt,0) coordinate (b2) edge ["\cubey \cubeunits"] (b2 |- a)
		%(c) +(3.5pt,-3.5pt) coordinate (c2) edge ["\cubez \cubeunits"'] ([xshift=3.5pt,yshift=-3.5pt]e)
		;
	},
	/cuboid/.search also={/tikz},
	/cuboid/.cd,
	width/.store in=\cubex,
	height/.store in=\cubey,
	depth/.store in=\cubez,
	units/.store in=\cubeunits,
	scale/.store in=\cubescale,
	width=10,
	height=10,
	depth=10,
	units=cm,
	scale=.1,
}


% other settings
\hbadness=99999  % removes unnecessary hbadness warnings

% the following are used to circumvent Roman numerals in the toc from running out of space
%\addtolength{\cftchapnumwidth}{10pt}
%\addtolength{\cftsecnumwidth}{10pt}
%\addtolength{\cftsubsecnumwidth}{10pt}
%\renewcommand{\thechapter}{\Roman{chapter}}  % just don't do this :)

% coloring
% for working at night
%\pagecolor{darkgray}
%\color{white}

\begin{document}

\title{
	{\Huge Experimentelle Methoden}\\[1em]
	{\Large Vorlesung von Prof. Dr. apl. Horst Fischer im Sommersemester 2019}}
\author{Markus Österle \hspace{5pt} Damian Lanzenstiel}
\date{ \today}
\maketitle
\tableofcontents

% Document

\setcounter{chapter}{-1}

\chapter{Einführung}

\section{Wichtige Infos}

\begin{description}
	\item[Vorlesung] Montag 14:15 - 15:45
	\item[Übungen] ILIAS
	\item[Kontakt] Horst Fischer Physikhochhaus Zi. 609\\
	\hfw (email usw. Folie 1)
\end{description}

\section{Programm der Vorlesung}

\begin{itemize}
	\item Grundlagen moderner Nachweissysteme
	\item Grundlagen der Statistik und Unsicherheitsbetrachtungen
	\item Grundlagen der Analogelektronik
\end{itemize}

\chapter{Wechselwirkung geladener Teilchen mit Materie}

Nachweis durch Wirkung des Teilchens auf die Materie

\begin{itemize}
	\item Ionisation, Szintillation
	\item \v{C}evenkov-, Übergangsstrahlung
	\item Rückstoß
\end{itemize}
$ \Rightarrow $ Teilcheneigenschaften verändert
\begin{itemize}
	\item Energieverlust
	\item Richtungsänderung
	\item Identitätsverlust
\end{itemize}

\section{Klassische Betrachtung der Rutherfordstreuung}

\begin{itemize}
	\item stimmt mit QM in niederster Ordnung überein\\[5pt]
	solange: ,,schwere Teilchen``\\
	$ v \gg v_{e \tx{ in Hülle}} $\\
	$ \Delta E \gg \tx{Bindungsenergie von } e^- $
\end{itemize}

%T1
\hft

Typisches Beispiel:
\begin{equation*}
\mu^+ + \tx{Atom} \to \mu^+ + (\tx{Atom} + e^-)
\end{equation*}
Coulomb-Kraft
\begin{equation*}
F_{\parallel}(x) = F_{\parallel}(-x)
\end{equation*}
\begin{equation*}
F_{\perp} = \frac{1}{4 \pi \epsilon_0} \frac{z \cdot e \cdot Z \cdot e}{r^2} \frac{b}{|\vec{r}|}
\end{equation*}
Impulsübertrag
\begin{equation*}
\Delta \rho_{T} = \int_{- \infty}^{\infty} F_{\perp} \dd f = \frac{e^2}{4 \pi \epsilon_0} \cdot \frac{2 Z \cdot z}{\beta c b}
\end{equation*}
$ \beta = \frac{v}{c} $
\lcom{Mehr zum Thema und die genaue Rechnung findet man im Lehrbuch von Jackson.}\\
\textbf{Energieübertrag}\\
\folie{Energieverlust: klassisch nach Bohr}
\begin{equation*}
\Delta E = \frac{\Delta \rho_{T}^2}{2 M} = \frac{e^4}{(4 \pi \epsilon_0)^2} \cdot \frac{Z^2 z^2}{M \beta^2 c^2 b^2} \propto \frac{1}{b^2}
\end{equation*}
bei Kohärenter Streuung
\begin{equation*}
\frac{\Delta E \tx{ Elektronenhülle}}{\Delta E \tx{ Kern}} = \frac{2 m_p}{m_e} \approx 4000
\end{equation*}
Hülle: $ M = Z \cdot m_e $\\
Kern: $ M = A \cdot m_p = 2 Z \cdot m_p $\\
\ribox{$ \Rightarrow $ Die Streuung am Kern ist vernachlässigbar}\\
Der gesamte (mittlere) Energieverlust ist dann:
\begin{align*}
\langle\dd E\rangle &= \int \Delta E \cdot \ub{2 \pi b \ \dd b}_{\mathclap{\tx{Volumenelement}}} \ \cdot Z \cdot \ub{\frac{\rho \cdot N_A}{A}}_{= n_e} \dd x
\end{align*}
\frbox{Bethe-Bloch Beziehung}{
\begin{align*}
\left\langle \frac{\dd E}{\dd x} \right\rangle &= D \cdot \ub{\frac{Z \cdot \rho}{A}}_{\mathclap{\tx{Medium}}} \ \cdot \ \ub{\left(\frac{z}{\beta}\right)^2}_{\mathclap{\tx{Projektil}}} \cdot \ub{\ln\left(\frac{b_{\tx{max}}}{b_{\tx{min}}}\right)}_{\frac{1}{2} \ln \left( \frac{2 m_e c^2 \gamma^2 \beta^2}{I} T_{\tx{max}}\right) } \\
&= D \cdot \ub{\frac{Z \cdot \rho}{A}}_{\mathclap{\tx{Medium}}} \ \cdot \ \ub{\left(\frac{z}{\beta}\right)^2}_{\mathclap{\tx{Projektil}}} \cdot \ \frac{1}{2} \ln \left( \frac{2 m_e c^2 \gamma^2 \beta^2}{I} T_{\tx{max}} \right)
\end{align*}}
\noindent
mit $ I = \hbar \omega $: Ionisationspotential des Streuzentrums\\
und $ T_{\tx{max}} $: der Energie des $ e^- $ tragen kann\\[5pt]
\folie{Energieverlust}\\
\folie{Mittlerer Energieverlust nach Bethe Bloch}\\
\folie{Relativistischer Anstieg}\\
\folie{Materialabhängigkeit des mittleren Energieverlusts}\\
\folie{Minimaler Energieverlust}\\
\folie{Abhängigkeit vom Ionisationspotential}\\
\folie{Reichweite von Teilchen in Materie}\\
\folie{Bragg-Kurve} (Einstrahl-Tiefe in einen Menschen)\\
\folie{Anwendung Teilchenidentifizierung}\\
\folie{Energieverlust von Teilchen durch Ionisation}

% 6.05.19

\section{Energieverlust von Elektronen \texorpdfstring{$ e^- $}{e-} und Positronen \texorpdfstring{$ e^+ $}{e+}}

Bremsstrahlung führt zu zusätzlichem Energieverlust.
\begin{equation*}
E_{K} \approx \frac{600 \dots 700}{Z} \, \tx{MeV} \qquad \textbf{kritische Energie}
\end{equation*}
$ Z $ des Materials. Unterschiede zwischen fest, flüssig, gasförmig.
\begin{center}
	\begin{tabular}{l|c}
		Material & $ E_K $ \\
		\hline
		Luft: & 84,0 MeV\\
		Pb: & 7,4 MeV
	\end{tabular}
\end{center}
\begin{equation*}
\prd{E}{x} \bigg|_{\tx{Brems}} \propto \frac{Z^2}{m^2} \ \ \begin{array}{l} \tx{Target} \\ \tx{Projektil} \end{array}
\end{equation*}
Bremsstrahlung wichtig für $ e^{\pm} $
\begin{equation*}
\frac{m_\mu^2}{m_e^2} \left(\frac{100}{0{,}5}\right)^2 = 40000
\end{equation*}
(Eigentlich 105 statt 100)\\[5pt]
Bremsstrahlung führt zu
\begin{align*}
\prd{E}{x} &= E_e \cdot 4 \alpha r_e^2 N_A \frac{\rho Z}{A} \left\{ \ln \frac{183}{Z^{\nicefrac{1}{3}}} + \frac{1}{18} - f(z) \right\}\\
f(z) &= \alpha Z \left\{  \frac{1}{1 + \alpha^2 Z^2} + 0.2 + \mathcal{O}(\alpha Z^2)\right\}
\end{align*}
$ \alpha $: gemessene Konstante $ \alpha = 5{,}3 $ für H $ \big| 3 $ Pb

\subsection{Strahlungslänge}

\begin{equation*}
\frac{1}{L_{\tx{rad}}} = 4 \alpha r_e^2 N_A \frac{\rho Z}{A} \left\{ \ln \frac{183}{Z^{\nicefrac{1}{3}}} + \frac{1}{18} - f(z) \right\} = \prd{E}{x} \cdot \frac{1}{E_e}
\end{equation*}
(Die Formale stammt von Bether Heitler).\\
Die Strahlungslänge ist die Distanz, in der die $ e^{\pm} $ den Bruchteil $ (1 - \nicefrac{1}{e}) $ der Energie durch Bremsstrahlung verlieren.
\begin{equation*}
\prd{E}{x} \bigg|_{\tx{Brems}} = \frac{E_e}{L_{\tx{rad}}}
\end{equation*}

\chapter{Wechselwirkungen von Quanten / Photonen}

\section{Photoeffekt}

Photoeffekt = Absorption eines Protons ist gebunden an Hüllenelektron
\begin{equation*}
\gamma e^- A \to e^- A^+
\end{equation*}


%T1
\hft


\noindent
Wichtig $ E_{\gamma} \overset{<}{\approx} E_{\tx{bindung}} \approx \mathcal{O}(100 \, \tx{keV}) $. 10\% der WW an $ e^- $ der inneren Schalen.
\begin{equation*}
\sigma_{\tx{tot}} \propto Z^5 \cdot \left(\frac{m_e c}{E_{\gamma}}\right)^{- \nicefrac{7}{2}}
\end{equation*}
\textbf{Wichtig:} $ \sigma_{\tx{Photoeffekt}} $ ist pro Atom

\section{Compton Streuung}

\folie{Wechselwirkung von Photonen mit Materie}\\[5pt]
Streuung an quasi-freien $ e^- $:

%T2
\hft


\noindent
Energie \& Impulserhaltung
\begin{equation*}
E_{\gamma'} = \frac{E_{\gamma}}{1 + \frac{E_{\gamma}}{m_e c^2} (1 - \cos \theta)}
\end{equation*}
\begin{equation*}
\lambda_{\gamma '}' = \lambda_{\gamma} + \lambda_C (1 - \cos \theta)
\end{equation*}
\frbox{Compton Wellenlänge}{
\begin{equation*}
\lambda_C \le \frac{\hbar}{m_e c^2} = \frac{r_e}{\alpha_{\tx{em}}} = 39 \cdot 10^{-13 \, \tx{m}}
\end{equation*}}
\textbf{Wichtig:}
\begin{equation*}
E_{\gamma'}^{\tx{max}} (\theta = 0) = E_{\gamma}
\end{equation*}
\begin{equation*}
E_{\gamma'}^{\tx{min}} = \frac{E_{\gamma}}{1 + 2 \frac{E_{\gamma}}{m_e c^2}}
\end{equation*}
Wegen der Impulserhaltung gilt:
\begin{equation*}
\theta_{e}^{\tx{max}} \le \frac{\pi}{2}
\end{equation*}
Wirkungsquerschnitt (aus der Quantenelektrodynamik (QED))
\begin{equation*}
\sigma_{\tx{Compton}} \propto \frac{1}{E_{\gamma}} \cdot \ln \frac{2 E_{\gamma}}{m_e c^2}
\end{equation*}
Erzeugung hochenergetischer Photonen durch inverse Compton-Streuung.

\section{Paarbildung}

Paarbildung ist nur möglich in der Nähe eines Kerns (wegen Energie- und Impulserhaltung).

\subsection{Schwellen}

\begin{equation*}
E_{\gamma} > 2 m_e \approx 1{,}02 \, \tx{MeV} \quad \tx{im Kernfeld}
\end{equation*}
\begin{equation*}
E_{\gamma} > 4 m_e \approx 2{,}04 \, \tx{MeV} \quad \tx{im Elektronenfeld}
\end{equation*}
\begin{equation*}
\gamma + A \to e^- e^+ (A)
\end{equation*}


%T3
\hft


\begin{equation*}
\gamma + e^- \to e^- e^+ e^-
\end{equation*}
(Indent-Reaktion)


%T4
\hft


\begin{equation*}
\sigma_{\tx{Paar}} \propto \ln 183 Z^{-\nicefrac{1}{3}} \propto \frac{1}{L_{\tx{rad}}}
\end{equation*}
Insgesamt erhalten wir also für den Photoeffekt, die Compton-Streuung und die Paarbildung zusammen:
\begin{equation*}
\sigma_{\tx{tot}} \propto \sigma_{\tx{Photo}} + \sigma_{\tx{Compton}} + \sigma_{\tx{Paar}}
\end{equation*}
\begin{equation*}
\sigma_{\gamma} \propto c_1 Z^5 E^{\nicefrac{7}{2}} + c_2 Z \frac{1}{E} \ln E + c_3 Z^2
\end{equation*}
Minimum bei $ \mathcal{O} (10 \, \tx{MeV}) $\\
$ \Rightarrow $ große Reichweite !


\chapter{Detektoren für die Orts- und Zeitmessung}

Programm Heute:
\begin{itemize}
	\item Ionisationsdetektoren
	\item Szintillation
	\item Photomultiplier (PM)
\end{itemize}
,,Rekation`` im Material auf elektrische geladenes Teilchen oder Quanten
\begin{equation*}
\tx{Ionisation durch Projektil} \begin{array}{c}
\nearrow \\ \searrow
\end{array} \begin{array}{c c l}
\tx{freie Ladungsträger} & \rightarrow & \tx{Nachweis } e^- \tx{ / Ionen} \\[10pt]
\tx{Szintillation / Fluoreszens} & \rightarrow & \tx{Nachweis Licht}
\end{array}
\end{equation*}

\section{Ionisationsdetektoren}

\begin{itemize}
	\item mit flüssigem oder gasförmigen Edelgas $ + $ Beimischungen als ,,Quentscher``
	\item Halbleiter
\end{itemize}
\begin{equation*}
\tx{beide messen} \begin{array}{c}
\nearrow \\[5pt] \searrow
\end{array} \begin{array}{c}
\begin{array}{c}
\tx{nur primär erzeugte } e^- \\ \tx{(Halbleiter, Ionisationskammer)}
\end{array} \\[20pt] \begin{array}{c}
\tx{primäre $ e^- $ $ + $ Influenz von driftenden Ionen} \\ \tx{(Prop Zähler)}
\end{array}
\end{array}
\end{equation*}



%T1
\hft Strohalmdetektor


\noindent
Elektrisches Feld aus statischen Maxwellgleichungen:
\begin{align*}
\vabla \cdot \vec{E} &= \frac{1}{\epsilon_0} \rho \\
\int_{S} \vec{E} \dd \vec{S} &= \frac{1}{\epsilon_0} \int_{V} \rho \dd V
\end{align*}
auf Drahtlänge $ \Delta z $ befindet sich die Ladung $ \Delta Q $
\begin{equation*}
E(r) \cdot 2 \pi r \Delta z = \frac{1}{\epsilon_0} \Delta Q
\end{equation*}
\begin{equation*}
\Rightarrow \quad E(r) = \frac{1}{2 \pi \epsilon_0} \frac{1}{r} \prd{Q}{z}
\end{equation*}
Das $ \vec{E} $-Feld wird durch angelegte Spannung erzeugt. Aus der Abbildung %ref T1
oben folgt dann:
\begin{equation*}
\int_{r_i}^{r_a} E(r) \dd r = U = \frac{1}{2 \pi \epsilon_0} \cdot \ln\left(\frac{r_a}{r_i}\right) \cdot \prd{Q}{z}
\end{equation*}
\begin{equation*}
E(r_0) = \frac{U}{r_0 \ln \left(\frac{r_a}{r_i}\right)}
\end{equation*}
\folie{Arbeitsbereiche von Gasionisationsdetektoren}\\
\folie{Funktionsprinzip Gasionisationsdetektoren}\\
Ionisationszähler $ \rightarrow $Dosimetrie / Dosimeter\\
Proportionalitätsbereich $ \rightarrow $ Teilchennachweis (Ort und Zeit)\\
Geiger Müller Zähler ist selbst verstärkend $ \rightarrow $ keine extra Geräte notwendig\\[5pt]
Elektronen \& Ionen, die im Abstand $ r_0 $ erzeugt werden driften zur Anode/Kathode.\\
z:.B.
\begin{equation*}
\Delta t^- = \int_{r_i}^{r_a} \frac{\dd r}{\theta_0^-} = \int_{r_i}^{r_a} \frac{\dd r}{\mu^- \cdot E} = \frac{\ln \left(\frac{r_a}{r_i}\right)}{2 \mu^- E}
\end{equation*}
\folie{Eigenschaften von Edelgasen}\\
\folie{Vieldrahtproportional- und Driftkammern}\\
\folie{MICROMEGAS}\\
\folie{GEM / THGEM}

\section{Halbleiterzähler}

Kristallines Si \& Ge. Ideoal für $ \prd{E}{x} $ hochauflösende Ortsmessung\\[5pt]

\subsection{Funktionsprinzip:}

\begin{itemize}
	\item Diode in Sperrrichtung
	\item ionisierende Strahlung erzeugt $ e^- $/Loch Paare
	\item äußere Betriebsspannung saugt $ e^- $/Löcher ab
\end{itemize}
\textbf{Vorteile:}
\begin{enumerate}[a)]
	\item $ \langle E \rangle $ zur Erzeugung eines $ e^- $/Loch Paars $ \langle E \rangle_{\tx{Si}} = 3{,}6 \, \tx{eV} $ und $ \langle E \rangle_{\tx{Ge}} = 2{,}8 \, \tx{eV} $.\par
	Zum Vergleich $ \langle E \rangle _{\tx{Gas}} \approx 10{.}40 \, \tx{eV} $ und $ \langle E \rangle_{\tx{Szint}} = 100 \, \tx{eV} \tx{-} 1 \, \tx{keV} $
	\item hohe spezifische Dichte $ \Rightarrow \prd{E}{x} $ groß
	\item sehr schnelle \hfw
	\item kompakte \hfw
\end{enumerate}

\subsection{Grundlagen}

Festkörper:


%T2
\hft



\noindent
Direkte Rekombination $ \mathcal{O}(s) $ weil $ e^- $ \& Loch Energie- und Impulserhaltung.\\
\folie{Funktionsprinzip (Halbleiter)}\\
\folie{Funktionsprinzip: Streifenzähler}\\
\folie{Ultrasonic Bonding}\\
\folie{ATLAS Silizium Spurdetektor}\\
\folie{Silizium Detektoren als Spur Detektor (CMS: Currently the Most Silicon)}\\
\folie{Halbleiter-Pixelzähler}\\
\folie{Zukunft: 3D-Technologie}

\section{Szintillationsdetektor}

$ \prd{E}{x} \to $ Anregung der Atome/Moleküle $ \to $ Lichtemission $ \propto \prd{E}{x} $.\\[5pt]
Wichtig dabei ist die Transparenz des Detektors für das erzeugte Licht. Vorteilhaft ist deswegen, wenn die Spektralemission im sichtbaren Bereich ist.\\[5pt]
Typen:
\begin{itemize}
	\item organische Kristalle, Flüssigkeiten oder Plastik
	\item anorganische Kristalle
	\item flüssige, gasförmige Edelgase
\end{itemize}

\subsection{Funktionsprinzip}

\begin{enumerate}[a)]
	\item Anorganisch
	\begin{itemize}
		\item Dotieren mit Farbzentren (Aktivatorzentren) (Leerstellen im Gitter)
		\item Ionisation führt zu freien $ e^- $
		\item $ \Rightarrow $ Rekombination in Aktivatorzentren $ \to $ Anregung selbige $ \to $ Übergang in Grundzustand unter \hfw $ \left. \right\} \mathcal{O}(\mu s) $
	\end{itemize}
	\item Organisch
	\begin{itemize}
		\item Ionisation \& Anregung von Molekülen
		\item $ \to $ emittiert beim Zerfall UV- Licht + Wellenlängenverschiebung $ \Rightarrow $ sichtbares Licht
	\end{itemize}
\end{enumerate}
\folie{Szintillatoren}\\
\folie{Einsatzprinzip}\\
\folie{Emission}\\
\folie{Organische Szintillatoren - Licht Absorption}

\section{Photomultiplier}

\folie{Photomultiplier}\\
\folie{Quanteneffizienz}\\
\folie{PMT und Szintillator Handhabung}


%20.05.19

\chapter{Teilchenidentifikation}

\subsubsection*{Programm Heute}

\begin{itemize}
	\item Cherenkov Strahlung
	\item Übergangsstrahlung
	\item Energiemessung
	\item Elektromagnetische Kalorimeter
	\item Hadronische Kalorimeter
\end{itemize}

\section{\texorpdfstring{\v{C}}{C}herenkov Strahlung}


%T1
%T2
\hft


\noindent
\v{C}herenkov Strahlung wird emittert wenn $ v > v_{\tx{Phase}} $ (Medium) also bei $ \beta > \frac{1}{n_{\tx{Medium}}} $.\\[5pt]
Der Abstrahlwinkel ist:
\begin{equation*}
\cos \theta = \frac{1}{\beta \cdot n} = \frac{v_{\tx{Phase} (\tx{Medium}) }}{v_{\tx{Teilchen}}} \Rightarrow \frac{c}{n} \cdot \frac{1}{v_{\tx{Teilchen}}}
\end{equation*}
mit $ n $ dem Brechungsindex im Medium.\\[5pt]
Der maximale Abstrahlwinkel $ \varangle $:
\begin{equation*}
\Theta^{\tx{max}} = \arccos \frac{1}{n}
\end{equation*}
Schwellenenergie $ E_s $ ab der \v{C}herenkov Strahlung auftritt
\begin{equation*}
\gamma_{s} = \frac{E_{s}}{\color{red} m \color{black} c^2} = \frac{1}{\sqrt{1 - \beta_{s}^{2}}} = \frac{1}{\sqrt{1 - \frac{1}{\color{red} n^2 \color{black}}}}
\end{equation*}
\folie{Leuchten eines Kernreaktors}\\
\folie{\v{C}herenkov Effekt}\\[5pt]
Anzahl emittierter Photonen
\begin{equation*}
\prd{\mu}{\lambda} = 2 \pi \alpha Z^2 \left(1 - \frac{1}{(p_n)^2}\right) \frac{\dd \lambda}{\lambda^2} \approx 2 \pi \alpha Z^2 \frac{\lambda_2 - \lambda_1}{\lambda_1 \lambda_2} \sin^2 \theta
\end{equation*}
$ n = n(\lambda) $\\
für $ [400 \, \tx{nm}, 700 \, \tx{nm}] $ $ \frac{\dd N}{\dd x} \approx \hfw $
\begin{equation*}
\left(\prd{E}{x}\right)_{\tx{\v{C}}} \approx 10^{-2} \left(\prd{E}{x'}\right)_{\tx{Ionis}}
\end{equation*}
als Radiator: alle transparente Stoffe: NaCl, Diamant, Bleiglas, H$ _{2} $O.\\[5pt]
\folie{\v{C}renenkov Winkel vs. Teilchengeschwinkdigkeit}\\
\folie{Photonenausbeute}\\
\folie{Verschiedene Typen}

\section{Übergangsstrahlung}

%T3
\hft


\noindent
Teilchen + Spiegelladung $ \Rightarrow $ veränderlicher Dipol $ \Leftrightarrow $ Übergangsstrahlung


\hfw

\noindent
Abstrahlungscharakteristik
\begin{equation*}
\prd{E}{\omega \dd \Omega} = \frac{h \alpha}{\pi^2} \beta^2 \cdot f(\theta)
\end{equation*}
$ \omega : $ Plasmafrequenz
\begin{enumerate}[a)]
	\item nicht relativistischer Fall $ f(\theta) = \sin^2 \theta $
	\item relativistischer Fall $ f(\theta) = \frac{\sin^2 \theta }{1 - \beta \cos ^2 \theta} $
	\item $ v = \frac{E}{m} \gg 1000 $
	u.s. Bereich der Röntgenstrahlung
\end{enumerate}

Polarisationsebene definiert durch
\begin{itemize}
	\item bewegte Ladung
	\item Abstrahlrichtung des Photonen
\end{itemize}
Bsp $ e^- $ mit $ E = 15 \, \tx{GeV} $, $ \gamma_e = 30000 $, $ \gamma_{\pi} = 110 $\\[5pt]
Wahrscheinlichster Abstrahlwinkel:
\begin{equation*}
\theta \approx \sqrt{\frac{1}{\gamma^2} + \frac{\omega_p}{\omega}} \approx \frac{1}{\gamma}
\end{equation*}
$ \Rightarrow $ \textbf{Kein Grenzwinkel!}
verstärkung des Effekts durch mehrfache Übergänge zwischen Medien.\\[5pt]
\folie{Winkelverteilung Übergangsstrahlungsstrahlung}\\
\folie{Übergangsstrahlungsdetektoren}

\section{Kalorimeter}

Aufgabe: Messung der Gesamtenergie in Abhängigkeit von der Bauweise
\begin{itemize}
	\item homogene Schauerzähler/Kalorimeter
	\item sampling Schauerzähler/Kalorimeter (Stichprobenmessung) $ \to $ Einfluss auf die Auflösung bei der Energiemessung
\end{itemize}
\begin{enumerate}[a)]
	\item Elektron-Photon Schauer\\
	Einfache Abschätzung
	\begin{equation*}
	N_{e^{\pm}, \gamma} \approx 2^t \qquad E(t) \approx \frac{E_0}{2^t}
	\end{equation*}
	$ t : $ konstante Zeit / Eindringtiefe Schnitte\\
	$ x_0 = \frac{1}{C_{\tx{RAD}}} $ Strahlungslänge
	\begin{itemize}
		\item Elektronen: $ 1 - \frac{1}{e} \approx 63\% $ der Energie wird durch Abstrahlung in Photonen abgegeben
		\item Photonen $ 1 - \frac{1}{e^{7/9}} \approx 54\% $ Intensität geht durch $ e^+, e^- $ Paarbildung verloren
	\end{itemize}
	$ t_{\tx{max}} = \frac{\ln\nicefrac{E_0}{E_{\tx{Krit}}}}{\ln2} \approx 10{,}5 X_0 $\\
	$ N_{\tx{max}} = \frac{E_0}{E_{\tx{Krit}}} \approx 1400 $ für $ Z = 82 $\\
	genau auf $ O(x3, \dots, x5) $ genauer mit EGS GEANT
	\begin{equation*}
	\prd{E}{t} \propto t^a e^{-bt}
	\end{equation*}
	mit $ t = \frac{x}{x_0} $ (Anzahl Strahlungslängen)
	
	\hfw
	
	
	\folie{Bremsstrahlung (Bethe-Heitler)}\\
	\folie{Naives Schauerbild}\\
	\folie{Longitudinal und Transverse Schauer Profile}\\
	\folie{Longitudinale Schauerentwicklung}
\end{enumerate}

\chapter{Statistik und Wahrscheinlichkeiten}

\subsubsection{Literatur}

\begin{itemize}
	\item S.Brandt ,,Datenanalyse``
	\item G.Cowan ,,Statistical Data Analysis``
	\item R.Barlow ,,A Guide to the Use os Statistical Methods in Physical Sciences``
	\item F.James ,,Statistical and Computational Methods in Experimental Physics``
\end{itemize}
\folie{Einführung: Compass Experiment}\\
\folie{Einführung in die Statistik}

\section{Einführung}

2 mögliche Ansätze
\begin{enumerate}[a)]
	\item \textbf{Frequentist (Zählmensch) Axiome:}\\[10pt]
	Ereignismenge:
	\begin{equation*}
	E \defeq \left\{ \dots, A, B \right\}
	\end{equation*}
	\begin{enumerate}[1)]
		\item \begin{equation*}
		P(A) \ge 0 \quad \forall A \in E
		\end{equation*}
		\item \begin{equation*}
		\sum_{A \in E} P(A) = 1 \quad \Rightarrow \tx{ d.h.  \quad P(E) = 1}
		\end{equation*}
	\end{enumerate}
	wenn $ A_i $ tatsächlich Ereignisse sind, dann schließen sich $ A $ und $ B $ gegenseitig aus
	\begin{enumerate}[3)]
		\item \begin{equation*}
		P(A \land B) = P(A) + P(B)
		\end{equation*}
	\end{enumerate}
	nur gültig für den Fall $ A,B $ exklusiv
	\begin{equation*}
	\begin{aligned}
	\lor : \tx{oder} & \qquad \neg : \tx{nicht} \\
	\land : \tx{und} & \qquad \setminus : \tx{ohne}
	\end{aligned}
	\end{equation*}
	\begin{equation*}
	P(A \land \ol{A}) = P(A) + P(\ol{A}) = 1
	\end{equation*}
	\begin{equation*}
	\Rightarrow \quad 0 \le P(A) \le 1
	\end{equation*}
	\begin{figure}[ht]
		\centering
		%t1
		%\hft zwei mengen und schnittmenge
		\begin{tikzpicture}[scale=0.75]
			\draw[green!80!black] (-1,-.5) ellipse (2cm and 1cm);
			\node[green!80!black] at (-1,-.5) {$ A $};
			\draw[red] (1,.5) ellipse (2cm and 1cm);
			\node[red] at (1,.5) {$ B $};
			\draw[rotate=16] (0,0) ellipse (3.3cm and 1.5cm);
			\node at (3,2) {$ E \Rightarrow P(E) = 1 $};
		\end{tikzpicture}
		\caption{Gesamtmenge $ E $ aller Ereignisse und zwei Ereinisse $ A, B \in E $.}
		\label{Mengen}
	\end{figure}
	\begin{align*}
	P(A + B + \dots) &= P(A) + P(B) + \dots \\
	P(\ol{A}) &= P(E \setminus A) = 1 - P(A) \\
	P(A \lor B) &= P(A) + P(B) - P(A \land B) \\
	P(A \land B) &= P(A) + P(B) - P(A \lor B) \\
	Ü(A \land B) + P(A \lor B) &= P(A) + P(B) \\
	\end{align*}
	\item \textbf{Bayes Statistik (bedingte Wahrscheinlichkeit)}\\[10pt]
	\begin{equation*}
	P(B|A) = \frac{P(A \land B)}{P(A)}
	\end{equation*}
	Allgemein:
	\begin{equation*}
	P(A|B) = \frac{P(B|A) \cdot P(A)}{\sum_i P(B|A_i) \cdot P(A_i)}
	\end{equation*}
	Beispielrechnung:\\
	Es gibt eine Krankheit und $ 0{,}1 \% $ der Bevölkerung sind erkrankt (Durchsendung). Es gibt einen Test um die Krankheit festzustellen mit einer $ 98\% $ Effizienz ($ \widehat{=} $ Gewissheit) und $ 3\% $ Fehlalarm ($ \widehat{=} $ Reinheit)\\[5pt]
	Frage: Was ist die Wahrscheinlichkeit erkrankt zu sein bei einem positiven Testergebnis (Befund) $ P(\tx{krank}|+) $
	\begin{align*}
	P(\tx{krank}) &= 0{,}001 \\
	P(\tx{gesund}) &= 0{,}999 \\
	P(+|\tx{krank}) &= 0{,}98 \\
	P(-|\tx{krank}) &= 0{,}02 \\
	\tx{Fehlalarm:} \quad P(+|\tx{gesund}) &= 0{,}03 \\
	P(-|\tx{gesund}) &= 0{,}97 \\
	\end{align*}
	\begin{align*}
	P(\tx{krank}|+) &= \frac{P(+|\tx{krang}) \cdot P(\tx{krank})}{P(+|\tx{krank}) \cdot P(\tx{krang}) + P(+|\tx{gesund}) \cdot P(\tx{gesund})} \\
	&= \frac{0{,}98 \cdot 0{,}001}{0{,}98 \cdot 0{,}001 + 0{,}03 \cdot 0{,}0999} = 0{,}032
	\end{align*}
	Die Wahrscheinlichkeit, bei einem positiven Testergebnis, krank zu sein ist also nur $ 3{,}2 \% $!!!
\end{enumerate}

\section{Verteilung einer Zufallsvariable}

\begin{description}
	\item[Population] Alle möglichen Eregnisse/Messungen
	\item[Stichprobenraum] Untermenge ausgewählter Stichproben
	\item[Zufallsvariable] \begin{itemize}
		\item diskret
		\item kontinuierlich
	\end{itemize}
	Verteilung einer Zufallsvariable $ x $ mit $ -\infty \le x \le + \infty $
\end{description}

Wahrscheinlichkeitsverteilung Beispiel: Würfel  $ P = \frac{1}{6} $


%T2
\hft Wkeit.vert. Würfel


\begin{equation*}
F(x) = \sum_{i=1}^{\tx{max}(x)} x p_i
\end{equation*}
monoton. $ \tx{max}(x) = $ ist der größte Wert für x.\\[10pt]
Wahrscheinlichkeitsdichte von $ x $
\begin{equation*}
f(x) = \prd{F}{x} = F'(x)
\end{equation*}
ist das Maß für die Wahrscheinlichkeit $ X $ eines Ereignisses $ X $
\begin{equation*}
x \le X \le x + \dd x
\end{equation*}

%T3
\hft
%T4
\hft


\folie{Histogrammdarstellung}

\subsection{Diskussion der Verteilungsfunktion}

\begin{enumerate}[a)]
	\item falls die Wahrscheinlichkeitsdichte differenzierbar ist\\
	$ \Rightarrow $ \textbf{Wahrscheinlichster Wert}
	\begin{equation*}
	\prd{}{x} f(x) = 0 \qquad \prd{^2}{x^2} f(x) < 0
	\end{equation*}
	\item \textbf{Median} $ x_{0.5} $ (50\% der Werte kleiner, 50\% der WErte größer)\\
	oder auch: Verteilungsfunktion hat den Wert; $ \frac{1}{2} $
	\begin{equation*}
	F(0.5) = P(X < x_{0.5})
	\end{equation*}
	\begin{equation*}
	\Rightarrow \quad \int_{-\infty}^{x_{0.5}} f(x) \dd x \overset{!}{=} 0.5
	\end{equation*}
	analog geht man vor für die Momente der Verteilung:
	\begin{equation*}
	E(x^n) = \int_{-\infty}^{\infty} x^n f(x) \dd x
	\end{equation*}
	hängt ab von der Wahrscheinlichkeitsdichteverteilung.\\[10pt]
	\textbf{Mittelwert (erstes Moment)}\\
	$ n = 1 $
	\begin{equation*}
	\mu = E(x) = \int_{-\infty}^{\infty} x f(x) \dd x
	\end{equation*}
	\textbf{Streuung um Mittelwert}
	\begin{equation*}
	E(x-\mu)^n = \int_{-\infty}^{\infty} (x-\mu)^n f(x) \dd x
	\end{equation*}
	um den Wahren Wert $ \mu $.\\[5pt]
	\textbf{Varianz}\\
	$ n = 2 $
	\begin{equation*}
	E(x-\mu)^2 = E(x^2) - \mu^2 = \int_{-\infty}^{\infty} f(x) \dd x
	\end{equation*}
	bei $ n = 3 $: Schiefe
\end{enumerate}

%03.06.19

\setcounter{section}{10}

\section{Wichtige Verteilunge}

\subsection*{Motivation}

Experiment + Theorie $ \to $ PDF $ f(N,\bar{E},w) $.

\subsection{Binomialverteilung}

Versuch mit zwei möglichen Ereignissen $ A $ und $ \bar{A} $ Z.B. Münzwurf. Die Wahrscheinlichkeiten dafür sind jeweils $ P(A) = p $ und $ P\bar{A} = 1 - p = q $. Bei $ N $ Wiederholungen des Versuchs ist die Wahrscheinlichkeit für 
\begin{equation*}
X = (\ub{A,A,A, \dots , A }_{n}, \ub{\bar{A},\bar{A},\bar{A}, \dots, \bar{A}}_{N - n})
\end{equation*}
gleich
\begin{equation*}
P(X) = p^n \cdot q^{N-n}
\end{equation*}
Diese Ereignisse können an
\begin{equation*}
\begin{pmatrix}
N \\ n
\end{pmatrix} = \frac{N!}{n! (N-n)!}
\end{equation*}
auftreten.
\begin{equation*}
B = f(n,N,p) = \begin{pmatrix}
N \\ n
\end{pmatrix} p^n q^{N-n}
\end{equation*}
mit der Anzahl des Auftretens von $ A $ als Laufparameter $ n \in \mathbb{N}_{0} $.\\[10pt]

\subsubsection{Erwartungswert}

\begin{equation*}
E(B(N,p)) = \sum_{n=0}^{N} n \begin{pmatrix}
N \\ n
\end{pmatrix} p^n q^{N-n} = \dots = N \cdot p
\end{equation*}
Summenregel für Erwartungswerte
\begin{equation*}
E[X_1+ \dots + X_N] = E[X_1] + E[X_2] + \dots + E[X_N] = N E[X_1] = N \cdot p
\end{equation*}

\subsubsection{Varianz}

\begin{equation*}
V[n] = E[n^2] - E[n]^2 = N \cdot p (1 - p) = N p q
\end{equation*}
Beispiel 1: Münzwurf wird übersprungen.\\[5pt]
\textbf{Beispiel 2: 10 Exp Messungen einer Größe}\\[10pt]
Fehler mit $ p = 0{,}683 $ liegt der Wahre Wert in dem Intervall $ [x, \pm \sigma] $
\begin{equation*}
B(10,10,0{,}683) = 0{,}683^{10} \approx 0{,}02
\end{equation*}
Man würde also erwarten, dass die Fehler überschätzt wurden.

\subsection{Poisson Verteilung}

Beschreibt im Wesentlichen die Binomialverteilung im Grenzfall für $ N \to \infty $ und $ p \to 0 $. $ \Rightarrow \ N \cdot p = \nu $ endlich.
\begin{equation*}
B(n, N, p) \to f(n,\nu) = \frac{\nu^n}{n!} e^{-\nu}
\end{equation*}

\subsubsection{Erwartungswert}

\begin{equation*}
E[n] =\sum_{n=0}^{\infty} n \cdot \frac{\nu^n}{n!} \cdot e^{-\nu} = \sum_{n=1}^{\infty} n \cdot \frac{\nu^n}{n!} \cdot e^{-\nu} = \nu e^{-\nu} \ub{\sum_{n=1}^{\infty} \frac{\nu^{n-1}}{(n-1)!}}_{e^{\nu}} = \nu
\end{equation*}

\subsubsection{Varianz}

\begin{equation*}
V[n] = \sum_{n=0}^{\infty} (n-\nu)^2 \frac{\nu^n}{n!} e^{-\nu} = \nu
\end{equation*}
Standartabweichung $ \sigma = \sqrt{\nu} $\\[5pt]
\textbf{Beispiel 3: 50 Szintillatoren 1 Jahr Messen}\\[10pt]
\begin{equation*}
P(A) = 0{,}01 = 1\%
\end{equation*}
Wobei $ A \widehat{=} $ Szintillator kaputt nach $ \le 1 $ Jahr
\begin{equation*}
\tx{Binomial :} \quad B(0,50,0{,}01) = \begin{pmatrix}
50 \\ 0
\end{pmatrix} 0{,}01^{0} \cdot 0{,}93^{50} = 0{,}067
\end{equation*}
\begin{equation*}
\tx{Poisson :} \quad f(0,0{,}5) = e^{-0{,}5} = 0{,}607
\end{equation*}
$ \nu = N \cdot p = 0{,}5 $

\subsection{Gleichverteilung}

\begin{equation*}
f(x; \alpha, \beta) = \casess{\frac{1}{\beta - \alpha}}{\alpha \le x \le \beta}{0}{\tx{sonst}}
\end{equation*}
Erwartungswert und Varianz:
\begin{align*}
E[x] &= \frac{1}{2} (\alpha + \beta) \\
V[x] &= \frac{1}{12} (\beta - \alpha)^2
\end{align*}

\subsection{Gauß-Verteilung / Normalverteilung}

\emph{Definition} einer Wahrscheinlichkeitsdichte:
\begin{enumerate}[1.]
	\item $ f(x) \ge 0 \ \forall x $
	\item $ f(x) $ muss integrierbar sein.
	\item $ \int_{-\infty}^{\infty} f(x) \dd x = 1 $
\end{enumerate}
\begin{equation*}
f(x;\mu,\sigma^2) = \frac{1}{\sqrt{2 \pi \sigma^2}} e^{\frac{-(x-\mu)^2}{2 \sigma^2}}
\end{equation*}
2 Parameter:
\begin{itemize}
	\item $ \mu = $ Mittelwert
	\item $ \sigma =  $ Standartabweichung
\end{itemize}

\subsubsection{Erwartungswert}

\begin{equation*}
E[X] = \int_{-\infty}^{\infty} x \cdot f(x; \mu, \sigma^2) \dd x = \dots = \mu
\end{equation*}

\subsubsection{Varianz}

\begin{equation*}
V[X] = \int_{-\infty}^{\infty} (x - \mu)^2 \cdot f(x; \mu, \sigma^2) = \dots = \sigma^2
\end{equation*}

\begin{equation*}
\tx{Verteilungs Dichte (PDF)} \overset{\tx{Integration}}{\longrightarrow} \tx{Verteilungsfunktion (CDF)}
\end{equation*}
Oft betrachtet man die Standard Normalverteilung:
\begin{equation*}
\mu = 0 \qquad \sigma = 1
\end{equation*}
\begin{equation*}
x = \frac{x}{\sigma} - \mu
\end{equation*}
\begin{equation*}
\Phi(x) = \frac{1}{\sqrt{2 \pi}} \int_{-\infty}^{\infty} e^{\frac{1}{2} t^2} \dd t
\end{equation*}
Verteilungsfunktion der Standard Normalverteilung

%\bibliographystyle{plain}
%\bibliography{literature}
%\addcontentsline{toc}{section}{Literatur}

\end{document}
