% packages
\PassOptionsToPackage{dvipsnames}{xcolor}  % needed to get colors working in certain environments
\documentclass[titlepage,11pt,a4paper,ngerman]{report}
\usepackage[utf8]{inputenc}
\usepackage[T1]{fontenc}
\usepackage[german]{babel}
\usepackage{graphicx}
\usepackage{wrapfig}  % used for wrapping a figure alternative to using minipages
\usepackage{amsmath}
\usepackage{amsfonts}
\usepackage{amssymb}
\usepackage[hidelinks]{hyperref}  % removes coloring for links
\usepackage{cleveref}
\usepackage{tikz}
\usepackage{tikz-cd}
\usepackage{nicefrac}  % adds \nicefrac alternative to regular \frac
\usepackage{mathtools}
\usepackage{enumerate}
\usepackage{cancel}  % adds \cancel which adds strikethrough
\usepackage{tocloft}
\usepackage{tcolorbox}
\usepackage{bm}  % adds \bm to make symbols bold, replaces outdated \boldsymbol
\usepackage[shortlabels]{enumitem}
\usepackage{placeins}
\usepackage{booktabs}
\usepackage{wasysym}

\usepackage[margin=1in]{geometry}  % changes the margins on all pages
\usepackage{url}

%SI-unix
%\usepackage{array}
\usepackage[per=slash,
            decimalsymbol=comma,
			loctolang={DE:ngerman,UK:english},
			]{siunitx}	
\sisetup{locale = DE}

\usetikzlibrary{calc}
\usetikzlibrary{decorations.pathmorphing,patterns}
\usetikzlibrary{arrows}
\usetikzlibrary{decorations.pathreplacing}
%\usetikzlibrary{snakes}

% Andrez:
%\usepackage{epigraph}  % adds \epigraph used to add fancy quotes to the beginning of chapters
%\usepackage{fancyhdr}
%\setlength{\parskip}{1em}
%\setlength{\headheight}{35pt}
%\setlength\epigraphwidth{.8\textwidth}
% alt math font
%\usepackage{eulervm}  % switches to alternate math font, usually requires extra download


%Environments und Newcommands:


% general commands

% zu zeigen symbol
\newcommand{\zz}{\fontfamily{cmss} \selectfont{Z\kern-.61em\raise-0.7ex\hbox{Z}:}}
% build over
\newcommand{\bov}[2]{\buildrel{#2} \over{#1}}
% better looking := (defined as)
\newcommand*{\defeq}{\mathrel{\vcenter{\baselineskip0.5ex \lineskiplimit0pt \hbox{\scriptsize.}\hbox{\scriptsize.}}}=}
\newcommand*{\eqdef}{=\mathrel{\vcenter{\baselineskip0.5ex \lineskiplimit0pt \hbox{\scriptsize.}\hbox{\scriptsize.}}}}

% integral differential d
\newcommand{\dif}{\mathop{}\!\mathrm{d}}
\newcommand{\difi}[1]{\mathrm{d}#1\mathop{}\!}

\newcommand{\prt}[2]{\frac{\partial #1}{\partial #2}}  % used for partial derivatives, tip: input #1 can be left blank
\newcommand{\prd}[2]{\frac{\tx{d} #1}{\tx{d} #2}}  % used for absolute (standart) derivatives, tip: input #1 can be left blank

\newcommand{\dd}{\tx{d}}


%Mathe:
\newcommand{\verteq}{\rotatebox{90}{$\,=$}}  % used for commands below
\newcommand{\equalto}[2]{\underset{\scriptstyle\overset{\mkern4mu\verteq}{#2}}{#1}}  % adds equal to underneath
\newcommand{\equaltoup}[2]{\overset{\scriptstyle\underset{\mkern4mu\verteq}{#2}}{#1}}  % adds equal to above
\newcommand{\custo}[3]{\underset{\scriptstyle\overset{\mkern4mu\rotatebox{-90}{$\,#1$}}{#3}}{#2}}  % same as above but replaces equal sign with input #1
\newcommand{\custoup}[3]{\overset{\scriptstyle\underset{\mkern4mu\rotatebox{-90}{$\hspace{-3pt} #1$}}{#3}}{#2}}
\newcommand{\casess}[4]{\left\{ \begin{array}{ll} {#1} & {#2} \\ {#3} & {#4} \end{array} \right.}  % used to indicate to the reader that something is missing here


%Text:
\newcommand{\tx}[1]{\textrm{#1}}
\newcommand{\const}{\tx{const.}}

\newcommand{\ul}[1]{\underline{#1}}
\newcommand{\ol}[1]{\overline{#1}}
\newcommand{\ub}[1]{\underbrace{#1}}
\newcommand{\ob}[1]{\overbrace{#1}}

\newcommand{\hfw}{\color{RubineRed}\tx{ $\star$hier fehlt was$\star$ } \color{black}}  % used to indicate to the reader that something is missing here 
\newcommand{\hft}{\color{green!80!black}\tx{ $\star$hier fehlt eine Grafik$\star$ } \color{black}}  % used to indicate to the reader that a tikz picture is migging


%Spezielles:


%Theo:
\newcommand{\lag}{\mathcal{L}}  % used for Lagrange function
\newcommand{\ham}{\mathcal{H}}  % used for Hamiltonian function
\newcommand{\gre}{\mathcal{G}}  % used for Green's function
\newcommand{\ssf}{\mathcal{Y}}  % used for Spherical Surface Function (Kugelflächenfunktion)
\newcommand{\eofr}{\vec{E}(\vec{r})}
\newcommand{\pofr}{\Phi(\vec{r})}
\newcommand{\grr}{\mathcal G(\vec{r},\vec{r}')}
\newcommand{\vphi}{\varphi}
\newcommand{\vabla}{\vec{\nabla}}


%LA:
\newenvironment{bew}[1]{\subsection{Bew: #1}}{\hfill$\square$}
\newcommand{\Bew}[2]{\begin{bew}{#1}#2\end{bew}}
\newcommand{\enph}{F: V \to V \textrm{ Endomorphismus}}

\newcommand{\im}{\tx{im}}
\newcommand{\spa}{\tx{span}}
\newcommand{\adj}{\tx{adj}}
\newcommand{\grad}{\tx{grad}}
\newcommand{\ord}{\tx{ord}}

\newcommand{\basis}[3]{\{#1_{#2}, \dots, #1_{#3}\}}
\newcommand{\ska}[2]{\langle #1 , #2 \rangle}  % scalar product of input 1, and 2 can also be used for braket notation
\newcommand{\dmat}[3]{\begin{pmatrix} #1_{#2}&&\\ &\ddots& \\ && #1_{#3} \end{pmatrix}}


%Ex:
\newcommand{\kq}{\frac{1}{4\pi\epsilon_0}}  % writes out the whole constant k from electrostatic
\newcommand{\kqq}{\frac{\mu_0}{4\pi}}  % writes out the constant from magnetostatic
\newcommand{\uind}{U_{\tx{ind}}}
\newcommand{\folie}[1]{\color{gray}[Folie: #1]\color{black}}  % used to tell the reader that there was multimedia content during a lecture
\newcommand{\versuch}[1]{\color{red!50!black} \textbf{Versuch:} \color{black} \textbf{#1}\\ }  % used to tell the reader that there was a live experiment

\newcommand{\mau}{$\buildrel \mathcal{O} \over{\textbf{.}}$}  % the extreme Waldmann exclamation mark recreated in latex by Markus


% Lab commands:
\newcommand\mean{\begin{equation}
\frac{\sum_{i=1}^n x_i}{n}\label{mean}
\end{equation}}  % shortcut for the standard Mean function

\newcommand\meanstd{\begin{equation}
s_x=\sqrt{\frac{1}{n-1}\sum_{i=1}^n(x_i-\overline{x})^2}\label{meanstd}
\end{equation}}  % shortcut for the standard derivative mean function

\newcommand\prodquo{\begin{equation}\left\vert\frac{\Delta z}{z}\right\vert=\sqrt{\left(a\frac{\Delta x}{x}\right)^2+\left(b\frac{\Delta y}{y}\right)^2+\ldots}\textrm{ f\"ur }z=x^a\ y^b\ldots\end{equation}}

\newcommand\tfuncd{\begin{equation}
t=\frac{\vert x_n-y_n\vert}{\sqrt{x_s^2+y_s^2}}
\end{equation}}

\newcommand\tfunc{\begin{equation}
t=\frac{\vert x-y_0\vert}{u_x}
\end{equation}}


% ANDREZ
%\newcommand{\summ}[2]{\sum_{#1}^{#2}}
%\newcommand{\intt}[2]{\int_{#1}^{#2}}
\newcommand{\lcom}[1]{\color{MidnightBlue}#1\color{black}}  % used to indicate to the reader that this content was transcribed from the lecture
\newcommand{\bei}{\emph{Beispiel:}}
\newcommand{\bem}{\emph{Bemerkung:}}


% Boxen:

\tcbuselibrary{theorems}

% mahlt eine box nur um den text mit titel
\newtcbox{\fribox}[1]{nobeforeafter,colback=white,colframe=red!75!black,fonttitle=\bfseries,title=#1,sharp corners,tcbox raise base}

% mahlt eine große box um alles mit titel
\newcommand{\frbox}[2]{\begin{tcolorbox}[colback=white,colframe=red!75!black,fonttitle=\bfseries,title=#1]#2\end{tcolorbox}}

% mahlt eine box nur um den text
\newtcbox{\ribox}{nobeforeafter,colback=white,colframe=red!75!black,sharp corners,tcbox raise base}

% mahlt eine große box um alles was drinnen ist
\newcommand{\rbox}[1]{\begin{tcolorbox}[colback=white,colframe=red!75!black]#1\end{tcolorbox}}

% mahlt eine box um mathe innerhalb mathmode
\newcommand{\rmbox}[1]{\tcboxmath[colback=white,colframe=red!75!black]{#1}}

% super box (looks like regular boxed but wraps around anything)
\newenvironment{supbox}{\begin{tcolorbox}[colback=white,colframe=black,sharp corners,boxrule=.5pt]}{\end{tcolorbox}}

% the Big Black Box, can be used to separate examples or review material from the main text (used for Wiederholung)
\newcommand{\bbb}[2]{\begin{tcolorbox}[colback=white,colframe=black,fonttitle=\bfseries,title=#1,sharp corners,tcbox raise base]#2\end{tcolorbox}}

% array type box with title
\newenvironment{zebox}[1]{\begin{array}{|c|}
		\multicolumn{1}{l}{\tx{#1}} \\
		\hline
		\displaystyle
	}{\\ \hline
\end{array}}

% arrow list
\newlist{arrowlist}{itemize}{1}
\setlist[arrowlist]{label=$\Rightarrow$}


% optional:

\renewcommand{\vec}[1]{\bm{#1}}

% changed because of preferred looks
\renewcommand{\epsilon}{\varepsilon}
\renewcommand{\paragraph}[1]{\subsubsection{#1}}  % changed oddly behaving paragraphs to simple subsubsections which are not numbered nor in the toc

% nur in Theo benutzt !!!
% \renewcommand{\Phi}{\varPhi}

% evtl:
% \renewcommand{\boxed}{\rmbox}


% Tikz definitions:

\def\centerarc[#1](#2)(#3:#4:#5)% Syntax: [draw options] (center) (initial angle:final angle:radius)
{ \draw[#1] ($(#2)+({#5*cos(#3)},{#5*sin(#3)})$) arc (#3:#4:#5); }

\def\checkmark{\tikz\fill[scale=0.4](0,.35) -- (.25,0) -- (1,.7) -- (.25,.15) -- cycle;}  % checkmark used for proofs

\tikzset{
	annotated cuboid/.pic={
		\tikzset{%
			every edge quotes/.append style={midway, auto},
			/cuboid/.cd,
			#1
		}
		\draw [every edge/.append style={pic actions, densely dashed, opacity=.5}, pic actions]
		(0,0,0) coordinate (o) -- ++(-\cubescale*\cubex,0,0) coordinate (a) -- ++(0,-\cubescale*\cubey,0) coordinate (b) edge coordinate [pos=1] (g) ++(0,0,-\cubescale*\cubez)  -- ++(\cubescale*\cubex,0,0) coordinate (c) -- cycle
		(o) -- ++(0,0,-\cubescale*\cubez) coordinate (d) -- ++(0,-\cubescale*\cubey,0) coordinate (e) edge (g) -- (c) -- cycle
		(o) -- (a) -- ++(0,0,-\cubescale*\cubez) coordinate (f) edge (g) -- (d) -- cycle;
		\path [every edge/.append style={pic actions, |-|}]
		%(b) +(0,-5pt) coordinate (b1) edge ["\cubex \cubeunits"'] (b1 -| c)
		%(b) +(-5pt,0) coordinate (b2) edge ["\cubey \cubeunits"] (b2 |- a)
		%(c) +(3.5pt,-3.5pt) coordinate (c2) edge ["\cubez \cubeunits"'] ([xshift=3.5pt,yshift=-3.5pt]e)
		;
	},
	/cuboid/.search also={/tikz},
	/cuboid/.cd,
	width/.store in=\cubex,
	height/.store in=\cubey,
	depth/.store in=\cubez,
	units/.store in=\cubeunits,
	scale/.store in=\cubescale,
	width=10,
	height=10,
	depth=10,
	units=cm,
	scale=.1,
}


% other settings
\hbadness=99999  % removes unnecessary hbadness warnings

% the following are used to circumvent Roman numerals in the toc from running out of space
%\addtolength{\cftchapnumwidth}{10pt}
%\addtolength{\cftsecnumwidth}{10pt}
%\addtolength{\cftsubsecnumwidth}{10pt}
%\renewcommand{\thechapter}{\Roman{chapter}}  % just don't do this :)

% coloring
% for working at night
%\pagecolor{darkgray}
%\color{white}

\begin{document}

\title{
	{\Huge Experimentelle Methoden}\\[1em]
	{\Large Vorlesung von Prof. Dr. apl. Horst Fischer im Sommersemester 2019}}
\author{Markus Österle \hspace{5pt} Damian Lanzenstiel}
\date{ \today}
\maketitle
\tableofcontents

% Document

\setcounter{chapter}{-1}

\chapter{Einführung}

\section{Wichtige Infos}

\begin{description}
	\item[Vorlesung] Montag 14:15 - 15:45
	\item[Übungen] ILIAS
	\item[Kontakt] Horst Fischer Physikhochhaus Zi. 609\\
	\hfw (email usw. Folie 1)
\end{description}

\section{Programm der Vorlesung}

\begin{itemize}
	\item Grundlagen moderner Nachweissysteme
	\item Grundlagen der Statistik und Unsicherheitsbetrachtungen
	\item Grundlagen der Analogelektronik
\end{itemize}

\chapter{Wechselwirkung geladener Teilchen mit Materie}

Nachweis durch Wirkung des Teilchens auf die Materie

\begin{itemize}
	\item Ionisation, Szintillation
	\item \v{C}evenkov-, Übergangsstrahlung
	\item Rückstoß
\end{itemize}
$ \Rightarrow $ Teilcheneigenschaften verändert
\begin{itemize}
	\item Energieverlust
	\item Richtungsänderung
	\item Identitätsverlust
\end{itemize}

\section{Klassische Betrachtung der Rutherfordstreuung}

\begin{itemize}
	\item stimmt mit QM in niederster Ordnung überein\\[5pt]
	solange: ,,schwere Teilchen``\\
	$ v \gg v_{e \tx{ in Hülle}} $\\
	$ \Delta E \gg \tx{Bindungsenergie von } e^- $
\end{itemize}

%T1
\hft

Typisches Beispiel:
\begin{equation*}
\mu^+ + \tx{Atom} \to \mu^+ + (\tx{Atom} + e^-)
\end{equation*}
Coulomb-Kraft
\begin{equation*}
F_{\parallel}(x) = F_{\parallel}(-x)
\end{equation*}
\begin{equation*}
F_{\perp} = \frac{1}{4 \pi \epsilon_0} \frac{z \cdot e \cdot Z \cdot e}{r^2} \frac{b}{|\vec{r}|}
\end{equation*}
Impulsübertrag
\begin{equation*}
\Delta \rho_{T} = \int_{- \infty}^{\infty} F_{\perp} \dd f = \frac{e^2}{4 \pi \epsilon_0} \cdot \frac{2 Z \cdot z}{\beta c b}
\end{equation*}
$ \beta = \frac{v}{c} $
\lcom{Mehr zum Thema und die genaue Rechnung findet man im Lehrbuch von Jackson.}\\
\textbf{Energieübertrag}\\
\folie{Energieverlust: klassisch nach Bohr}
\begin{equation*}
\Delta E = \frac{\Delta \rho_{T}^2}{2 M} = \frac{e^4}{(4 \pi \epsilon_0)^2} \cdot \frac{Z^2 z^2}{M \beta^2 c^2 b^2} \propto \frac{1}{b^2}
\end{equation*}
bei Kohärenter Streuung
\begin{equation*}
\frac{\Delta E \tx{ Elektronenhülle}}{\Delta E \tx{ Kern}} = \frac{2 m_p}{m_e} \approx 4000
\end{equation*}
Hülle: $ M = Z \cdot m_e $\\
Kern: $ M = A \cdot m_p = 2 Z \cdot m_p $\\
\ribox{$ \Rightarrow $ Die Streuung am Kern ist vernachlässigbar}\\
Der gesamte (mittlere) Energieverlust ist dann:
\begin{align*}
\langle\dd E\rangle &= \int \Delta E \cdot \ub{2 \pi b \ \dd b}_{\mathclap{\tx{Volumenelement}}} \ \cdot Z \cdot \ub{\frac{\rho \cdot N_A}{A}}_{= n_e} \dd x
\end{align*}
\frbox{Bethe-Bloch Beziehung}{
\begin{align*}
\left\langle \frac{\dd E}{\dd x} \right\rangle &= D \cdot \ub{\frac{Z \cdot \rho}{A}}_{\mathclap{\tx{Medium}}} \ \cdot \ \ub{\left(\frac{z}{\beta}\right)^2}_{\mathclap{\tx{Projektil}}} \cdot \ub{\ln\left(\frac{b_{\tx{max}}}{b_{\tx{min}}}\right)}_{\frac{1}{2} \ln \left( \frac{2 m_e c^2 \gamma^2 \beta^2}{I} T_{\tx{max}}\right) } \\
&= D \cdot \ub{\frac{Z \cdot \rho}{A}}_{\mathclap{\tx{Medium}}} \ \cdot \ \ub{\left(\frac{z}{\beta}\right)^2}_{\mathclap{\tx{Projektil}}} \cdot \ \frac{1}{2} \ln \left( \frac{2 m_e c^2 \gamma^2 \beta^2}{I} T_{\tx{max}} \right)
\end{align*}}
\noindent
mit $ I = \hbar \omega $: Ionisationspotential des Streuzentrums\\
und $ T_{\tx{max}} $: der Energie des $ e^- $ tragen kann\\[5pt]
\folie{Energieverlust}\\
\folie{Mittlerer Energieverlust nach Bethe Bloch}\\
\folie{Relativistischer Anstieg}\\
\folie{Materialabhängigkeit des mittleren Energieverlusts}\\
\folie{Minimaler Energieverlust}\\
\folie{Abhängigkeit vom Ionisationspotential}\\
\folie{Reichweite von Teilchen in Materie}\\
\folie{Bragg-Kurve} (Einstrahl-Tiefe in einen Menschen)\\
\folie{Anwendung Teilchenidentifizierung}\\
\folie{Energieverlust von Teilchen durch Ionisation}

% 6.05.19

\section{Energieverlust von Elektronen \texorpdfstring{$ e^- $}{e-} und Positronen \texorpdfstring{$ e^+ $}{e+}}

Bremsstrahlung führt zu zusätzlichem Energieverlust.
\begin{equation*}
E_{K} \approx \frac{600 \dots 700}{Z} \, \tx{MeV} \qquad \textbf{kritische Energie}
\end{equation*}
$ Z $ des Materials. Unterschiede zwischen fest, flüssig, gasförmig.
\begin{center}
	\begin{tabular}{l|c}
		Material & $ E_K $ \\
		\hline
		Luft: & 84,0 MeV\\
		Pb: & 7,4 MeV
	\end{tabular}
\end{center}
\begin{equation*}
\prd{E}{x} \bigg|_{\tx{Brems}} \propto \frac{Z^2}{m^2} \ \ \begin{array}{l} \tx{Target} \\ \tx{Projektil} \end{array}
\end{equation*}
Bremsstrahlung wichtig für $ e^{\pm} $
\begin{equation*}
\frac{m_\mu^2}{m_e^2} \left(\frac{100}{0{,}5}\right)^2 = 40000
\end{equation*}
(Eigentlich 105 statt 100)\\[5pt]
Bremsstrahlung führt zu
\begin{align*}
\prd{E}{x} &= E_e \cdot 4 \alpha r_e^2 N_A \frac{\rho Z}{A} \left\{ \ln \frac{183}{Z^{\nicefrac{1}{3}}} + \frac{1}{18} - f(z) \right\}\\
f(z) &= \alpha Z \left\{  \frac{1}{1 + \alpha^2 Z^2} + 0.2 + \mathcal{O}(\alpha Z^2)\right\}
\end{align*}
$ \alpha $: gemessene Konstante $ \alpha = 5{,}3 $ für H $ \big| 3 $ Pb

\subsection{Strahlungslänge}

\begin{equation*}
\frac{1}{L_{\tx{rad}}} = 4 \alpha r_e^2 N_A \frac{\rho Z}{A} \left\{ \ln \frac{183}{Z^{\nicefrac{1}{3}}} + \frac{1}{18} - f(z) \right\} = \prd{E}{x} \cdot \frac{1}{E_e}
\end{equation*}
(Die Formale stammt von Bether Heitler).\\
Die Strahlungslänge ist die Distanz, in der die $ e^{\pm} $ den Bruchteil $ (1 - \nicefrac{1}{e}) $ der Energie durch Bremsstrahlung verlieren.
\begin{equation*}
\prd{E}{x} \bigg|_{\tx{Brems}} = \frac{E_e}{L_{\tx{rad}}}
\end{equation*}

%\bibliographystyle{plain}
%\bibliography{literature}
%\addcontentsline{toc}{section}{Literatur}

\end{document}
